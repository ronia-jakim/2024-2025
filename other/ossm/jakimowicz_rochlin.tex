\documentclass{../templatka_abstrakty/knmtabstrakt}

\usepackage[utf8]{inputenc}
\usepackage[polish]{babel}
\usepackage{microtype}

\title{Twierdzenie Rochlina}
\author{Weronika Jakimowicz}
\date{kwiecień 2025}

% \def\podziekowania{}

% \godzina{14:30}
% \sala{601}

\begin{document}

\maketitle 

Czterowymiarowe rozmaitości wydają się trudne do zrozumienia i badania. Okazuje się jednak, że jeśli ograniczymy się do rozważania zamkniętych 4-rozmaitości z dokładnością do kobordyzmu, możemy sporo zrozumieć patrząc na ich formę przecięcia $H_2(M;\mathbb{Z})\times H_2(M;\mathbb{Z})\to \mathbb{Z}$. W 1952 roku Vladimir Rokhlin udowodnił, że sygnatura orientowalnych, zamkniętych 4-rozmaitości jest podzielna przez $16$. W moim referacie postaram się przybliżyć zagadnienia związane z formą przecięcia na rozmaitości 4 wymiarowej oraz podać zarys geometrycznego dowodu twierdzenia Rochlina przedstawiony przez Matsumoto w 1986. 

\end{document}
