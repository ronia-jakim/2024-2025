%! TeX program = lualatex
\documentclass{article}

\usepackage[utf8]{inputenc}
\usepackage[T1]{fontenc}

\usepackage{fontspec}

\usepackage[default]{cantarell}


\usepackage[
  a4paper,
  total={170mm, 247mm},
  left=20mm,
  top=25mm,
  % showframe
]{geometry}

\usepackage[skip=3mm]{parskip}
\renewcommand{\baselinestretch}{1.2}
\selectfont


\usepackage{amsfonts}
\usepackage{amsmath}
\usepackage{amssymb}

\usepackage{mathastext}

\DeclareMathOperator{\Spec}{Spec}

\usepackage{xcolor}

\definecolor{rosewater}{HTML}{dc8a78}
\definecolor{flamingo}{HTML}{dd7878}
\definecolor{pink}{HTML}{ea76cb}
\definecolor{mauve}{HTML}{8839ef}
\definecolor{red}{HTML}{d20f39}
\definecolor{maroon}{HTML}{e64553}
\definecolor{peach}{HTML}{fe640b}
\definecolor{yellow}{HTML}{df8e1d}
\definecolor{green}{HTML}{40a02b}
\definecolor{teal}{HTML}{179299}
\definecolor{sky}{HTML}{04a5e5}
\definecolor{sapphire}{HTML}{209fb5}
\definecolor{blue}{HTML}{1e66f5}
\definecolor{lavender}{HTML}{7287fd}
\definecolor{subtext2}{HTML}{4c4f69}
\definecolor{subtext1}{HTML}{5c5f77}
\definecolor{subtext0}{HTML}{6c6f85}
\definecolor{overlay2}{HTML}{7c7f93}
\definecolor{overlay1}{HTML}{8c8fa1}
\definecolor{overlay0}{HTML}{9ca0b0}

\definecolor{text}{HTML}{000000}
\color{text}

\makeatletter 
\let\htitle\@title
\let\fauthor\@author
\makeatother

\usepackage{fancyhdr}

\fancyfoot[CE, CO]{}
\fancyfoot[LE, RO]{\color{overlay2}\thepage}

\fancyhead[RE, LO]{$ \quad $\color{subtext1}\fauthor $ \quad $}
\fancyhead[LE, RO]{$ \quad $\color{green!40!subtext2}\bfseries\htitle $ \quad $}

\fancypagestyle{plain}{%
  \fancyhf{}%
  \fancyfoot[CE, CO]{}
  \fancyfoot[LE, RO]{\color{overlay2}\thepage}

  \fancyhead[RE, LO]{$ \quad $\color{subtext1}\fauthor $ \quad $}
  \fancyhead[LE, RO]{$ \quad $\color{green!40!subtext2}\bfseries\htitle $ \quad $}
}
\usepackage{dashrule}

\renewcommand{\headrule}{\color{overlay0}\hdashrule[.5ex]{\headwidth}{1pt}{1pt}}

\pagestyle{fancy}

\usepackage{titlesec}

\titleformat{\chapter}[block]{\bfseries\Huge}{\color{green!40!subtext2}\filright\Huge\thechapter.}{1ex}{\color{green!40!subtext2}\Huge\filright}

\titlespacing*{\chapter}{0pt}{0pt}{20pt}

\usepackage{amsthm}
\usepackage{thmtools}
\usepackage{tcolorbox}

% \RequirePackage[framemethod=TikZ]{mdframed}

\tcbuselibrary{theorems}
\tcbuselibrary{breakable}
\tcbuselibrary{skins}

\tcbset{greenTHM/.style={
    fonttitle = \bfseries\large,
    coltitle = green!50!black,
    description color = green!50!black, 
    description font = \bfseries\large,
    colbacktitle = white, %green!40!black!75,
    breakable, 
    enhanced,
    attach boxed title to top left = {
      xshift=0.5cm, 
      yshift=-\tcboxedtitleheight/2
    },
    boxed title style = {
      boxrule=0pt,
      colframe=white
    },
    top=6mm,
    bottom=4mm,
    colback=white,
    frame hidden,
    borderline={1pt}{0pt}{green!40},
    arc=0mm,
    after skip=5mm,
    before skip=5mm
  }
}

\tcbset{orangeTHM/.style={
    fonttitle = \bfseries\large,
    coltitle = orange!60!black,
    description color = orange!60!black, 
    description font = \bfseries\large,
    colbacktitle = white, %green!40!black!75,
    breakable, 
    enhanced,
    attach boxed title to top left = {
      xshift=0.5cm, 
      yshift=-\tcboxedtitleheight/2
    },
    boxed title style = {
      boxrule=0pt,
      colframe=white
    },
    top=6mm,
    bottom=4mm,
    colback=white,
    frame hidden,
    borderline={1pt}{0pt}{orange!40}, 
    arc=0mm,
    before skip=5mm,
    after skip=5mm
  }
}

\NewTcbTheorem[
  auto counter, 
  % number within=chapter
  number within=section
]{definition}{Definition}{greenTHM}{def} 

\NewTcbTheorem[
  use counter from=definition
]{theorem}{Theorem}{orangeTHM}{th}

\NewTcbTheorem[
  use counter from=definition
]{proposition}{Proposition}{orangeTHM}{prop}


\usepackage{svg}

\renewenvironment{proof}{{\bfseries\color{green!60!black} Dowód}$ $\newline}{
  \begin{flushright} \includesvg[width=4mm]{../../../../duck.svg} \end{flushright}$ $\newline
}

\usepackage{soul}

\sethlcolor{green!15}

\makeatletter
%\font\SOUL@tt="LMMono10-Regular"
\setbox\z@\hbox{\SOUL@tt-}
\SOUL@ttwidth\wd\z@ %
\makeatother

\newcommand{\buff}[1]{
  {\bfseries\color{green}#1}
}

\usepackage{enumitem}


\title{Kombinatoryka}
\date{17.01.2025}
\author{Weronika Jakimowicz}

\begin{document}
\fontsize{12pt}{18pt}\selectfont

\maketitle\thispagestyle{empty}

\begin{zadanie}
  Niech $p_n=\binom{2n}{n}2^{-2n}$ oraz $s_n=\sum_{k=0}^n p_k$. Pokaż, że $s_n=(2n+1)p_n$. Który z ciągów, $\frac{s^2_n}{n}$ czy $\frac{s_n^2}{n+1}$, jest rosnący, a który malejący?
\end{zadanie}

\rozwiazanie{
  $$p_0=1$$
  $$p_1=\frac{1}{2}$$
  $$p_2=\frac{2}{3}$$
  $$s_0=p_0=1=(2\cdot 0+1)p_0$$
  $$s_1=p_0+p_1=\frac{1}{2}+1=\frac{3}{2}=(2+1)p_1$$

  Indukcja? 
  \begin{align*}
    s_{n+1}&=\sum_{k=0}^{n+1}p_k=\sum_{k=0}^np_k+p_{n+1}=\\ 
           &=(2n+1)p_n+p_{n+1}=(2n+1)\binom{2n}{n}2^{-2n}+\binom{2n+2}{n+1}2^{-(2n+1)}=\\ 
           &=2(n+1)\frac{(2n+2)(2n+1)}{2\cdot2(n+1)(n+1)}\binom{2n}{n}2^{-2n}+\binom{2n+2}{n+1}2^{-2(n+1)}=\\ 
           &=2(n+1)\binom{2n+2}{n+1}2^{-2(n+1)}+\binom{2n+2}{n+1}2^{-2(n+1)}
  \end{align*}

  Ten z $n$ w mianowniku rośnie, a z $(n+1)$ - maleje.
}

\begin{zadanie}
  Niech
  $$P_n=\left\{(a_i)_{i=1}^{2n}\;:\;a_i=\pm1,\;\sum_{i=1}^{2n}a_i=0\right\}$$
  Oblicz $|P_n|$
\end{zadanie}

\rozwiazanie{To nie poludzku napisana ilość ciągów długości $2n$ które sumują się do $0$. Czyli $\binom{2n}{n}$.}

\begin{zadanie}
  Niech $Z$ będzie zbiorem o $n$ elementach. Na ile sposobów można wybrać $A\subseteq B\subseteq Z$? Zakładamy, że każdy zbiór zawiera siebie i zbiór pusty.
\end{zadanie}

\rozwiazanie{Wybieramy najpierw $k$-elementowy $A$ na $\binom{n}{k}$ sposobów, a potem z reszty na $2^{n-k}$ dobieramy elementy do $B$. Całość to suma iloczynu i wychodzi ostatecznie $3^n$.}

\begin{zadanie}
  Pokaż, że $2^{2n}=\sum_{k=0}^n\binom{2n}{k}\binom{2(n-k)}{n-k}.$
\end{zadanie}
\rozwiazanie{Pokaż, że dla każdego ciągu $(c_i)_{i=1}^{2n}$ istnieje jedyne takie $0\leq k\leq n$, że $(c_i)_{i=1}^{2k}$ jest ciągiem $\pm1$ sumującym się do $0$ oraz $(c_{i-2k}^{2(n-k)}_{i=1}$ nie ma zerującego się podciągu}

\begin{zadanie}
  Na płaszczyźnie dany jest skończony zbiór punktów $Z$ o tej własności, że żadne dwie odległości punktów zbioru $Z$ nie są równe. Punkty $A$ i $B$ należące do $Z$ łączymy wtedy i tylko wtedy, gdy $A$ jest punktem najbliższym $B$ lub $B$ jest punktem najbliższym $A$. Udowodnić, że żaden punkt zbioru $Z$ nie będzie połączony z więcej niż pięcioma innymi.
\end{zadanie}

\rozwiazanie{
  Kąt między dwoma takimi ziomkami jest większy niż $60^\circ$.
}

\begin{zadanie}
  Płaszczyznę pokryto kołami o jednakowym promieniu w ten sposób, że środek każdego z tych kół nie należy do żadnego innego koła. Dowieść, że każdy punkt płaszczyzny należy do co najwyżej pięciu kół.
\end{zadanie}

\rozwiazanie{
  Nie wprost - zakładamy, że punkt $A$ leży w kołach o środkach $O_1$, $O_2$, ..., $O_6$. Wtedy jeden z kątów $O_iAO_j$ ma nie więcej niż $60^\circ$, bo jakieś sześć się dodaje do $360$ - jak po równo rozłożymy, to mamy po $60$ na każdy.

  Czyli mamy trójkąt $O_iAO_j$ w którym odcinek $O_iO_j$ nie jest najdłuższy (bo kąt naprzeciwko $\leq60$). Czyli BSO $O_iA\geq O_iO_j$ i mamy, że $O_j$ jest w okręgu o środku $O_i$.
}

\begin{zadanie}
  Żaba skacze po stawie, na którym pływa $8$ liści ułożonych w okrąg. Na ile sposobów może przeskoczyć na najbardziej odległy od siebie liść w $2n$ skokach?
\end{zadanie}

\rozwiazanie{
  Tak naprawdę możemy patrzeć na $n$ kroków i albo stoimy w miejscu, albo przemieszczamy o dwa (to sensowny krok). Rysujemy wtedy czworokąt i mamy cztery równania rekurencyjne, na ile sposobów możemy w $n$ kroków dostać się na każdy z tych czterech wierzchołków.
}

\begin{zadanie}
  Udowodnić, że dla liczby naturalnej $n$ większej od $1$ następujące warunki są równoważne:
  \begin{enumerate}[label=\alph*)]
    \item $n$ jest liczbą parzystą
    \item istnieje permutacja $(a_0, a_1, a_2,..., a_{n-1})$ zbioru $\{0,1,...,n-1\}$ o tej własności, że ciąg reszt z dzielenia przez $n$ liczb $a_0$, $a_0+a_1$, $a_0+a_1+a_2$,..., $a_0+a_1+...+a_{n-1}$ jest też permutacją tego zbioru.
  \end{enumerate}
\end{zadanie}

\rozwiazanie{
  $a\implies b$
  $$a_i=\begin{cases}i&i\text{ nieparzyste}\\ n-i & i\text{ parzyste}\\0&i=0\end{cases}$$

  \begin{align*}
    \sum_{k=0}^{2i} a_k&=a_1+a_3+...+a_{2i-1}+a_2+a_4+...+a_{2i}=\\ 
                       &= 1+3+...+(2i-1)+(n-2)+(n-4)+...+(n-2i)=\\ 
                       &= 2+4+...+2i-i+n\cdot i-2-4-...-2i=i(n-1)
  \end{align*}

  \begin{align*}
    \sum_{k=0}^{2i+1}a_k&=a_1+a_3+...+a_{2i+1}+a_2+a_4+...+a_{2i}=\\ 
                        &= 1+3+...+(2i+1)+(n-2)+(n-4)+...+(n-2i)=\\ 
                        &=2+4+...+2i+(2i+1)-i+n\cdot i-2-4-...-2i=\\ 
                        &=i(n+1)+1
  \end{align*}

  $b\implies a$

  Po pierwsze zauważamy, że musi być $a_0=0$, bo inaczej $\sum_{k=0}^ia_k=\sum_{k=0}^{i+1}a_k$ (jeśli $a_{i+1}=0$).

  Dalej wiemy, że żadna z sum $\sum_{k=0}^ia_k$ nie może być podzielna przez $n$. Czyli w szczególności 
  $$\sum_{k=0}^{n-1}a_k=\sum_{k=0}^{n-1}k=\frac{(n-1)n}{2}$$
  nie może być podzielne przez $n$, czyli $\frac{n-1}{2}$ nie jest całkowite.
}

\begin{zadanie}
  Ile najwięcej kawałków sera można uzyskać z pojedynczego grubego kawałka za pomocą $n$ cięć nożem? Zakładamy, że każde cięcie jest wyznaczone przez płaszczyznę przecinającą kawałek sera.
\end{zadanie}

\rozwiazanie{
  Zaczynamy od rozwiązania 2D. 
  $$L_n=\text{największa liczba obszarów na które }n\text{ prostych dzieli płaszczyznę}$$
  Rysując $n$-tą prostą przetniemy $(n-1)$ prostych, które do tej pory narysowaliśmy. W ten sposób tworzymy $n$ nowych obszarów. Czyli mamy 
  $$L_n=L_{n-1}+n=n+(n-1)+L_{n-2}=...=n+(n-1)+...+1+L_0=\frac{n(n+1)}{2}+1$$
  
  Teraz przechodzimy do 3D i oznaczamy zależność $P_n$. $n$-ta płaszczyzna przecina poprzednie $(n-1)$ i tym samym tworzy $(n-1)$ prostych. Nad każdym regionem płaszczyzny wisi nowy obszar przestrzeni 3D, czyli relacja to 
  $$P_n=P_{n-1}+L_{n-1}.$$
  Jak rozwiążemy tę rekurencję, to dostajemy $\frac{n^3+5n+6}{6}$.
}


\begin{zadanie}
  W turnieju szachowym uczestniczy $2n$ zawodników, przy czym każdych dwóch spośród nich rozgrywa między sobą co najwyżej jedną partię. Dowieść, że taki przebieg rozgrywek, w którym żadna trójka uczestników nie rozgrywa trzech partii między sobą jest możliwy wtedy i tylko wtedy, gdy liczba wszystkich partii rozegranych w turnieju nie przekracza $n^2$.
\end{zadanie}

\rozwiazanie{
  Żadna trójka uczestników $\iff$ liczba wszystkich partii nie przekracza $n^2$.

  $\impliedby$ Dzielimy pysiów na dwa rozłączne zbiory, wtedy mamy $n^2$ parowań między tymi zbiorami. Wtedy każdy gra co najwyżej jedną rundę ze sobą i nie ma trójek.

  $\implies$ indukcja po $n$. Gdy $n=2$ to mamy $4$ uczestników. Pierwszy mecz to dowolne pary, przegrani grają ze sobą tak samo jak wygrani. Mamy $4$ rozgrywki i śmiga.

  Mamy teraz $2(n+1)$ uczestników. Dwóch, $X$ oraz $Y$, oddzielamy i pozostałych $2n$ potrafimy rozegrać tak jak trzeba, czyli mamy nie więcej niż $n^2$ meczów. Oczywiście, $X$ gra z $Y$. Gdyby ilość meczów rozegranych przez $X$ oraz $Y$ z pozostałymi przekraczała $2n$, to wtedy oboje zagraliby z kimś razem. 
}

\begin{zadanie}
  Każdemu wierzchołkowi sześcianu przyporządkowano liczbę $1$ lub $-1$, a każdej ścianie - iloczyn liczb przyporządkowanych wierzchołkom tej ściany. Wyznaczyć zbiór wartości, które może przyjąć suma $14$ liczb przyporządkowanych ścianom i wierzchołkom.
\end{zadanie}

\rozwiazanie{tutaj dzbanie https://archom.ptm.org.pl/?q=node/554 }

\begin{zadanie}
  Ile jest połączeń w pary wierzchołków wypukłego $2k$-kąta tak, by odpowiadające mu przekątne (lub boki) nie przecinały się.
\end{zadanie}

\rozwiazanie{
  to liczba catalana jest
  $$C_k=\binom{2k}{k}-\binom{2k}{k-1}$$
  czy też 
  $$C_k=\sum_i\binom{k}{2i}C_i2^{n-2i}$$
  $$C_k=\sum_{i=0}^{k-1}C_iC_{k-i}$$
  wybieramy wierzchołek, łączymy go z czymś i liczymy na lewo i na prawo od tej przekątnej.
}

\begin{zadanie}
  Niech $n=p_1^{n_1}p_2^{n_2}\cdot\dots\cdot p_s^{n_2}$, $\phi$ będzie funkcją Eulera i 
  $$\psi(n)=\text{lcm}(\phi(p_1^{n_1}),\phi(p_2^{n_2}), ...,\phi(p_s^{n_2})).$$
  Udowodnij, że dla $a$ względnie pierwszego z $n$ zachodzi $n|a^{\psi(n)}-1$.
\end{zadanie}

\rozwiazanie{
  Mamy $\phi(p_i^{n_i})=p_i^{n_i}-p_i^{n_i-1}$, bo wykluczamy wszystkie możliwe wartości gcd. I teraz pewnie chcemy wykorzystać fakt, że jeśli $a$ jest względnie pierwsze z $n$, to $a^{\phi(n)}-1$ jest podzielne przez $n$. Jest też fakt, że $\phi(mn)=\phi(m)\phi(n)\cdot{gcd}{\phi(gcd)}$, które się z kolei udowadnia $\phi(n)=n\prod (1-\frac{1}{p_i})$ dla $p_i$ w rozkładzie $n$.
}

\begin{zadanie}
  Na polach szachownicy $n\times n$ rozmieszczono $n^2$ różnych liczb całkowitych, po jednej na każdym polu. W każdej kolumnie pole z największą liczbą pomalowano na czerwono. Zbiór $n$ pól szachownicy nazwiemy dopuszczalnym, jeżeli żadne dwa z tych pól nie znajdują się w tym samum wierszu ani w tej samej kolumnie. Spośród wszystkich zbiorów dopuszczalnych wybrano zbiór, dla którego suma liczb umieszczonych na jego polach jest największa. Wykazać, że w tak wybranym zbiorze jest czerwone pole.
\end{zadanie}

\rozwiazanie{nie chce mi sie pisac https://archom.ptm.org.pl/?q=node/391 }

\begin{zadanie}
  Dane są karty $3$ pola na $3$. W każdym z pól możemy zrobić dziurkę. Karty są na tyle symetryczne, że możemy je obracać wokół środka i odwracać na drugą stronę nie wiedząc potem w jakiej pozycji były one na początku. Pokaż, że istnieje $8$ rozróżnialnych kart $3\times 3$ z dwoma dziurkami. Narysuj te karty.
\end{zadanie}

\rozwiazanie{
  Można to zrobić Korzystając z liczby orbit etc. Tzn. $|X/G|=\frac{1}{|G|}\sum_{g\in G}|X^g|$ czyli ilość orbit to $\frac{1}{|G|}$ suma ilości elementów niezmienniczych na każdy z elementów grupy.
}

\end{document}
