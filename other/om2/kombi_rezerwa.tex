%! TeX program = lualatex
\documentclass{article}

\usepackage[utf8]{inputenc}
\usepackage[T1]{fontenc}

\usepackage{fontspec}

\usepackage[default]{cantarell}


\usepackage[
  a4paper,
  total={170mm, 247mm},
  left=20mm,
  top=25mm,
  % showframe
]{geometry}

\usepackage[skip=3mm]{parskip}
\renewcommand{\baselinestretch}{1.2}
\selectfont


\usepackage{amsfonts}
\usepackage{amsmath}
\usepackage{amssymb}

\usepackage{mathastext}

\DeclareMathOperator{\Spec}{Spec}

\usepackage{xcolor}

\definecolor{rosewater}{HTML}{dc8a78}
\definecolor{flamingo}{HTML}{dd7878}
\definecolor{pink}{HTML}{ea76cb}
\definecolor{mauve}{HTML}{8839ef}
\definecolor{red}{HTML}{d20f39}
\definecolor{maroon}{HTML}{e64553}
\definecolor{peach}{HTML}{fe640b}
\definecolor{yellow}{HTML}{df8e1d}
\definecolor{green}{HTML}{40a02b}
\definecolor{teal}{HTML}{179299}
\definecolor{sky}{HTML}{04a5e5}
\definecolor{sapphire}{HTML}{209fb5}
\definecolor{blue}{HTML}{1e66f5}
\definecolor{lavender}{HTML}{7287fd}
\definecolor{subtext2}{HTML}{4c4f69}
\definecolor{subtext1}{HTML}{5c5f77}
\definecolor{subtext0}{HTML}{6c6f85}
\definecolor{overlay2}{HTML}{7c7f93}
\definecolor{overlay1}{HTML}{8c8fa1}
\definecolor{overlay0}{HTML}{9ca0b0}

\definecolor{text}{HTML}{000000}
\color{text}

\makeatletter 
\let\htitle\@title
\let\fauthor\@author
\makeatother

\usepackage{fancyhdr}

\fancyfoot[CE, CO]{}
\fancyfoot[LE, RO]{\color{overlay2}\thepage}

\fancyhead[RE, LO]{$ \quad $\color{subtext1}\fauthor $ \quad $}
\fancyhead[LE, RO]{$ \quad $\color{green!40!subtext2}\bfseries\htitle $ \quad $}

\fancypagestyle{plain}{%
  \fancyhf{}%
  \fancyfoot[CE, CO]{}
  \fancyfoot[LE, RO]{\color{overlay2}\thepage}

  \fancyhead[RE, LO]{$ \quad $\color{subtext1}\fauthor $ \quad $}
  \fancyhead[LE, RO]{$ \quad $\color{green!40!subtext2}\bfseries\htitle $ \quad $}
}
\usepackage{dashrule}

\renewcommand{\headrule}{\color{overlay0}\hdashrule[.5ex]{\headwidth}{1pt}{1pt}}

\pagestyle{fancy}

\usepackage{titlesec}

\titleformat{\chapter}[block]{\bfseries\Huge}{\color{green!40!subtext2}\filright\Huge\thechapter.}{1ex}{\color{green!40!subtext2}\Huge\filright}

\titlespacing*{\chapter}{0pt}{0pt}{20pt}

\usepackage{amsthm}
\usepackage{thmtools}
\usepackage{tcolorbox}

% \RequirePackage[framemethod=TikZ]{mdframed}

\tcbuselibrary{theorems}
\tcbuselibrary{breakable}
\tcbuselibrary{skins}

\tcbset{greenTHM/.style={
    fonttitle = \bfseries\large,
    coltitle = green!50!black,
    description color = green!50!black, 
    description font = \bfseries\large,
    colbacktitle = white, %green!40!black!75,
    breakable, 
    enhanced,
    attach boxed title to top left = {
      xshift=0.5cm, 
      yshift=-\tcboxedtitleheight/2
    },
    boxed title style = {
      boxrule=0pt,
      colframe=white
    },
    top=6mm,
    bottom=4mm,
    colback=white,
    frame hidden,
    borderline={1pt}{0pt}{green!40},
    arc=0mm,
    after skip=5mm,
    before skip=5mm
  }
}

\tcbset{orangeTHM/.style={
    fonttitle = \bfseries\large,
    coltitle = orange!60!black,
    description color = orange!60!black, 
    description font = \bfseries\large,
    colbacktitle = white, %green!40!black!75,
    breakable, 
    enhanced,
    attach boxed title to top left = {
      xshift=0.5cm, 
      yshift=-\tcboxedtitleheight/2
    },
    boxed title style = {
      boxrule=0pt,
      colframe=white
    },
    top=6mm,
    bottom=4mm,
    colback=white,
    frame hidden,
    borderline={1pt}{0pt}{orange!40}, 
    arc=0mm,
    before skip=5mm,
    after skip=5mm
  }
}

\NewTcbTheorem[
  auto counter, 
  % number within=chapter
  number within=section
]{definition}{Definition}{greenTHM}{def} 

\NewTcbTheorem[
  use counter from=definition
]{theorem}{Theorem}{orangeTHM}{th}

\NewTcbTheorem[
  use counter from=definition
]{proposition}{Proposition}{orangeTHM}{prop}


\usepackage{svg}

\renewenvironment{proof}{{\bfseries\color{green!60!black} Dowód}$ $\newline}{
  \begin{flushright} \includesvg[width=4mm]{../../../../duck.svg} \end{flushright}$ $\newline
}

\usepackage{soul}

\sethlcolor{green!15}

\makeatletter
%\font\SOUL@tt="LMMono10-Regular"
\setbox\z@\hbox{\SOUL@tt-}
\SOUL@ttwidth\wd\z@ %
\makeatother

\newcommand{\buff}[1]{
  {\bfseries\color{green}#1}
}

\usepackage{enumitem}


\title{Kombinatoryczna rezerwa}
\date{17.01.2025}
\author{Weronika Jakimowicz}

\begin{document}
\fontsize{12pt}{18pt}\selectfont

\maketitle\thispagestyle{empty}

\begin{zadanie}
  Karty są ponumerowane od $1$ do $n$ i ułożone w rzędzie długości $n\geq 5$. W jednym kroku możemy wybrać dowolny blok kolejnych kart, których liczby są w rosnącym lub malejącym porządku i obrócić ten blok Np. dla $n=9$ ułożenie 9 1 \underline{6 5 3} 2 7 4 8 może zostać zamienione w 9 1 \underline{3 5 6} 2 7 4 8. Pokaż, że w co najwyżej $2n-6$ ruchach można ułożyć te $n$ kart tak, że ich liczby są ustawione rosnąco lub malejąco.
\end{zadanie}

\rozwiazanie{
  Indukcja? Załóżmy, że mamy permutację zaczynającą się od $k$. W $2(n-1)+6$ krokach umiemy ją ustawić w (k, 1, 2, ..., n) lub (k, n, n-1, n-2, ..., 1). Pierwszy z nich potrafimy w jednym kroku zmienić w (k, k-1, k-2, ..., 1, k+1, k+2, ..., n) a drugi w (k, k+1, k+2, ..., n-1, n, k-1, ..., 1)

  Jeszcze jeden krok naprawia porządek, czyli mamy $2(n-1)+6+2$ i wystarczy pokazać przypadek bazowy, czyli $n=5$.
}

\begin{zadanie}
  $n$ uczniów jest ustawionych w okręgu. Ich wysokości to $h_1<h_2<...<h_n$, numerowanych zgodnie z ruchem wskazówek zegara. Jeśli student wzrostu $h_k$ stoi bezpośrednio za studentem wzrostu $h_{k-2}$ lub mniej, to mogą oni zamienić się miejscami. Pokaż, że takich zamian możliwych jest co najwyżej $\binom{n}{3}$.
\end{zadanie}

\rozwiazanie{
  Niech $s_k$ będzie liczbą możliwych przestawień student wysokości $k$ zmieni się z kimś krótszym. Oczywiście $s_1=s_2=0$. Myślimy teraz ile jest osób między $k$-tym a $k-1$ studentem wzrostu nie więcej niż $h_{k-2}$. Oczywiście jest tego $(k-2)$ sztuki (jeśli wszyscy tam stoją). Ta liczba zmniejsza się o $1$ przy każdej zmianie studenta $k$ i wzrasta, gdy $k-1$ uczeń się zmienia. Jeśli ktoś wyższy od $k$ lub $k-1$ się zmieni z nimi, to ta liczba osób krótszych się nie zmienia.

  Stąd $s_k-s_{k-1}$ jest ilością studentów krótszych od $k-2$ między $k$tym a $k-1$szym. Stąd $s_k-s_{k-1}\leq k-2$. Można pokazać, że $s_k\leq \binom{k-1}{2}$ i jak to zsumujemy to dostajemy $\binom{n}{3}$ (trójkąt pascala).
}

\end{document}
