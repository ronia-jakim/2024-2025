%! TeX program = lualatex
\documentclass{article}

\usepackage[T1]{fontenc}

\usepackage{fontspec}
\usepackage[default]{cantarell}

\usepackage[nolessnomore,noequal,noparenthesis,nospecials,noplusnominus,noexclam,italic]{mathastext}

\usepackage{amsfonts}
\usepackage{amsmath}
\usepackage{amssymb}

\usepackage{array}


\usepackage{ntheorem}

\theorembodyfont{\normalfont}
\theoremseparator{.\hspace{0.5em}}
\theorempreskip{}
\theorempostskip{}

\theorempostwork{\vspace*{4mm}}

\newtheorem{zadanie}{Zadanie}


\pagestyle{empty}

\usepackage{tikz}
\usetikzlibrary{calc}

\usepackage[a4paper, total={170mm, 237mm}]{geometry}

\usepackage{enumitem}

\newcommand{\rozwiazanie}[1]{\slshape }%#1}

\makeatletter
\def\@maketitle{
  \begin{center}
    {\LARGE\@title}
    \medskip 

    \large\@author
  \end{center}
}
\makeatother



\title{Kombinatoryczna rezerwa}
\date{17.01.2025}
\author{Weronika Jakimowicz}

\begin{document}
\fontsize{12pt}{18pt}\selectfont

\maketitle\thispagestyle{empty}

\begin{zadanie}
  Karty są ponumerowane od $1$ do $n$ i ułożone w rzędzie długości $n\geq 5$. W jednym kroku możemy wybrać dowolny blok kolejnych kart, których liczby są w rosnącym lub malejącym porządku i obrócić ten blok Np. dla $n=9$ ułożenie 9 1 \underline{6 5 3} 2 7 4 8 może zostać zamienione w 9 1 \underline{3 5 6} 2 7 4 8. Pokaż, że w co najwyżej $2n-6$ ruchach można ułożyć te $n$ kart tak, że ich liczby są ustawione rosnąco lub malejąco.
\end{zadanie}

\rozwiazanie{
  Indukcja? Załóżmy, że mamy permutację zaczynającą się od $k$. W $2(n-1)+6$ krokach umiemy ją ustawić w (k, 1, 2, ..., n) lub (k, n, n-1, n-2, ..., 1). Pierwszy z nich potrafimy w jednym kroku zmienić w (k, k-1, k-2, ..., 1, k+1, k+2, ..., n) a drugi w (k, k+1, k+2, ..., n-1, n, k-1, ..., 1)

  Jeszcze jeden krok naprawia porządek, czyli mamy $2(n-1)+6+2$ i wystarczy pokazać przypadek bazowy, czyli $n=5$.
}

\begin{zadanie}
  $n$ uczniów jest ustawionych w okręgu. Ich wysokości to $h_1<h_2<...<h_n$, numerowanych zgodnie z ruchem wskazówek zegara. Jeśli student wzrostu $h_k$ stoi bezpośrednio za studentem wzrostu $h_{k-2}$ lub mniej, to mogą oni zamienić się miejscami. Pokaż, że takich zamian możliwych jest co najwyżej $\binom{n}{3}$.
\end{zadanie}

\rozwiazanie{
  Niech $s_k$ będzie liczbą możliwych przestawień student wysokości $k$ zmieni się z kimś krótszym. Oczywiście $s_1=s_2=0$. Myślimy teraz ile jest osób między $k$-tym a $k-1$ studentem wzrostu nie więcej niż $h_{k-2}$. Oczywiście jest tego $(k-2)$ sztuki (jeśli wszyscy tam stoją). Ta liczba zmniejsza się o $1$ przy każdej zmianie studenta $k$ i wzrasta, gdy $k-1$ uczeń się zmienia. Jeśli ktoś wyższy od $k$ lub $k-1$ się zmieni z nimi, to ta liczba osób krótszych się nie zmienia.

  Stąd $s_k-s_{k-1}$ jest ilością studentów krótszych od $k-2$ między $k$tym a $k-1$szym. Stąd $s_k-s_{k-1}\leq k-2$. Można pokazać, że $s_k\leq \binom{k-1}{2}$ i jak to zsumujemy to dostajemy $\binom{n}{3}$ (trójkąt pascala).
}

\end{document}
