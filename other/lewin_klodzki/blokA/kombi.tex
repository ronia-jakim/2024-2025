%! TeX program = lualatex
\documentclass{uoom}

\autor{Weronika Jakimowicz}{dupa}
\temat{Kombinatoryka (i grafy)}

\theoremstyle{definition}
\newtheorem{zadd}{Zadanie}

\newenvironment{zadanie}{
  \begin{zadd}
  }
  {
    \end{zadd}\vspace{\ooodstep}
  }

\renewcommand{\stopka}{
    \blfootnote{Materiały przygotowane przez \aaautor.}
  }

\RequirePackage{tcolorbox}

\tcbuselibrary{theorems}
\tcbuselibrary{breakable}
\tcbuselibrary{skins}

  \NewTColorBox{mybox}{ O{red} m d"" !O{} }
{
  enhanced,
  fonttitle = \bfseries\large,
  coltitle = green!50!black,
  description color = green!50!black, 
  description font = \bfseries\large,
  colbacktitle = white, %green!40!black!75,
  breakable, 
  enhanced,
  attach boxed title to top left = {
    xshift=0.5cm, 
    yshift=-\tcboxedtitleheight/2
  },
  boxed title style = {
    boxrule=0pt,
    colframe=white
  },
  top=6mm,
  bottom=4mm,
  colback=white,
  frame hidden,
  borderline={1pt}{0pt}{green!40},
  arc=0mm,
  after skip=5mm,
  before skip=5mm,
  fonttitle=\bfseries,
  title={#2}
}

\usepackage{xcolor}
\definecolor{green}{HTML}{40a02b}

\usepackage{tikz-cd}

\begin{document}

\noindent\makebox[\linewidth]{\rule{\textwidth}{0.4pt}}

\noindent\textbf{Rozgrzewka}

\begin{zadanie}
  Ile jest przestawień (nie)słowa LEWINKLODZKI?
\end{zadanie}

\begin{zadanie}
  Ile jest przestawień liter $ABC$ i cyfr $1234$ tak, że najpierw stoją litery, a potem liczby? Ile jest przestawień, że litery nie stoją obok siebie?
\end{zadanie}

\begin{zadanie}
  Ile wynosi $n$, jeśli liczba permutacji zbioru mającego $(n+1)$ elementów jest o $600$ większa od liczby permutacji zbioru mającego $n$ elementów?
\end{zadanie}


\noindent\makebox[\linewidth]{\rule{\textwidth}{0.4pt}}

\begin{mybox}{Dwumian Newtona}
  Liczbę
  $$\binom{n}{k}=\frac{n!}{k!(n-k)!}$$
  nazywamy dwumianem Newtona. Warto kojarzyć tzw. trójkąt Pascala
  \begin{center}
    \begin{tikzcd}[column sep=tiny, row sep=tiny]
      n=0 &   &   &   &   & 1 \\ 
      n=1 &   &   &   & 1 &   & 1 \\ 
      n=2 &   &   & 1 &   & 2 &   & 1\\ 
      n=3 &   & 1 &   & 3 &   & 3 &   & 1\\ 
      n=4 & 1 &   & 4 &   & 6 &   & 4 &   & 1
    \end{tikzcd}
  \end{center}
  \medskip
  który w wierszu $n$ ma wartości $\binom{n}{k}$ dla $k$ odpowiadającemu numerowi kolumny (licząc od $k=0$ do $k=n$).
\end{mybox}

\begin{zadanie}
  Chcemy się wszyscy na sali przywitać uściskiem dłoni. Ile zostanie wymienionych uścisków dłoni?
\end{zadanie}

\begin{zadanie}[trudniejsze]
  Zakładając, że jesteśmy po poniedziałku 4 XI 2024, zgadnij (warto spojrzeć na $\triangle$ Pascala) wzór na
  $$\sum_{k=0}^n\binom{n}{k}$$
  i udowodnij go indukcyjnie.

  Możesz też spróbować zinterpretować powyższą wartość jako zliczanie ciągów długości $n$ złożonych z samych $0$ i $1$.
\end{zadanie}

\begin{zadanie}
  Dodaj do siebie $n$ pierwszych liczb naturalnych. Teraz dodaj do siebie kwadraty $n$ pierwszych liczb naturalnych. Czy umiesz wytłumaczyć ten wzór przy pomocy $\triangle$ Pascala i bez jego pomocy?
\end{zadanie}


\begin{zadanie}
  Na ile sposobów potrafisz rozłożyć $25$ skarpet do $5$ szuflad tak, żeby żadna szuflada nie była pusta?
\end{zadanie}

\begin{zadanie}
  Na ile sposobów umiesz zapisać liczbę $25$ jako sumę $5$ liczb naturalnych różnych od $0$? A jeśli chcesz wysumować w ten sposób liczbę $x$? 
  % Robert budowniczy buduje wieże wysokości $n$ z czerwonych i niebieskich bloków o wymiarach $1\times 1\times 1$. Ile różnych wież może ułożyć? 
\end{zadanie}

\begin{zadanie}
  Na ile sposobów potrafisz wysumować liczbę $10$ z $3$ liczb naturalnych (włączając $0$)? 
\end{zadanie}

\begin{mybox}{Zasada włączeń i wyłączeń}
  Przypomnijmy zasadę włączeń i wyłączeń z wczoraj. Niech $A_1$, ..., $A_n$ będą dowolnymi skończonymi zbiorami. Wtedy ilość elementów ich sumy mnogościowej wyraża się wzorem
  $$|A_1\cup...\cup A_n|=\sum_{i=1}^n|A_i|-\sum_{1\leq i< j\leq n}|A_i\cap A_j|+\sum_{i<j<k}|A_i\cap A_j\cap A_k|-...+(-1)^{n+1}|A_1\cap...\cap A_n|.$$
\end{mybox}
  
\begin{zadanie}
  Ile jest rozwiązań równania $x_1+x_2+x_3=80$, jeśli $0\leq x_i\leq 30$ dla $i=1,2,3$?
\end{zadanie}

\begin{mybox}{Pojęcie grafu}
  Grafem nazywamy parę zbiorów $G=(V, E)$, gdzie $V$ to zbiór wierzchołków, a elementy zbioru $E$ to (nieuporządkowane) pary $\{v, w\}$ dla $v,w\in V$. Zbiór $E$ nazywamy zbiorem krawędzi grafu $G$.
\end{mybox}

\begin{zadanie}
  Narysuj grafy $G=(V, E)$ dla:
  \begin{enumerate}
    \item $V=\{1,2,3,4\}$, $E=\{\{1,2\}, \{2,3\}, \{3, 4\} \}$ (taki graf nazwiemy drogą długości $4$ i oznaczymy $P_4$)
    \item $V=\{1,2,3,4\}$, $E=\{\{1,2\},\{2,3\},\{3,4\},\{4,1\}\}$ (taki graf oznaczamy $C_4$ - jak go nazwać?)
    \item $V$ - zbiór (dodatnich) liczb naturalnych nie większych niż $7$, $E = \{\{x, y\}\;:\;x, y \in V,\; x + y \text{ jest nieparzysta}\}$ (ten to z kolei $K_{3,4}$)
    \item $V$ to dwuelementowe podzbiory zbioru $\{1,2,3,4,5\}$, zaś $E=\{\{x,y\}\;:\;x,y\in V,\;x\cap y=\emptyset\}$ (taki graf to graf Petersona).
  \end{enumerate}
\end{zadanie}

\begin{zadanie}
  Czy potrafisz narysować grafy $C_6$, $K_{4,5}$?
\end{zadanie}

\begin{mybox}{Kiedy dwa grafy są tym samym?}
  Czasem zdarza się, że dwa grafy $G_1=(V_1,E_1)$ i $G_2=(V_2, E_2)$ mają wierzchołki podpisane w inny sposób, ale potrafimy narysować je w identyczny sposób. Takie grafy nazywamy izomorficznymi.

  Formalniej, potrafimy przeprowadzić wierzchołki $G_1$ na wierzchołki $G_2$ i vice versa tak, żeby wierzchołki połączone krawędzią nadal takie były.
\end{mybox}

\begin{zadanie}
  Ile jest nieizomorficznych grafów o $3$, $4$ lub $5$ wierzchołkach?
\end{zadanie}

\begin{zadanie}
  Drzewo to graf, w którym nie istnieje podzbiór wierzchołków $x_i\in V$ dla $0\leq i\leq k$ taki, że $x_0=x_k$ oraz $\{\{x_i, x_{i+1}\}\;:\;0\leq i\leq k\}\subseteq E$ (nie ma w nim cyklu). Policz, ile jest różnych drzew o:
  \begin{enumerate}
    \item trzech wierzchołkach
    \item czterech wierzchołkach
    \item pięciu wierzchołkach
  \end{enumerate}
\end{zadanie}

\begin{zadanie}
  Czy umiesz pokazać, ile krawędzi ma drzewo o $n$ wierzchołkach?
\end{zadanie}

\begin{mybox}{Stopień wierzchołka}
  Dla wierzchołka $v\in V$ przez jego zbiór sąsiadów rozumiemy zbiór
  $$\Gamma(v)=\{w\in V\;:\;\{v,w\}\in E\}.$$
  Ilość elementów tego zbioru nazywamy stopniem wierzchołka $v$ i oznaczamy $d(v)=|\Gamma(v)|$.
\end{mybox}

\begin{zadanie}
  Wybierz dowolny wierzchołek grafu Petersona, znajdź zbiór jego sąsiadów i oblicz jego stopień. Czy każdy wierzchołek tego grafu ma taki sam stopień?
\end{zadanie}

\begin{zadanie}
  Narysuj graf, którego wierzchołki mają stopnie $1,1,1,2,2,3,4$. A co, jeśli dołożymy jeszcze jeden wierzchołek stopnia $1$?
\end{zadanie}

\begin{zadanie}
  Jaki jest związek między liczbą krawędzi grafu a sumą stopni wszystkich jego wierzchołków? Czy umiesz wytłumaczyć dlaczego nie udało Ci się narysować drugiego grafu z poprzedniego zadania?
\end{zadanie}

\begin{mybox}{Graf regularny}
  Graf, w którym wszystkie wierzchołki mają ten sam stopień równy $k$ nazywamy grafem $k$-regularnym.
\end{mybox}

\begin{zadanie}
  Podaj przykład
  \begin{enumerate}
    \item grafu $2$-regularnego o $5$ wierzchołkach
    \item grafu $4$-regularnego, który nie jest izomorficzny z $K_5$ ani z $K_{4,4}$.
  \end{enumerate}
\end{zadanie}

\begin{zadanie}
  Ile jest grafów regularnych o $6$ wierzchołkach?
\end{zadanie}

\begin{zadanie}
  Dla jakich $n$ i $r$ istnieją grafy $r$-regularne o $n$ wierzchołkach?
\end{zadanie}

\begin{mybox}{Wzór Eulera}
  Graf planarny to taki, który umiesz narysować na kartce bez przecinających się krawędzi. Ścianą takiego grafu nazywamy dowolny obszar ograniczony przez jego krawędzie  lub przez krawędzie i brzegi kartki (tzn. $\triangle$ jako graf ma dwie ściany - środek trójkąta i obszar na zewnątrz). 

  Niech $v$ oznacza ilość wierzchołków grafu planarnego, $e$ ilość jego krawędzi, a $f$ - ilość ścian. Zachodzi wówczas równanie
  $$v-e+f=2$$
\end{mybox}

\begin{zadanie}
  "Gra w kropki" polega na tym, że po narysowaniu na kartce dowolnej ilości wierzchołków (kropek) gracze na przemian wykonują następujący ruch: łączą krawędzią (krzywą ciągłą) dwa wierzchołki i zaznaczają na niej nowy wierzchołek. Warunek, jaki musi być spełniony jest taki, że krawędzie nie mogą się przecinać i stopień wierzchołków nie może przekraczać $3$. Udowodnij, że gra zawsze kończy się po skończonej liczbie ruchów.
\end{zadanie}

\begin{mybox}{Kolorowanie grafu}
  Kolorowanie grafu $G=(V, E)$ przy pomocy $n$ kolorów to przypisanie jego wierzchołkom dokładnie jednego z tych kolorów tak, żeby wierzchołki połączone krawędzią nie miały tego samego koloru.
\end{mybox}

\begin{zadanie}
  Ile kolorów potrzebujesz, żeby pokolorować graf Petersena? Narysuj graf, dla którego potrzebujesz co najmniej $6$ kolorów.
\end{zadanie}

\end{document}
