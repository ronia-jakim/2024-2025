%! TeX program = lualatex
\documentclass{article}

\usepackage[bezRozw]{../template} 

\title{Wielomiany}
\author{Weronika Jakimowicz}
\date{}

\begin{document}

\noindent\makebox[\linewidth]{\rule{\textwidth}{0.4pt}}

\textbf{\large Rozgrzewka}

  \begin{zadanie}
    Udowodnij, że jeśli $a+b=c+d$ oraz $ab=cd$, to wówczas $a=b$ i $c=d$ lub $a=d$ i $b=c$.
  \end{zadanie}
  %
  % \begin{mybox}{Delta $\Deta$}
  %   Dla równania $ax^2+bx+c$ wyrażenie $\Delta=b^2-4ac$ nazywa się wyróżnikiem. Rozwiązania takiego równania to wtedy 
  %   $$\frac{-b\pm\sqrt{\Delta}}{2a}.$$
  % \end{mybox}
  %
  % \begin{zadanie}
  %   Dla każdej liczby $a>0$ wyznaczyć liczbę pierwiastków wielomianu $x^3+(a+2)x^2-x-3a$.
  % \end{zadanie}

  \begin{zadanie}
    Wykonaj dzielenie wielomianów
    \begin{enumerate}
      \item $(x^6-2x^4+2x^3-2x+1):(x^3-2x+1)$
      \item $(2x^7-3x^6+4x^4-x^2+2x+4):(2x^5+x^4-1)$
      \item $(x^4+x^3+10x^2+9x+9):(x^2+2x+1)$ \rozwiazanie{dzielę $(x-3i)(x+3i)(x-i)(x+i)$ przez $(x+1)^2$}
      \item $(38x^3+7x^2-8x-1):(x+\frac{1}{2})$
    \end{enumerate}
  \end{zadanie}

  \begin{zadanie}
    Rozłóż na czynniki wielomiany
    \begin{enumerate}
      \item $x^3+3x^2-4x-12$
      \item $2x^4-6x^3-8x^2$
      \item $9x^2-30x+25$
      \item $x^4+3x^3-15x^2-19x+30$
    \end{enumerate}
  \end{zadanie}

  \bigskip
\noindent\makebox[\linewidth]{\rule{\textwidth}{0.4pt}}

\begin{zadanie}
  Wielomian $W(x)$ przy dzieleniu przez $(x-5)$ daje resztę $1$, a przy dzieleniu przez $(x+3)$ daje resztę $-7$. Wyznacz resztę z dzielenia tego wielomianu przez wielomian $x^2-2x-15$.
\end{zadanie}

\begin{zadanie}
  Dany jest wielomian $W(x)=x^4+ax^3+bx^2+cx+d$. Pokaż, że jeśli $W(x)$ ma cztery pierwiastki rzeczywiste, to na to, żeby istniało $m$ takie, że $W(x+m)=x^4+px^2+q$ potrzeba i wystarczy, aby suma pewnych dwóch pierwiastków była równa sumie pozostałych dwóch. 
\end{zadanie}

% \begin{zadanie}
%   Reszta z dzielenia wielomianu $W(x)$ przez trójmian kwadratowy $P(x)=x^2+2x-2$ jest równa $R(x)=2x+5$. Wyznacz resztę z dzielenia tego wielomianu przez dwumian $(x-1)$.
% \end{zadanie}

\begin{zadanie}
  Podaj przykład takiego wielomianu $W(x)$ stopnia szóstego, który w wyniku podzielenia przez wielomian $P(x)=2x^3+8$ daje resztę będącą wielomianem stopnia drugiego.
\end{zadanie}

\begin{zadanie}
  Wielomian $W(x)$ o współczynnikach całkowitych daje przy dzieleniu przez wielomian $(x^2-12x+11)$ resztę $(990x-889)$. Wykaż, że wielomian ten nie ma pierwiastków całkowitych.
\end{zadanie}

\begin{zadanie}
  Dla jakich wartości parametrów $a$, $b$ wielomian $W(x)$ jest podzielny przez wielomian $P(x)$, jeśli:
  \begin{enumerate}
    \item $W(x)=x^4-2x^3+ax^2-3x+b$, $P(x)=x^2-3x+3$
    \item $W(x)=x^4-x^3-9x^2+ax+2$, $P(x)=x^2+2x+b$
  \end{enumerate}
\end{zadanie}

\begin{zadanie}
  Wielomian $W(x)$ jest stopnia drugiego i ma jeden pierwiastek dwukroty równy $3$. Czy wielomian $P(x)=[W(x)]^3(x^3+5x^2-9x-45)$ ma pierwiastki wielokrotne? Jeśli tak, to jakie? Podaj krotność pierwiastka wielokrotnego.
\end{zadanie}

\begin{zadanie}
  Przedstaw wielomian 
  \begin{enumerate}
    \item $W(x)=x^4+2x^3+5x^2+4x+3$ 
    \item $P(x)=x^4-3x^3+6x^2-5x+3$
  \end{enumerate}
  w postaci iloczynu wielomianów o współczynnikach całkowitych (dla $W$ - całkowitych dodatnich). 
  % Czy umiesz rozłożyć te wielomiany na czynniki liniowe?
\end{zadanie}

% \begin{mybox}{Czynniki liniowe}
%   Niech $P(x)$ będzie wielomianem stopnia $n$ z pierwiastkami (potencjalnie zespolonymi) $x_1,...,x_n$. Wówczas możemy zapisać
%   $$P(x)=a(x-x_1)(x-x_2)...(x-x_n).$$
% \end{mybox}

% \begin{zadanie}
%   Jednym z rozwiązań równania $3x^3+ax^2+bx+12=0$, gdzie $a,b\in\C$, jest liczba $1+\sqrt{3}$. Znajdź liczby $a$ i $b$.
% \end{zadanie}
%
% \begin{zadanie}
%   Dla jakich wartości parametru $a$ rozwiązania $x_1$, $x_2$, $x_3$, $x_4$ równania $x^4+5x^3+ax^2-40x+64=0$ spełniają warunki $x_2=-2x_1$, $x_3=4x_1$ i $x_4=-8x_1$? Wyznacz wszystkie rozwiązania równania. 
% \end{zadanie}

\begin{zadanie}
  Wiadomo, że $x_1$, $x_2$, $x_3$ są rozwiązaniami równania $x^3-2x^2+x+1=0$. Ułóż równanie, którego rozwiązaniami są $y_1=x_1x_2$, $y_2=x_1x_2$ i $y_3=x_2x_3$.
\end{zadanie}

\begin{mybox}{Podzielność}
  Jeśli $P(x)$ jest wielomianem o współczynnikach całkowitych, a $a,b\in\Z$, to wówczas
  $$a-b|P(a)-P(b).$$
\end{mybox}

% \begin{zadanie}
%   Wiadomo, że $x_1$, $x_2$, $x_3$ xą rozwiązaniami równania $x^3-x^2-1=0$. Ułóż równanie, którego rozwiązaniami są $y_1=x_1+x_2$, $y_2=x_1+x_3$ i $y_3=x_2+x_3$.
% \end{zadanie}

\begin{zadanie}
  Dla każdej liczby dodatniej $a$ wyznaczyć liczbę pierwiastków wielomianiu $x^3+(a+2)x^2-x-3a$.
\end{zadanie}

\begin{zadanie}
  Udowodnić, że jeżeli liczby $x_1$ i $x_2$ są pierwiastkami równania $x^2+px-1=0$, gdzie $p$ jest liczbą nieparzystą, to dla każdego naturalnego $n$ liczby $x_1^n+x_2^n$ i $x_1^{n+1}+x_2^{n+1}$ są całkowite i względnie pierwsze.
\end{zadanie}

\begin{zadanie}
  Dane są trzy różne liczby całkowite $a$, $b$, $c$. Udowodnić, że nie istnieje wielomian $w(x)$ o współczynnikach całkowitych taki, że $w(a)=b$, $w(b)=c$ i $w(c)=a$.
\end{zadanie}

\begin{zadanie}
  Pokaż, że jeśli wielomian $W(x)$ o współczynnikach całkowitych dla czterech różnych liczb całkowitych przyjmuje wartość $1$, to nie ma liczby całkowitej $p$ takiej, że $W(x)=-1$.
\end{zadanie}

\end{document}
