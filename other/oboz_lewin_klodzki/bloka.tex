%! TeX program = lualatex
\documentclass{article}

\usepackage[bezRozw]{./template} 

\title{Wielomiany}
\author{Weronika Jakimowicz}
\date{}

\begin{document}

\begin{zadanie}
  Przez co trzeba podzielić $50$, żeby otrzymać resztę $5$? Znajdź wszystkie możliwości.
\end{zadanie}

\begin{zadanie}
  Uzasadnij, że jeśli $m$ i $n$ są liczbami całkowitymi niepodzielnymi przez $3$, to jedna z liczb $mn+1$, $m-n$ jest podzielna przez $3$.
\end{zadanie}

\begin{zadanie}
  Uzasadnij, że liczba $321^{645}+123^{456}$ jest podzielna przez $10$.
\end{zadanie}

\begin{zadanie}
  Czy liczbę $1100$ można przedstawić w postaci iloczynu dwóch liczb, których największy wspólny dzielnik wynosi $11$?
\end{zadanie}

\begin{zadanie}
  Znajdź liczby całkowite $k$, $l$, $m$ dla których $6^k\cdot 10^l\cdot 15^m=9^{2000}$.
\end{zadanie}

\begin{zadanie}
  Dane są takie liczby całkowite $k$, $l$, że liczba $k+2l$ jest podzielna przez $3$. Wykaż, że liczba $2k+l$ też jest podzielna przez $3$.
\end{zadanie}

\begin{zadanie}
  Dane są takie liczby całkowite $k$, $l$, $m$, że liczba $2k+3l+4m$ jest podzielna przez $5$ Wykaż, że liczba $k+2m+4l$ też jest podzielna przez $5$.
\end{zadanie}

\begin{zadanie}
  Wyznaczyć reszty, jakie mogą dawać;
  \begin{enumerate}
    \item kwadraty liczb całkowitych przy dzieleniu przez $3$, $4$, $5$, $7$, $10$, $16$
    \item szcześciany liczb całkowitych przy dzieleniu przez $4$, $7$, $9$.
  \end{enumerate}
\end{zadanie}

\begin{zadanie}
  Udowodnić, że nie istnieją liczby $a,b,c\in\N$ takie, że $a^2+b^2=8c+6$.
\end{zadanie}

\begin{zadanie}
  Znajdź liczbę dwucyfrową równą podwojonemu iloczynowi swoich cyfr.
\end{zadanie}

\begin{zadanie}
  Znajdź wszystkie liczby trzycyfrowe, które są $11$ razy większe od sumy swoich cyfr.
\end{zadanie}

\begin{zadanie}
  Znajdź wszystkie liczby trzycyfrowe, które przy dowolnym przestawieniu ich cyfr dają liczbę podzielną przez $27$.
\end{zadanie}

\begin{zadanie}
  Dla jakich trójek cyfr $(a,b,c)$ zachodzi równość
  $$\overline{\underbrace{aa....aa}_{n}\underbrace{bbb...bb}_{n}}+1=(\overline{\underbrace{ccc...ccc}_n}+1)^2)$$
\end{zadanie}

\begin{zadanie}
  Dane są liczby $a,b,c,d,e\in\Z$, dla których $a^3+b^3+c^3+d^3+e^3$ jest podzielna przez $9$. Dowieść, że przynajmniej jedna z tych pięciu liczb jest podzielna przez $3$.
\end{zadanie}

\begin{zadanie}
  Wyznaczyć najmniejszą liczbę naturalną, jakiej może być równe wyrażenie $36^k-5^l$ dla pewnych liczb naturalnych $k$, $l$.
\end{zadanie}

\end{document}
