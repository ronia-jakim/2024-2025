% !TeX TS-program = lualatex
\documentclass{article}

\usepackage{triviaclass}

\tikzset{
  pics/torus/.style n args={3}{
    code = {
      \begin{scope}[
          yscale=cos(#3),
          outer torus/.style = {draw,line width/.expanded={\the\dimexpr2\pgflinewidth+#2*2},line join=round},
          inner torus/.style = {draw=back,line width={#2*2}}
        ]
        \draw[outer torus] circle(#1);\draw[inner torus] circle(#1);
        \draw[outer torus] (180:#1) arc (180:360:#1);\draw[inner torus,line cap=round] (180:#1) arc (180:360:#1);
      \end{scope}
    }
  }
}

\title{Grupa podstawowa}
\short{
  Pozwala rozróżnić przestrzenie badając ile rożnych, czyli niedających się przekształcić w siebie bez rozcinania, pętli można w niej znaleźć. 
  Na przykład 
  $$\pi_1(S^1)=\Z,$$ 
  bo mamy tylko jedną pętlę, którą możemy wielokrotnie nawijać na siebie.
}
\fields{topologia algebraiczna}

\begin{document}

\maketitle

\begin{slide}
  Grupę podstawową okręgu $S^1$ często ilustruje się helisą zawieszoną nad okręgiem. To, ile razy obejdziemy $S^1$ po sprężynce oznacza, który element grupy podstawowej reprezentujemy.
  \bigskip

\centering
  \scalebox{.8}{\centering
  \begin{tikzpicture}
     \begin{axis}[
        view={-20}{-20},
        axis line style = {draw=none},
        axis lines=middle,
        height=12cm,
        xtick=\empty,
        ytick=\empty,
        ztick=\empty,
        clip=false,
      ]
      \addplot3+[domain=0:11*pi,samples=500,samples y=0,no marks,thick, subtitle] 
        ({sin(deg(x))}, 
        {cos(deg(x))}, 
        {6*x/(pi)});

      \addplot3+[domain=0:pi, samples=100, samples y = 0, no marks, ultra thick, accent] (
          {sin(deg(x))},
          {cos(deg(x))},
          -.5
        );
      
      \addplot3+[domain=0:pi, samples=100, no marks, samples y=0, ultra thick, accent] (
          {sin(deg(pi+x))},
          {cos(deg(pi+x))},
          -.5
        );
    \end{axis}
  \end{tikzpicture}
}
\end{slide}

\begin{slide}
  Niech $(X, x)$ będzie przestrzenią topologiczną z wyróżnionym punktem $x\in X$. Przez \buff{pętlę} na $(X, x)$ rozumiemy ciągłe odwzorowanie
  $$\gamma:[0,1]\to (X, x)$$
  takie, że $\gamma(0)=\gamma(1)=x$. 

  Mając dwie pętle $\gamma_1$ i $\gamma_2$ możemy wyprodukować nową pętle, $\gamma_1\cdot\gamma_2$, która najpierw podróżuje trasą wyznaczoną przez $\gamma_1$, a potem przez $\gamma_2$. Czy umiesz wyrazić to wzorem?
\bigskip 

\begin{center}
  \begin{tikzpicture}
    \draw[ultra thick, orange, ->] (0,0) to[out=30, in=0] (0, 2);
    \draw[ultra thick, orange] (0,2) to[out=180, in=150] (0,0);

    \node at (1, 1) {{\color{orange}\boldmath $\gamma_1$}};
    
    \draw[ultra thick, blue, ->] (0,0) to[out=0, in=150] (1, -1);
    \draw[ultra thick, blue] (1, -1) to[out=-30, in=90] (2, -1.5) to[out=-90, in=-90] (-1,-1) to[out=90, in=-150] (0,0);
    
    \node at (1, -.5) {{\color{blue}\boldmath $\gamma_2$}};
    
    \fill[accent] (0,0) circle (2pt);

    \fill[orange] (3, 2) rectangle (3.1, -2.2);
    \fill[blue] (3.5, 2) rectangle (3.6, -2.2);
    \fill[orange] (4, 2) rectangle (4.1, -.1);
    \fill[blue] (4, -.1) rectangle (4.1, -2.2);

    \draw (2.5, 2)--(4.5, 2);
    \node at (4.8, 2) {$0$};
    
    \draw (2.5, -2.2)--(4.5, -2.2);
    \node at (4.8, -2.2) {$1$};

    \draw[->, orange, thick] (3, 1.8) to[out=200, in=70] (.3, 2);
    \draw[->, blue, thick] (3.5, -2) to[out=-90, in=-70] (1.8, -2);

    \draw [decorate,decoration={brace,amplitude=5pt,raise=1ex}] (4.2,2) -- (4.2,-2.2) node[midway, xshift=3em]{{\boldmath$\color{orange}\gamma_1\color{accent}\cdot\color{blue}\gamma_2$}};
  \end{tikzpicture}
\end{center}
\end{slide}

\begin{slide}
  Dwie pętle $\gamma_1$, $\gamma_2$ są \buff{homotopijne}, jeśli istnieje ciągłe odwzorowanie 
  $$h:[0,1]\times[0,1]\to X$$
  takie, że 
  \begin{itemize}
    \item $h(0,t)=h(1,t)=x$ dla wszystkich $t$ 
    \item oraz $h(y, i)=\gamma_i(y)$ dla $i=0,1$. 
  \end{itemize}
  Pierwsza współrzędna homotopii opisuje położenie $y$ na pętli, a druga przekształcanie tej pętli w czasie $t$.
  \bigskip

  \begin{center}
  \begin{tikzpicture}
    \node at (0, 1.7) {{\boldmath$\color{orange}\gamma_1$}};
    \node at (0, 4.3) {{\boldmath$\color{blue}\gamma_2$}};
    \draw[orange, very thick] (0,0) to[out=0, in=0] (0, 2) to[out=180, in=180] (0,0);
    
    \draw[blue, very thick] (0,0) to[out=0, in=0] (0, 4) to[out=180, in=180] (0,0);

    \foreach \r in {2.25, 2.5,..., 3.75} \draw[dashed] (0,0) to[out=0, in=0] (0, \r) to[out=180, in=180] (0,0);
    
    \fill[accent] (0,0) circle (2pt);
  \end{tikzpicture}
\end{center}
\end{slide}

\begin{slide}
  Relacja homotopijności jest \buff{relacją równoważności pętli}. Dwie pętle są w tej samej klasie abstrakcji, jeśli \buff{jedną jesteśmy w stanie w ciągły sposób przekształcić w drugą}.
  \bigskip

  Możemy więc zdefiniować \buff{grupę podstawową $\pi_1(X, x)$} zbazowanej przestrzeni $(X, x)$ w formalny sposób jako \emph{grupę klas abstrakcji relacji homotopijnej równoważności pętli na przestrzeni $(X, x)$} z działaniem składania pętli.
  \bigskip

  \begin{center}
    \begin{tikzpicture}
      \pic{torus={1.5cm}{.4cm}{70}};

      % \draw (0,0)--(0,-.55);
      % \draw (1.9, 0)--(0,0);
      % \draw (1.9, 0) to[out=200, in=0] (0, -.55) to[out=180, in=-20] (-1.9, 0);
      \draw[green, thick] (-1.9, 0) arc (180:360:1.9 and .55);
      \draw[green, dashed] (-1.9, 0) arc (180:0:1.9 and .55);
      % \draw (0,0)--(.5, -.08);
      % \draw (0,0)--(.5, -.91);
      \draw[orange, thick] (.5, -.08)to[out=0, in=0](.5, -.905);
      \draw[orange, dashed] (.5, -.08)to[out=180, in=180](.5, -.905);

      \node at (-2.2, 0) {{\boldmath\color{green}$\gamma_1$}};
      \node at (.5, -1.2) {{\boldmath\color{orange}$\gamma_2$}};

      \node at (0, -2) {$\pi_1(\mathbb{T}^2)=\langle \gamma_1, \gamma_2\rangle\cong \Z^2$};

      \begin{scope}[shift={(3.5, 0)}]
        \draw (0,0) to[out=-90, in=180] 
              (1, -1) to[out=0, in=180] 
              (3, -.3) to[out=0, in=-120]
              (3.6, 0) to[out=60, in=-20]
              (3.3, .8) to[out=160, in=40]
              (2, .65);
        \draw(0,0) to[out=90, in=180]
             (1, 1) to[out=0, in=180]
             (3, .3) to[out=0, in=-20]
             (2.9, .5) to[out=160, in=40]
             (2.4, .45);

           \filldraw[fill=back] (2.4, .45) to[out=210, in=-90] (2, .66); 
           \draw[dashed] (2, .65) to[out=220, in=0] 
             (1, 0) to[out=180, in=-60]
             (.5, 0.3);
           \draw[dashed] (2.4, .45) to[out=220, in=0] (1, -.4) to[out=180, in=60] (.5, -.4);
           \draw (.5, -.4) to[out=180, in=180] (.5, .3) to[out=0, in=0] (.5, -.4);

           \node at (1.5, -2) {$\pi_1(K)=\langle a,b\;|\;abab^{-1}\rangle\cong\Z\rtimes\Z$};
      \end{scope}
    \end{tikzpicture}

  \end{center}
\end{slide}

\end{document}
