\documentclass[a4paper,12pt]{article}
\usepackage[utf8]{inputenc}
\usepackage[OT4,plmath]{polski}
\usepackage[pdftex]{color,graphicx}
\usepackage[a4paper,tmargin=3cm,bmargin=3cm,lmargin=3cm,rmargin=3cm]{geometry}
\usepackage{wrapfig}
\usepackage{amsmath}
\usepackage{amssymb}
\usepackage{lipsum}

\usepackage{svg}


\pagestyle{empty}


\begin{document}


\hrule
\vspace*{3ex}
\begin{minipage}{0.25\textwidth}
\begin{flushleft}
\includesvg[width=3cm]{logo.svg}
\end{flushleft}
\end{minipage}
\begin{minipage}{0.7\textwidth}
\Large{\bf Koło Naukowe \\ Matematyków Teoretyków UWr}
\end{minipage}
\vspace*{5ex}
\hrule

\vspace*{5ex}
%\begin{flushleft}
%\large{\textbf{data}: piątek, 22 marca 2023\\
%\textbf{godzina}: 12:15 \\
%\textbf{miejsce}: Sala 604, IM UWr}
%\end{flushleft}

\vspace*{5ex}

\begin{center}
\LARGE{\sc Moduł Alexandera węzła}
\end{center}
\begin{center}
\large{Weronika Jakimowicz} \\  
\end{center}

\vspace*{8ex}

Węzeł $K$ to włożenie okręgu $S^1$ w sferę trójwymiarową $S^3$. Istotną częścią teorii węzłów jest szukanie niezmienników, które mogą być użyte do rozróżniania węzłów nierównoważnych. 
Interesującym, a zarazem trudnym, niezmiennikiem jest grupa podstawowa dopełnienia węzła, zwana też grupą węzła, oznaczana $\pi_1(K):=\pi_1(S^3-K)$. Jest wiele sposobów uproszczenia informacji w niej zawartej i jedną z nich jest przejście od grup do modułów nad pierścieniem wielomianów Laurenta $\mathbb{Z}[\mathbb{Z}]$ i badanie modułu Alexandera wraz z jego rezolwentą. Na moim referacie przedstawię jak na podstawie prezentacji Wirtingera grupy węzła uzyskać moduł Alexandera wraz z jego rezolwentą.


\vspace*{5ex}
\begin{flushright}
\large{\sc Serdecznie zapraszam!}
\end{flushright}


\end{document}
