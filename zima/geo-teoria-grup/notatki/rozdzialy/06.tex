\section{06.11.2024}{Grupy o dwóch końcach}

Z quasi-izometryczności grup współmiernych, jakimi są grupy i ich podgrupy skończonego indeksu, wynika, że jeśli grupa zawiera skończonego indeksu podgrupę $\Z$, to ma wówczas dwa końce. Celem wykładu będzie udowodnienie implikacji w drugą stronę, czyli opisanie grup o dwóch końcach.

\begin{theorem}{}{2 konce}
  Każda grupa o $2$ końcach zawiera skończonego indeksu podgrupę acykliczną (izomorficzną z $\Z$).
\end{theorem}

Wynika więc z tego, że każda grupa q.i. z $\Z$ zawiera $\Z$ jako podgrupę skończonego indeksu.

\subsection{Działanie grupy na przestrzeni końców}

Skończenie geneorwalna grupa $G$ indukuje w naturalny sposób działanie przez permutacje na zbiorze swoich końców homomorfizm 
$$h^E:G\to Sym(\Ends(G))$$
zadany na jeden z dwóch sposobów:
\begin{enumerate}
  \item izometria $\phi:X\to X$ wyznacza automorfizm 
    $$\phi^{\mathcal{K}}:\mathcal{K}\to\mathcal{K}$$
    przez 
    $$\phi^{\mathcal{K}}(K)=\phi(K),$$
    zaś dla każdego $K\in\mathcal{K}$ bijekcję
    $$\phi_K:\Pi^X_K\to \Pi_{\phi(K)}^X$$
    zadaną przez $\phi_K(C)=\phi(C)$. To razem daje automorfizm 
    $$\phi^X:\Pi^X\to \Pi^X,$$ 
    który indukuje homeomofrizm granic odwrotnych (w szczególności bijekcję).
  \item izometria $\phi:X\to X$ zadaje 
    $$h_\phi^E:\Ends(X)\to \Ends(X)$$
    przez 
    $$h_\phi^E([\rho])=[\phi\circ\phi],$$
    gdzie $\rho$ to promień w $X$.
\end{enumerate}

\begin{fact}{}{kozwarcie skonczony indeks}
  Niech grupa $\Gamma\acts X$ działa właściwie, kozwarcie, gdzie $X$ jest właściwa geodezyjna oraz $H\leq \Gamma$. Wówczas $[\Gamma:H]<\infty$ $\iff$ $H\acts X$ działa kozwarcie.
\end{fact}

\begin{proof}
  $[\Gamma:H]<\infty\implies H\acts X$ kozwarcie

  Niech $\Gamma=hg_1\cup...\cup Hg_m$, gdzie $[\Gamma:H]=m$. Jeśli $\bigcup_{\gamma\in\Gamma}\gamma\cdot K=X$, to 
  $$\bigcup_{h\in H}h(g_1K\cup...\cup g_mK)=X,$$
  stąd kozwartość działania $H$.

  $[\Gamma:H]<\infty\impliedby H\acts X$ kozwarcie

  Niech $L\subseteq X$ będzie zwartym zbiorem z definicji kozwartości, tzn. 
  $$\bigcup_{h\in H}h\cdot L=X.$$
  Z własności działania grupy $\Gamma$ wiemy, że
$$\left|\{\gamma\in\Gamma\;:\;\gamma L\cap L\neq \emptyset\}\right|<\infty,$$
powiedzmy że jest to zbiór $\{\gamma_1,...,\gamma_m\}$. Wówczas $H\gamma_1\cup...\cup H\gamma_m=\Gamma$, bo dla $\gamma\in \Gamma$ istnieje $h\in H$ takie, że $h\cdot L\cap \gamma\cdot L\neq\emptyset$. Więc $L\cap h^{-1}\gamma L\neq\emptyset$, więc $h^{-1}\gamma=\gamma_j$, czyli $\gamma=h\gamma_j\in H\gamma_j$.
\end{proof}

Dla właściwej geodezyjnej przestrzeni $X$ i dla dowolnego zwartego $K\subseteq X$, liczba komponent (ograniczonych i nieograniczonych) w $X-K$ jest skończona i każda z tych komponent jest otwarta w $X$. Stąd, każdy $K\subseteq X$ możemy uzupełnić o ograniczone komponenty $X-K$, otrzymując nowy zbiór $K_+$ taki, że komponenty $X-K_+$ to dokładnie nieograniczone komponenty $X-K$.

Możemy więc myśleć, że \hl{$X-K$ dla zwartych $K$ to zawsze skończona suma nieograniczonych komponent tego dopełnienia} (i wszystkie one są otwarte w $X$).

\subsection{Grupy o 2 końcach zawierają cykliczną podgrupę skończonego indeksu}

Wróćmy do dowodu twierdzenia \ref{th:2 konce}.

\begin{proof}
  Niech $\Gamma$ będzie dowolną grupą o $2$ końcach. Rozważmy homomorfizm $h^{\#}:\Gamma\to Sym(\Ends(G))=\Z_2$ i jego jądro $\Gamma_0=\ker(h^E)<\Gamma$ - podgrupę indeksu $\leq 2$.

  $\Gamma_0$, jako podgrupa skończonego indeksu, w dalszym ciągu działa kozwarcie na $X$ (fakt \ref{fac:kozwarcie skonczony indeks}), oraz zachowuje wszystkie końce. Naszym celem będzie znalezienie generatora $g\in \Gamma_0$ dla cyklicznej podgrupy $\langle g\rangle$ skończonego indeksu w $\Gamma_0$.

  Niech $K$ będzie takie, że $X-K=E\cup E'$ jest sumą nieograniczonych komponent spójności, oraz $\bigcup_{\gamma\in\Gamma_0}\gamma\cdot K=X$. Ustalmy $z\in E$ taki, że $d_X(z, K)>2\cdot \text{diam}K$.

  Ponieważ orbity zbioru $K$ przez działanie $\Gamma_0$ pokrywają $X$, to możemy znaleźć $g\in\Gamma_0$ takie, że $z\in g\cdot K$. Wówczas $d_X(K, gK)>\text{diam}K$ (bo $\text{diam}(gK)=\text{diam}(K)$). Stąd $gK\subseteq E$ ($gK\cap E\neq\emptyset$ oraz pokazaliśmy, że między $gK$ oraz $K$ mieści się co najmniej jedna średnica $K$).

  Ponieważ $gK\cap E'=\emptyset$ (bo $E\cap gK\neq\emptyset$ a są to zbiory spójne), to $E'$ zawiera się w dokładnie jednej komponencie $X-gK$, którymi są $gE$ oraz $gE'$. Ponieważ $g\in \Gamma_0$ zachowuje końce, to komponentą tą musi być $gE'$, zatem $E'\subseteq gE'$.

  {\large\color{red}TODO: DOKOŃCZYĆ DOWÓD}
\end{proof}







