\section{06.11.2024}{Grupy o dwóch końcach}

Z quasi-izometryczności grup współmiernych, jakimi są grupy i ich podgrupy skończonego indeksu, wynika, że jeśli grupa zawiera skończonego indeksu podgrupę $\Z$, to ma wówczas dwa końce. Celem wykładu będzie udowodnienie implikacji w drugą stronę, czyli opisanie grup o dwóch końcach.

\begin{theorem}{}{}
  Każda grupa o $2$ końcach zawiera skończonego indeksu podgrupę acykliczną (izomorficzną z $\Z$).
\end{theorem}

Wynika więc z tego, że każda grupa q.i. z $\Z$ zawiera $\Z$ jako podgrupę skończonego indeksu.


