\section{08.01.2025}{Brzeg Gromova $\partial G$ grupy hiperbolicznej $G$ o nieskończenie wielu końcach.}

Do tej pory dowiedzieliśmy się, że
\begin{itemize}
  \item $G$ ma $\infty$ wiele końców $\iff$ $\Ends(G)\cong$ zbiór Cantora,
  \item jeśli $G$ jest hiperboliczna, to końce $G$ odpowiadają komponentom spójności $\partial G$.
\end{itemize}

Pozostaje pytanie, jak wygląda $\partial G$, gdy hiperboliczna grupa $G$ ma $\infty$ wiele końców?

Rozważmy przypadek $\Gamma=G\ast H$, gdzie $G$ i $H$ to nieskończone grupy hiperboliczne. Niech $S$ będzie skończonym zbiorem generatorów $G$, a $T$ - skończonym zbiorem generatorów $H$. $S\cup T$ jest więc zbiorem generatorów $\Gamma$ i graf Cayleya $C=C(\Gamma, S\cup T)$ to suma drzewiasta grafów $G$ oraz $H$.

Promienie geodezyjne w $C$ to sklejone kawałki geodezyjnych z $C(G,S)$ oraz $C(H,T)$. Można podzielić je na dwa rodzaje
\begin{itemize}
  \item promienie, które od pewnego miejsca są w pojedynczej kopii $C(G,S)$ lub $C(H,T)$
  \item promienie, które przechodzą przez nieskończenie wiele kopii grafów grup składowych.
\end{itemize}
Dla każdej kopii $C_0$ grafu $C(G,S)$ w $C$ promienie geodezyjne o początku w $e$ pozostające od pewnego miejsca w tej kopii wyznaczają podzbiór w $\partial\Gamma=\partial C$, który oznaczymy przez $\partial C_0$.

\begin{fact}{}{}
  $\partial C_0$ jest metrycznie przeskalowaną kopią brzegu $\partial G=\partial C(G,S)$, o czynnik $a^{-D}$, gdzie $D$ jest to odległość od $e$ do tego wierzchołka w $C_0$ przez który wchodzą do $C_0$ promienie o początku w $e$.
\end{fact}

\begin{fact}{}{}
  Dla różnych kopii $C_0$ i $C_0'$ grafów $C(G,S)$ lub $C(H,T)$ w $C$ podzbiory $\partial C_0,\partial C_0'\subseteq\partial (G\ast H)$ są rozłączne.
\end{fact}











