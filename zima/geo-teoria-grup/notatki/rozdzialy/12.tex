\section{08.01.2025}{Brzeg Gromova grupy hiperbolicznej o nieskończenie wielu końcach.}

Do tej pory dowiedzieliśmy się, że
\begin{itemize}
  \item $G$ ma $\infty$ wiele końców $\iff$ $\Ends(G)\cong$ zbiór Cantora,
  \item jeśli $G$ jest hiperboliczna, to końce $G$ odpowiadają komponentom spójności $\partial G$.
\end{itemize}

Pozostaje pytanie, jak wygląda $\partial G$, gdy hiperboliczna grupa $G$ ma $\infty$ wiele końców?

Rozważmy przypadek $\Gamma=G\ast H$, gdzie $G$ i $H$ to nieskończone grupy hiperboliczne. Niech $S$ będzie skończonym zbiorem generatorów $G$, a $T$ - skończonym zbiorem generatorów $H$. $S\cup T$ jest więc zbiorem generatorów $\Gamma$ i graf Cayleya $C=C(\Gamma, S\cup T)$ to suma drzewiasta grafów $G$ oraz $H$.

Promienie geodezyjne w $C$ to sklejone kawałki geodezyjnych z $C(G,S)$ oraz $C(H,T)$. Można podzielić je na dwa rodzaje
\begin{itemize}
  \item promienie, które od pewnego miejsca są w pojedynczej kopii $C(G,S)$ lub $C(H,T)$
  \item promienie, które przechodzą przez nieskończenie wiele kopii grafów grup składowych.
\end{itemize}
Dla każdej kopii $C_0$ grafu $C(G,S)$ w $C$ promienie geodezyjne o początku w $e$ pozostające od pewnego miejsca w tej kopii wyznaczają podzbiór w $\partial\Gamma=\partial C$, który oznaczymy przez $\partial C_0$.

\begin{fact}{}{}
  $\partial C_0$ jest metrycznie przeskalowaną kopią brzegu $\partial G=\partial C(G,S)$, o czynnik $a^{-D}$, gdzie $D$ jest to odległość od $e$ do tego wierzchołka w $C_0$ przez który wchodzą do $C_0$ promienie o początku w $e$.
\end{fact}

\begin{fact}{}{}
  Dla różnych kopii $C_0$ i $C_0'$ grafów $C(G,S)$ lub $C(H,T)$ w $C$ podzbiory $\partial C_0,\partial C_0'\subseteq\partial (G\ast H)$ są rozłączne.
\end{fact}

\begin{definition}{gęsty amalgamat}{}
  Dla dowolnego układu $X_1$,..., $X_k$ niepustych zwartych przestrzeni metrycznych, zwartą przestrzeń metryczną $Y$ nazywamy \buff{gęstym amalgamatem} przestrzeni $X_1$, ..., $X_k$ gdy można wyróżnić w niej nieskończoną przeliczalną rodzinę $\mathcal{Y}$ podzbiorów, podrozbitą jako $\mathcal{Y}=\mathcal{Y}_1\sqcup ...\sqcup \mathcal{Y}_k$ taką, że 
  \begin{enumerate}
    \item podzbiory z $\mathcal{Y}$ są parami rozłączne, zaś dla $1\leq i\leq k$ podrodzina $\mathcal{Y}_i$ składa się z włożonych kopii przestrzeni $X_i$
    \item rodzina $\mathcal{Y}$ jest zerowa, tzn. dla dowolnej metryki na $Y$ zgodnej z topologią, średnice zbiorów z $\mathcal{Y}$ dążą do zera
    \item każdy $Z\in\mathcal{Y}$ jest zbiorem brzegowym, tzn. jego dopełnienie $Y-Z$ jest gęste w $Y$ (lub każdy $z\in Z$ jest granicą ciągu punktów z $Y-Z$)
    \item dla każdego $i$ suma $\bigcup \mathcal{Y}_i$ rodziny zbiorów $\mathcal{Y}_i$ jest gęsta w $Y$
    \item dowolne $2$ punkty z $Y$ nie należące do tego samego podzbioru z $\mathcal{Y}$ można oddzielić od siebie $\mathcal{Y}$-nasyconym otwarto-domkniętym podzbiorem $H\subseteq Y$ ($H$ jest $\mathcal{Y}$-nasycony, gdy każdy $Z\in\mathcal{Y}$ jest rozłączny z $H$ albo zwarty w $H$)
  \end{enumerate}
\end{definition}

\begin{fact}{}{}
  \begin{enumerate}
    \item Gęsty amalgamat układu złożonego z jednej $1$-punktowej przestrzeni to zbiór Cantora
    \item Dla dowolnego skończonego układu $X_1$,..., $X_k$ niepustych zwartych przestrzeni metrycznych, ich gęsty amalgamat istnieje i jest jednoznaczny z dokładnością do homeomorfizmu.
  \end{enumerate}
\end{fact}

Amalgamat przestrzeni $X_1$, ..., $X_k$ oznaczamy $\color{blue} \amalg (X_1,...,X_k)$.

\begin{lemma}
  Niech $G,H$ będą nieskończonymi hiperbolicznymi grupami takimi, że $\partial G\neq \emptyset$ i $\partial H\neq\emptyset$. Wówczas
  $$\partial (G\ast H)\cong \amalg(\partial G, \partial H) $$
\end{lemma}

Jeśli grupa $G$ ma nieskończenie wiele końców, to zgodnie z twierdzeniem Stallingsa, rozkłada się (nietrywialnie) nad skończoną podgrupą w produkt wolny z amalgamacją lub w HNN-rozszerzenie.

Gdy $\Gamma$ jest hiperboliczna, to faktory powyższego rozkładu też są hiperboliczne i jeśli któryś ma nieskończenie wiele końców, to można go ponownie rozłożyć.

Z twierdzenia Dunwoody iterowanie rozkładów nad skończonymi podgrupami jak wyżej zawsze się kończy, a te końcowe faktory są albo skończone, albo o jednym końcu.

Niech $H_1,...,H_k$ będzie układem terminalnych (nieskończonych) faktorów rozkładu $\Gamma$ nad skończonymi podgrupami. Wówczas
$$\partial\Gamma\cong\amalg(\partial H_1,...,\partial H_k)$$
łącznie z przypadkiem $\partial\Gamma=\amalg\emptyset=\text{zbiór Cantora}$, gdy $k=0$.
W takiej sytuacji kopie $\partial H_i$ w $\partial \Gamma$ są komponentami spójności $\partial\Gamma$, a ponadto jest jeszcze dużo $1$-punktowych komponent spójności.




