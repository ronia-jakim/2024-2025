\section{18.12.2024}{Końce a brzeg grupy hiberbolicznej}

\begin{theorem}{}{}
  Dla grupy hiperbolicznej $G$ przyporządkowanie 
  $$\partial_\infty G\to \Ends(G)$$
  zadane przez 
  $$[\rho]_{\partial G}\mapsto [\rho]_{\Ends(G)}$$
  jest dobrze określone i zadaje bijekcję pomiędzy komponentami spójności $\partial G$ a końcami $G$.
\end{theorem}

Stąd wynika, że grupa hiperboliczna $G$ ma jeden koniec $\iff$ $\partial_\infty G$ jest spójny.

Przypomnijmy, że punkty $x$ i $y$ są w tej samej komponencie spójności $X$ $\iff$ nie da się ich oddzielić zbiorami otwarto-domkniętymi. Inaczej: komponenta spójności punktu $x\in X$ to przekrój wszystkich zbiorów otwarto-domkniętych, które go zawierają.

\begin{enumerate}
  \item Przy ustalonym $\epsilon>0$ zbiór punktów w $X$, które można połączyć $\epsilon$-drogą z ustalonym punktem $x_0\in X$ ($\epsilon$-komponenta punktu $x_0$) jest otwarty i domknięty w $X$. 
  \item Gdy $X$ jest zwarta i metryczna, punkty $x,y$ należą do tej samej komponenty w $X$ $\iff$ dla dowolnego $\epsilon>0$ istnieje w $X$ $\epsilon$-droga od $x$ do $y$.
\end{enumerate}

\begin{fact}{}{}
  Jeśli promienie geodezyjne $\rho_1$, $\rho_2$ w $C(G,S)$ o początku w $e$ reprezentują punkty z tej samej komponencie w $\partial G$ (np. ten sam punkt), to promienie te są współkońcowe.
\end{fact}

\begin{fact}{}{}
  Współkońcowe promienie geodezyjne $\rho_1$, $\rho_2$ w $C(G,S)$ (o początku w $e$) reprezentują punkty z tej samej komponenty spójności w $\partial G$.
\end{fact}

Dla dowolnej grupy hiperbolicznej $G$ istnieje stała $C>0$, zależna od $G$ i $S$, taka, że dla każdego $x\in C(G,S)$ w odległości nie większej niż $C$ od $x$ przebiega promień geodezyjny o początku w punkcie $e$.











