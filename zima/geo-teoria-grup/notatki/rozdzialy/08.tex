\section{20.12.2024}{Funkcja wzrostu i metryka Riemanna}

% \subsection{Związek ze wzrostem objętości}

\begin{definition}{}{}
  Niech $M$ będzie zupełną rozmaitościa z metryką Riemanna $g$, która indukuje metrykę $d_g$ i miarę objętości $\vol_g$.

  \buff{Funkcja wzrostu objętości} (volume growth) na rozmaitości $M$ względem punktu bazowego $p\in M$ to funkcja $\beta_g^{M,p}:\R_{\geq 0}\to \R_{\geq0}$ zadana przez 
  % $$r\overset{\mapsto}{\beta_{g}^{M,p}} \vol_g(B_r(p)), $$
  $$\beta_g^{M,p}(r)=\vol_g(B_r(p))$$
  
  gdzie $B_r(p)$ to kula względem metryki $d_g$.
\end{definition}

\begin{lemma}{Milnor-\v{S}varc}{}
  Niech $M$ będzie zamkniętą, spójną rozmaitością Riemannowską i niech $\widetilde{M}$ będzie jej nakryciem uniwersalnym z indukowaną metryką $\widetilde{g}$. Wówczas dowolna funkcja wzrostu objętości $\beta_{\widetilde{g}}^{\widetilde{M},p}$ na $\widetilde{M}$ jest quasi-równoważna z funkcją wzrostu $\beta_{\pi_1 M}$ grupy podstawowej $\pi_1M$.
\end{lemma}

\begin{example}
  Niech $M=\mathbb{T}^2=S^1\times S^1$ ze standardową metryką produktową. Wówczas $(\widetilde{M}, \widetilde{g})=\R^2$ oraz $\pi_1M=\Z^2$. Obie funkcje wzrostu są tutaj kwadratowe.
\end{example}

\begin{proof}
  Zacznijmy od kilku stwierdzeń
  \begin{itemize}
    \item grupa podstawowa zamkniętej rozmaitości jest skończenie generowalna
    \item oraz działa na nakryciu uniwersalnym (przez deck-transformacje) $\pi_1 M\acts\widetilde{M}$.
  \end{itemize}
  Niech teraz $S=\{s_1,...,s_n\}$ będzie skończonym układem generatorów $\pi_1M$ i $p\in\widetilde{M}$ będzie dowolnym punktem. \buff{Promień injektywności} $M$ definiujemy jako
  $$\text{injrad}(M, g):=\frac{1}{2}\inf \{|\gamma|_g\;:\;\gamma\text{ - gładka, homotopijnie nietrywialna pętla w }M\}.$$
  Dla rozmaitości zamkniętych jest to zawsze liczba dodatnia. 

  W takim razie dla $0<\epsilon<\text{injrad}(M, g)$ mamy rozłączne kule w $\widetilde{M}$
  $$B_\epsilon(p)\cap B_\epsilon(\gamma\cdot p)=\emptyset$$
  dla dowolnego $\gamma\in\pi_1(M)-\{1\}$. W nakryciu jeśli mamy krzywą $c$ łączącą $p$ z $\gamma\cdot p$ to długość odpowiadającej jej pętli w $M$ $\pi(c)\geq\text{injrad}(M, g)$. Czyli długość samego $c$ też jest większa niż $\text{injrad}(M, g)$ a więc kule o mniejszym promieniu są rozłączne.

  Niech teraz $D:=\max\{d_{\widetilde{g}}(p, s\cdot p)\;:\;s\in S\}<\infty$. Można pokazać, że dla każdego $\gamma\in\pi_1M$ zachodzi
  $$d_{\widetilde{g}}(p, \gamma\cdot p)\leq D|g|_S.$$
  To oznacza, że dla dowolnego $m\geq 0$ kula $B_{Dm}(p)$ zawiera wszystkie punkty $\gamma\cdot p$ takie, że $|\gamma|_S\leq m$. Czyli jeśli dodamy do promienia $\epsilon>0$, to kula $B_{Dm+\epsilon}(p)$ zawiera te punkty $\gamma\cdot p$ jak i ich małe otoczenia $B_\epsilon(\gamma\cdot p)=\gamma\cdot B_\epsilon(p)$.

  Zatem 
  $$\vol_{\widetilde{g}}(B_{Dm+\epsilon}(p))\geq \beta_{\pi_1M, S}(m)\cdot\vol_g(B_\epsilon(p))$$
  $$\beta_{\pi_1M,S}(m)\leq\frac{1}{\vol_g(B_\epsilon(p))}\cdot \vol_{\widetilde{g}}(B_{Dm+\epsilon}(p))$$
  Co oznacza, że funkcja objętości $\vol_{\widetilde{g}}$ quasi-dominuje funkcję wzrostu $\beta_{\pi_1M,S}$.
  \bigskip

  Dowód odwrotnej quasi-dominacji zaczynamy od przypomnienia, że działanie $\pi_1M\acts \widetilde{M}$ jest kozwarte. Niech $D>0$ będzie takie, że 
  $$\bigcup_{\gamma\in\pi_1M}\gamma\cdot B_D(p)=\widetilde{M}\quad\left[=\bigcup_{\gamma\in\pi_1M}B_D(\gamma\cdot p)\right]$$
  Dla dowolnego $r>0$ kula $B_r(p)$ zawiera się w sumie kul $B_D(\gamma\cdot p)$ dla których $d(p, \gamma\cdot p)\leq r+D$:
  $$ B_r(p)\subseteq \bigcup\left\{ B_D(\gamma\cdot p)\;:\;d(p, \gamma\cdot p)\leq r+D \right\} $$

  Lemat Milnora-\v{S}varca \ref{lmm:M-S lemma} mówi, że odwzorowanie
  \begin{center}
    \begin{tikzcd}
      \pi_1M\ni \gamma\arrow[r, mapsto] & \gamma\cdot p\in(\widetilde{M}, d_{\widetilde{g}})
    \end{tikzcd}
  \end{center}
  jest quasi-izometrią. Niech $(L,C)$ będą stałymi z definicji q.i., czyli 
  $$d(p,\gamma\cdot p)\geq \frac{1}{C}|g|_S-L.$$
  Wtedy zbiór 
  $$\{\gamma\in\pi_1M\;:\;d(p,\gamma\cdot p)\leq r+D\}$$
  zawiera się w zbiorze
  $$\{\gamma\in\pi_1M\;:\;|g|_S\leq C(r+D+L)\}.$$
  W takim razie
  $$\vol_g(B_r(p))\leq\beta_{\pi_1M,S}(C\cdot r+C(D+L))\cdot \vol_g(B_D(p)),$$
  a więc funkcja wzrostu objętości jest quasi-zdominowana przez funkcję wzrostu $\beta_{\pi_1M,S}$ grupy $\pi_1M$.
\end{proof}

Funkcje wzrostu objętości w rozmaitościach $\widetilde{M}$ o ujemnej krzywiźnie są wykładnicze, zaś w rozmaitościach o nieujemnej są quasi-zdominowane przez wielomiany stopnia $\dim(M)$. Wynik Milnora kładł nacisk na własności wzrostu grup podstawowych $\pi_1M$ - traktowany jako analog wcześniej znanego faktu, że grupy podstawowe rozmaitości o dodatniej krzywiźnie są skończone.

