\section{30.10.2024}{Końce skończenie generowalnych grup - twierdzenie Freudanthala-Hopfa}


% $\xleftswishingghost{abc\dots z} $

Główne twierdzenie na dzisiaj:
\begin{theorem}{Freudanthal-Hopf}{}
  Skończenie generowalna grupa $G$ ma $0$, $1$, $2$ lub nieskończenie wiele końców.

  Gdy $|\Ends(G)|=\infty$, to $|\Ends(G)|$ jest przestrzenią bez punktów izolowanych - w szczególności mamy continuum. W istocie, $\Ends(G)$ jest wtedy zbiorem Cantora.
\end{theorem}

Zanim przejdziemy dalej, warto wiedzieć kilka rzeczy o zbiorze Cantora, np. jak jest on charakteryzowany w matematyce:
\begin{itemize}
  \item[$\pumpkin$] jest to {\slshape jedyna z dokładnością do homeomorfizmu przestrzeń metryczna, która jest całkowicie niespójna (0-wymiarowa)}, to znaczy, że każdy punkt posiada bazę otoczeń otwarto-domkniętych 
  \item[$\skull$] nie ma on punktów izolowanych.
\end{itemize}

Niech $X=(\Lambda, \mathcal{X}, \mathcal{F})$ będzie systemem odwrotnym zbiorów skończonych. Załóżmy, że wszystkie odwzorowania $f_{\lambda,\mu}\in\mathcal{F}$ są surjekcjami oraz $\forall\;\lambda\in\Lambda\;\forall\;x\in X\;\forall\mu>\lambda$ takie, że $|f^{-1}_{\lambda\mu}(x)|\geq 2$ to wówczas $\varprojlim \underline{X}$ jest homeomorficzny ze zbiorem Cantora. To znaczy, że $\underline{X}$ rozdziela się w każdym kroku na co najmniej dwie części dokładnie tak jak zbiór Cantora.

\begin{proof}
  Wiemy, że $|\Ends(G)|=0,1,2$ jest możliwe, bo $0$ końców mają grupy skończone, $1$ ma $\Z^2$, a $\Z$ ma końców $2$ sztuki.

  Załóżmy, że $|\Ends(G)|\geq3$. Oznacza to, że dla $X=Cay(G, S)$ istnieje zwarty $K\subseteq X$ taki, że $\Pi_K^X$ ma co najmniej $3$ elementy (tzn. $X-K$ ma co najmniej $3$ nieograniczone komponenty spójności). 

  Naszym celem jest pokazanie, że dla dowolnego $L\subseteq X$ zwartego i dowolnej nieograniczonej komponenty $C$ w $X-L$ istnieje większy zbiór $L\subseteq L'\subseteq X$ oraz nieograniczone komponenty $C_1'\neq C_2'$ w $\Pi^X_{L'}$ takie, że $C_1',C_2'\subseteq C$ (czyli $f_{LL'}(C_i)=C$ dla $i=1,2$). Jako ćwiczenie pozostawione zostanie pokazanie, że wówczas $|\Ends(G)|=\infty$ (to pokazuje, że nici sklejają się).

  Ustalmy zwarty $L\subseteq X$ oraz nieograniczoną komponentę $C$ w $X-L$. Niech $M\subseteq X$ będzie zbiorem z definicji kozwartości działania $G \acts X$, tzn. takim, że 
  $$\bigcup_{g\in G}gM=X.$$ 
  Bez straty ogólności załóżmy, że $K\subseteq M$, a co za tym idzie $|\Pi_M^X|\geq3$. 

  \begin{center}
    \begin{tikzpicture}
      \draw (-1, 1.5) to [out=-70, in=150] (0,0);
      \draw (0,0) to [out=-30, in=170] (2, -.7);

      \draw (-1, -3) to[out=60, in=170] (0, -2);
      \draw (0, -2) to[out=-10, in=120] (2, -3);

      \draw (0,0)--(0,-2);
      \draw (2, -.7)--(2,-3);
      \node at (1, -1.3) {$L$};

      \draw (2, -.7) to[out=-10, in=-90] (5, 1.5);
      \draw (2, -3) to[out=-60, in=55] (2.25, -4.5) to[out=-130, in=75] (1, -7); 
      \draw (3, -7) to[out=70, in=180] (4, -5) to[out=0, in=90] (5.5, -7);
      \draw (6.5, -7) to[out=100, in=-50] (4.5, -4);
      \draw (4.5, -4) to[out=130, in=-90] (4, -2) to[out=90, in=-90] (6, 1.5);

      \draw (4, -2) -- (2.25, -4.5) -- (4, -5) -- (4.5, -4);

      \node at (3.5, -4) {$g_0\cdot K$};

      \draw[<->] ($(4, -2)!0.5!(2.25, -4.5) + (-.2, 0)$)--(2.2, -1.35);
      \node[anchor=west] at (2.5, -2) {$\geq \text{diam}L$};

      \node at (3, -1) {$C_0$};
      \node at (2.7, -5.5) {$C_1$};
      \node at (5, -5) {$C_2$};
    \end{tikzpicture}
  \end{center}

  % {\Large\color{red}OBRAZEK}

  Niech $x_0\in C$ będzie takim punktem, że 
  $$d(x_0,L)\geq \text{diam}L+2\text{diam}M.$$ 
  Niech teraz $g_0\in G$ będzie taki, że $x_0\in g_0M$. Wtedy ponieważ $\text{diam}(g_0M)=\text{diam}(M)$, mamy 
  $$d_X(L, g_0M)\geq \text{diam}M$$ 
  ale też $\geq\text{diam}L$. Więc tym bardziej 
  $$d_X(L, g_0K)\geq \text{diam}M\geq \text{diam}K$$ 
  ale też $\geq\text{diam}L$.

  Twierdzimy, że 
  \begin{enumerate}
    \item $g_0K\subseteq g_0M\subseteq C$,
    \item $L$ zawiera się w dokładnie jednej spośród komponent $C_0,...,C_m$ w $X-g_0K$ ($m\geq0$), BSO w $C_0$,
    \item pozostałe komponenty $C_1,..., C_m$ w $X-g_0K$ zawierają się wtedy w $C$ (przynajmniej $2$ spośród nich są nieograniczone).
  \end{enumerate}
  % {\Large\color{red}no i spadlo mi sie z rowerka}

  \begin{description}
    \item[Dowód 1.] Wystarczy, że $g_0M\subseteq C$. Gdyby $x_1\in g_0M$ leżał w innej $C'\neq C$ komponencie $X-L$, to geodezyjna $[x_0, x_1]$ przechodziłaby przez $L$ (składowe spójności = składowe łukowej spójności), ale $d_X(x_0, x_1)\geq 2\cdot \text{diam}M$ dawałoby sprzeczność.
    \item[Dowód 2.] Argument analogiczny do 1., zastosowany symetrycznie.
    \item[Dowód 3.] Załóżmy, że komponenty $C_1,...,C_m$ są rozłączne z $L$, bo $L\subseteq C_0$. Więc każda z nich zawiera się w pojedynczej komponencie $X-L$. Każda spośród $C_1$,..., $C_m$ posiada punkty dowolnie bliskie zbioru $g_0K$, czyli należące do $C$. Np. pierwszy punkt na geodezyjnej od dowolnego punktu $a\in C_i$ do dowolnego punktu $b\in g_0K$ nienależący do $C_i$  musi należeć do $g_0K$. 

  Skoro $C_i$ zahacza o $C$, to musi być zawarte w $C$.

  \end{description}


  Dla ukończenia realizacji CELU (i dowodu twierdzenia) weźmy $L'=L\cup g_0K$. Wtedy $C_1,..., C_m$ są komponentami w $X-L'$, bo są rozłączne zarówno z $g_0K$ jak i z $L$. Wszystkie komponenty z $X-L'$ są zawarte w komponentach $X-g_0K$. Co najmniej $2$ z nich są nieograniczone, co daje nam szukane $C_1', C_2'$. 
\end{proof}

Dalsze wyniki:
\begin{enumerate}
  \item[$\skull$] Grupa ma $2$ końce $\iff$ jest wirtualnie $\Z$ (zawiera $\Z$ jako podgrupę skończonego indeksu $\equiv$ jest współmierna z $\Z$)
  \item[$\skull$] Jeśli $|\Ends(G)|=\infty$, to $G$ rozkłada się w sposób  nietrywialny i nie $2$-końcowy nad skończoną podgrupą $H$, tzn. 
    $$G=G_1\star_H G_2$$
    i $[G_i:H]\geq 3$ dla przynajmniej jednego $i$, lub 
    $$G=\star_H G_0$$
    (HNN-rozszerzenie), $\phi_i:H\hookrightarrow G_0$, $[G_0:\phi_i(H)]\geq 2$ dla pewnego $i$.
  \item[$\skull$] Iterowany proces rozkładów nad skończonymi podgrupami kończy się. Końcowe wektory maja $\leq 1$ końców i są w pewnym sensie jednoznaczne. Dla kończenie generowalnych grup nie jest to jednak prawdą.
\end{enumerate}

Jeśli więc skończone grupy można uznać za nieciekawe, to najciekawsze są grupy z jednym końcem (1-ended).

