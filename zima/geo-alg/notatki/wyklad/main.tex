%! TeX program = lualatex
\documentclass[twoside, a4paper, 12pt]{book}

\title{Algebraic geometry}
\author{\color{subtext1}a desperate attempt to avoid failure}
\date{}

\usepackage[pl, dates]{../../../../template}

\begin{document}
\frontmatter 
\maketitle
\thispagestyle{empty}
\setcounter{page}{0}
%
% \tableofcontents

\mainmatter

\pagestyle{fancy}

 
% \chapter{Schemes}

\section{Spectrum}

\begin{definition}{Zariski topology}{zariski top}
  Let $A$ be a commutative ring with unity and let $\Spec A$ be the set of all prime ideals of $A$ For any $I\subseteq A$ let us define sets
  $$V(I):=\{\mathfrak{p}\in\Spec A\;:\;I\subseteq \mathfrak{p}\}$$
  $$D(f):=\Spec A - V(fA)$$
  A topology on $\Spec A$ such that all sets $V(I)$ are closed and $D(f)$ are open is called the \buff{Zariski topology} on $\Spec A$.
\end{definition}

A \hl{prime ideal} is an ideal $P$ such that 
\begin{itemize}
  \item if $ab\in P$ then $a\in P$ or $b\in P$
  \item $P\subsetneq A$.
\end{itemize}
For $\mathfrak{p}\in\Spec A$ we have $\{\mathfrak{p}\}$ is a closed set $\iff$ $\mathfrak{p}$ is a maximal ideal in $A$.

\begin{proposition}{}{}
  For a ring $A$ the following hold true:
  \begin{enumerate}[label=\alph*)]
    \item for any $I,J\trianglelefteq A$ we have $V(I)\cup V(J)=V(I\cap J)$
    \item for a family of ideals $(I_\alpha)_\alpha$ we have $\bigcap_\alpha V(I_\alpha)=V(\sum_\alpha I_\alpha)$
    \item $V(A)=\emptyset$ and $V(0)=\Spec A$.
  \end{enumerate}
\end{proposition}


 


% \chapter{Zariski}

\section{04.10.2024}{Topologia Z, noetherowskość}

\subsection{Konwencje}

pierścień := pierścień przemienny z 1

homomorfizmy z definicji zachowują 1

Dla $A\subseteq R$ ideał przez niego generowany to $(A)=AR\triangleleft R$. Dla ideałów $I,J\triangleleft R$ znamy operacje $I+J$, $IJ$, $I\cap J$ i $\sqrt{I}$ jako radykał.

$R$-algebra to homomorfizm pierścieni $R\to S$, a homomorfizm $R$-algebr to strzałka $f$ taka, że diagram
\begin{center}
  \begin{tikzcd}
    S_1\arrow[rr, "f"] & & S_2\\ 
                       & R\arrow[ul]\arrow[ur]
  \end{tikzcd}
\end{center}
komutuje.

Jeśli $K$ to ciało, to $K\to R$ jest injekcją, czyli $K$-algebry można utożsamiać z rozszerzeniami ciała $K\subseteq R$. Dla rozszerzenia ciał $K\subseteq L$ definiujemy stopień przestępny $trdeg_K(L)=|B|$ dla $B\subseteq L$ będącego największym zbiorem liniowo niezależnym nad $K$.

Niech $K$ będzie ustalonym ciałem algebraicznie domkniętym, np. $\C$. Wtedy $A^n$ lub $A^n(K)$ to $K^n$ rozważane jako obiekt geometryczny. Będziemy to nazywać $n$-przestrzenią afiniczną, czyli $A^1=K$ to prosta afiniczna i $A^2=K^2$ - płaszczyzna afiniczna.

\subsection{Topologia Zariskiego}

\begin{definition}{zbiory Zariskiego}{}
  Dla dowolnego $A\subseteq K[\overline{X}]$, gdzie $\overline{X}=(X_1,..., X_n)$ definiujemy zbiór zer $A$ w $K^n$
  $$V(A):=\{\overline{a}\in K^n\;:\;(\forall\;F\in A)\;F(\overline{A}=0\}.$$
  Zbiory tej postaci nazywamy \buff{afinicznymi zbiorami algebraicznymi} lub \buff{zbiorami domkniętymi Zariskiego}.
\end{definition}

\begin{example}[m]
  \item Gdy popatrzymy na $A=\{y-x^2\}$ to zbiór zer jest parabolką, która jest spójna {\large\color{red}narysowac}
  \item dla $A=\{yx-1\}$ zbiór zer to hiperbola, która już spójna nie jest.
  \item Jeśli $F\in K[\overline{X}]$ jest nierozkładalny, to dla $n=2$ $V(F)$ jest \hl{krzywa planarna}, dla $n=3$ jest \hl{powierzchnia planarna} a dla $n>3$ jest \hl{hiperpowierzchnia planarna}.
  \item $\{\overline{a}\}$ singleton jest domkniętym zbiorem Zariskiego jako $V(X-a_1,..., X_n-a_n)$
  \item $\emptyset=V(1)$
  \item $A=V(0)$
\end{example}

\begin{lemma}{podwały topologii}{top}
  Jeśli $I,J\triangleleft K[\overline{X}]$ oraz $A_i\subseteq K[\overline{X}]$, to wtedy 
  \begin{enumerate}
    \item $A_0\subseteq A_1\implies V(A_1)\subseteq V(A_0)$
    \item $V(\bigcup A_i)=\bigcap V(A_i)$
    \item $V(A_0)=V((A_0))$, czyli zbiór rozwiązań zbioru jest taki sam jak zbiór rozwiązań jego ideału
    \item $V(I\cap J)=V(IJ)=V(I)\cup V(J)$
    \item $V(I+J)=V(I)\cap V(J)$
  \end{enumerate}
\end{lemma}

\begin{proof}
  1 i 2 są oczywiste. 
  
  Jedno zawieranie w punkcie 3 jest wnioskiem z 1, bo $A_0\subseteq (A_0)$, czyli $V(A_0)\subseteq V((A_0))$. Dla zawierania w drugą stronę bierzemy dowolne $\overline{a}\in V(A_0)$ oraz $F\in (A_0)$, chcemy pokazać $F(\overline{a})=0$. Ponieważ $A_0$ generuje ten ideał, to istnieją $F_1,..., F_k\in A_0$ oraz $H_1,..., H_k\in K[\overline{X}]$, że $F=\sum H_iF_i$.

  W 4 wiemy, że $I\cap J\supseteq IJ$, czyli $V(IJ)\supseteq V(I\cap J)\supseteq V(I)\cup V(J)$. Wystarczy pokazać, że $V(IJ)\subseteq V(I)\cup V(J)$ Weźmy więc $\overline{a}\in V(IJ)$ i załóżmy, że $\overline{a}\notin V(I)$, będziemy pokazywać $\overline{a}\in V(J)$. Niech $H\in J$ i $F\in I$. Czyli $FH\in IJ$. Ale $\overline{a}\in V(IJ)$, czyli $FH(\overline{a})$, ale skoro $\overline{a}\notin V(I)$, to $F(\overline{a})\neq 0$ czyli pozostaje $H(\overline{a})=0$.

  W ostatnim podpunkcie z 2 i 4 wiemy, że 
  $$V(I)\cap V(J)=V(I\cup J)=V((I\cup J))=V(I+J),$$
  bo $I\cup J=I+J$.
\end{proof}

\begin{conclusion}{}{}
  Z przykładu 5 i 6 i lematu \ref{lmm:top} wiemy, że zbiory domknięte Zariskiego są zbiorami domkniętymi pewnej topologii na $A^n$, nazywanej \buff{topologią Zariskiego}. Singletony są domknięte, czyli topologia Zariskiego jest $T_1$, ale nie jest Hausdorffa.
\end{conclusion}


\begin{example}
  Na $A^1=K$ niezerowe wielomiany mają zawsze skończenie wiele zer, czyli $V\subseteq A^1$ jest domknięty $\iff$ jest skończony lub jest wszystkim. Zbiory otwarte Zariskiego są natomiast koskończone lub puste, czyli przekrój dowolnych dwóch niepustych zbiorów otwartych jest niepusty.
\end{example}

\begin{remark}{}{}
  Dla $K=\C$ jest $A^n=\C^n=\R^{2n}$ i na $\R$ zwykłą topologię, którą na $\R^{2n}$ nazywamy \hl{euklidesową}, która jest znacznie bogatsza od topologii Zariskiego.

  \begin{center}
    \begin{tikzpicture}
      \draw (0,0) rectangle (3, 3);
      \node at (3, 3.5) {$\C$};
      \fill[green!40] (1, 1) rectangle (2, 1.5); 
      \node[anchor=north west] at (2.5, 2) {\begin{varwidth}{3cm}domknięty euklidesowo, ale nie domknięty Zariskiego\end{varwidth}};
      \draw [->] (2.6, 2.1) to[in=90, out=180-45] (1.8, 1.55);
    \end{tikzpicture}
  \end{center}
\end{remark}

\begin{remark}{}{}
  Topologia Zariskiego na $A^2=A^1\times A^1$ nie jest topologią produktową. Np. Parabola i prosta nie są domknięte w topologii produktowej.
\end{remark}

\subsection{Przestrzenie noetherowskie}

\begin{proposition}{}{}
  Dla wszystkich $A\subseteq K[\overline{X}]$ istnieje skończony $A_0\subseteq A$ taki, że $V(A_0)=V(A)$.
\end{proposition}

\begin{proof}
  Z twierdzenia Hilberta o bazie pierścień $K[\overline{X}]$ jest Noetherowski. Ideał generowany przez $A$ jest skończenie generowany. W takim razie istnieje $A_0$ wybrany z dowolnego skończonego zbioru generatorów i z \ref{lmm:top} wiemy, że $V(A_0)=V((A))=V(A)$.
\end{proof}

\begin{definition}{przestrzeń noetherowska}{}
  Mówimy, że przestrzeń topologiczna $X$ jest \buff{noetherowska}, jeśli każdy zstępujący ciąg zbiorów domkniętych się stabilizuje. To znaczy, że dla każdego
  $$...\subseteq X_n\subseteq X_{n-1}\subseteq...\subseteq X_0\subseteq X$$
  istnieje $N$ takie, że dla wszystkich $n\geq N$ $X_n=X_N$.
\end{definition}

\begin{remark}{}{}
  \begin{enumerate}
    \item  Jeśli $X$ jest noetherowska, to $X$ jest quasi-zwarta, ale niekoniecznie Hausdorffa.
    \item $X$ jest noetherowska i Hausdorffa $\iff$ $X$ jest skończona i dyskretna (punkty są otwarte).
    \item Z przykładu wyżej $A^1$ z topologią Zariskiego jest Noetherowska.
    \item Podprzestrzeń przestrzeni noetherowskiej jest nadal noetherowska.
  \end{enumerate}
\end{remark}

\begin{proposition}{}{}
  $A^n$ jest noetherowska 
\end{proposition}

\begin{proof}
  Niech $A^n\supseteq V_0\supseteq V_1\supseteq...$ będzie zstępującym ciągiem domkniętych zbiorów Zariskiego. Niech $A_i\subseteq K[\overline{X}]$ takie, że $V(A_i)=V_i$. Niech $I_i:=(A_0\cup...\cup A_i)$. Wtedy z \ref{lmm:top} 
  $$V(A_0\cup...\cup A_i)=V(A_0)\cap ...\cap V(A_i) = V(A_i)=V_i,$$
  bo to zbiory zstępujące.

  Teraz $I_0\subseteq I_1\subseteq ...$ jest wstępującym ciągiem w pierścieniu noetherowskim $K[\overline{X}]$, czyli stabilizuje się od pewnego momentu. W takim razie zbiory $V_i$ przez nie generowane też się stabilizują.
\end{proof}

\subsection{Przestrzenie nierozkładalne}

\begin{definition}{nierozkładalność}{}
  Niepusta przestrzeń topologiczna $X$ jest \buff{nierozkładalna}, gdy dla każdych $A, B\subsetneq X$ domkniętych $X\neq A\cup B$.
\end{definition}

\begin{remark}{}{}
  \begin{enumerate}
    \item nierozkładalna $\implies$ spójna
    \item nierozkładalna i $T_2$ $\implies$ singleton
    \item $A^1$ z topologią Zariskiego jest nierozkładalna
    \item $Y\subseteq X$ ($X$ potencjalnie noetherowska), to $Y$ jest nierozkładalny $\iff$ $\overline{Y}$ jest nierozkładalny
  \end{enumerate}
\end{remark}

\begin{proposition}{}{}
  Niech $X$ będzie noetherowską przestrzenią topologiczną. Wtedy 
  \begin{enumerate}
    \item istnieją $X_1,..., X_k\subseteq X$ domknięte, nierozkładalne, to wówczas $X=X_1\cup...\cup X_k$
    \item jeśli dla wszystkich $i\neq j$ $X_i\not\subseteq X_j$, to rozkład z punktu 1 jest jednoznaczny z dokładnością do permutacji.
  \end{enumerate}
\end{proposition}

\begin{proof}
  1. Prawie taki sam jak dowód faktu, że dla $r\in R-R^*$ w pierścieniu noetherowskim istnieją nierozkładalne $p_i$ takie, że $r=p_1...p_k$.

  Załóżmy nie wprost, że $X$ nie ma takiego rozkładu, wtedy $X$ nie może być nierozkładalny. W takim razie istnieją domknięte $A, B\subsetneq X$ takie, że $X=A\cup B$. Wtedy $A$ lub $B$ nie mają rozkładu, BSO $A$ nie ma. Powtarzamy ten tok rozumowania dla $A$. W ten sposób moglibyśmy dostać nieskończony, niestabilizujący się ciąg zstępujących zbiorów domkniętych, co jest sprzeczne z noetherowskością $X$.
\end{proof}



%
% \section{25.02.2025}{Produkty i koprodukty kategorii}

\subsection{O obiektach początkowych i końcowych słów kilka}

\begin{definition}{obiekt początkowy i końcowy}{}
  Powiemy, że obiekt $C\in \Cc_0$ jest \buff{początkowy}, jeśli dla każdego $D\in\Cc_0$ istnieje dokładnie jeden morfizm $C\to D$, $|\Cc(C, D)|=1$. Analogicznie definiujemy \buff{obiekt końcowy} $C$: $\forall\;D\in\Cc_0\;|\Cc(D, C)|=1$.
\end{definition}

\begin{example}[m]
  \item W kategorii, której obiektami jest odcinek $\Cc_0=[0,1]$, a morfizmy to relacja $\leq$ obiektem początkowym jest $0$, a końcowym - $1$.
  \item W kategorii zbiorów obiektem początkowym jest $\emptyset$, a obiektem końcowym jest singleton.
  \item W $Gr$ grupa trywialna jest zarówno obiektem początkowym jak i końcowym.
  \item Kategoria, która ma dwa obiekty bez morfizmów między nimi nie ma obiektu końcowego ani początkowego.
\end{example}

\begin{fact}{}{}
  Obiekty końcowe i początkowe, jeśli istnieją, to są jedyne z dokładnością do izomorfizmu.
\end{fact}

\begin{proof}
  Niech $C$ i $C'$ będą obiektami końcowymi kategorii $\Cc$. Wiemy, że $\Cc(C, C)=\{id_C\}$, czyli komutujący diagram
  \begin{center}
    \begin{tikzcd}
      C \arrow[rr, "id_C"]\arrow[dr, "\exists!f" below left] & & C\\ 
                           & C'\arrow[ur, "\exists!g" below right]
    \end{tikzcd}
  \end{center}
  daje $g\circ f=id_C$. Analogiczny diagram daje $f\circ g=id_{C'}$. Stąd $f$ i $g$ to para wzajemnie odwrotnych izomorfizmów między $C$ i $C'$
\end{proof}

\subsection{(Ko)granice funktorów a (ko)produtky}

Niech $F:\mathcal{I}\to \Cc$ będzie funktorem, gdzie o kategorii $\mathcal{I}$ myślimy jako o kategorii indeksów. Przez $\Cc^{\mathcal{I}}$ oznaczmy kategorię wszystkich takich funktorów. 
Powiemy, że funktor $C$ jest stały, jeżeli $C(i)=C$ dla każdego $i\in\mathcal{I}_0$ oraz $C(f)=id_C$ dla każdego morfizmu.

Budujemy kategorię, której 
\begin{itemize}
  \item obiekty to wszystkie naturalne przekształcenia funktora $F$ w funktory stałe $C$, $\phi:F\implies C$, czyli komutujące diagramy (kostożki) 
    \begin{center}
      \begin{tikzcd}
        F(i)\arrow[rr, "F(f)"]\arrow[dr, "\phi_i" below left] & & F(j)\arrow[dl, "\phi_j"]\\ 
                                                  & C
      \end{tikzcd}
    \end{center}
  \item a morfizmy to strzałki $C\to D$ takie, że diagram
    \begin{center}
      \begin{tikzcd}
        C\arrow[rr] & & D\\ 
                    & F\arrow[ur, Rightarrow, blue, "\phi" below right]\arrow[ul, Rightarrow, orange, "\psi" below left]
      \end{tikzcd}
    \end{center}
    komutuje.
\end{itemize}

Diagram wyżej można rozpisać jako:
\begin{center}
  \begin{tikzcd}[column sep=large]
    & F(i)\arrow[d] \arrow[ddl, "\phi_i" above left, blue]\arrow[ddr, "\psi_i" above right, orange] \\ 
    & F(j)\arrow[dl, "\phi_j" above, blue]\arrow[dr, "\psi_j" above, orange]\\ 
    D & & C\arrow[ll]
  \end{tikzcd}
\end{center}

\begin{definition}{kogranica funktora}{}
  \buff{Kogranicą} (\acc{granica prosta}) funktora $F$, $\varinjlim F$, nazywamy obiekt początkowy w wyżej zdefiniowanej kategorii naturalnych przekształceń. 
  % \buff{Granica} (\acc{granica odwrotna}) to wtedy obiekt końcowy powyższej kategorii ze wszystkimi strzałkami zdualizowanymi $\varprojlim F$.
\end{definition}

Diagram wyżej możemy zdualizować i zamiast rozpatrywać naturalne przekształcenia $\phi:F\implies C$ możemy rozważyć naturalne przekształcenia $\phi:C\implies F$, czyli diagramy (stożki)
\begin{center}
  \begin{tikzcd}
    & C \arrow[dl, "\phi_i" above left] \arrow[dr, "\phi_j" above right]\\ 
    F(i)\arrow[rr, "F(f)" below] & & F(j)
  \end{tikzcd}
\end{center}
z morfizmami {definiowanymi analogicznie. 

\begin{definition}{granica funktora}{}
  \buff{Granica} (\acc{granica odwrotna}) to obiekt końcowy powyższej kategorii stożków, $\varprojlim F$.
\end{definition}

% {\color{red}tutaj jest zdjecie
%
% przyklad dla kategorii zbiorów
%
% ja chyba chce wziąć dwuelementową kategorię $\mathcal{I}$ i tutaj policzyć, jeśli $F(1)=G$, a $F(2)=H$.
% }
%
Rozważmy kategorię $\mathcal{I}$, która ma dwa obiekty $\mathcal{I}_0=\{0,1\}$. Niech $F:\mathcal{I}\to Set$ będzie funktorem, dla którego $F(0)=A$, a $F(1)=B$. Niech $\phi$ oraz $\psi$ będzie parą naturalnych przekształceń, dla których
\begin{center}
  \begin{tikzcd}[column sep=large, row sep=large]
     & \varinjlim F\arrow[dl, "\phi_0" above left] \arrow[dr, "\phi_1"] \\ 
    F(0)=A & D \arrow[l, "\psi_0"] \arrow[r, "\psi_1" below right] \arrow[u, "\exists!f", dashed] & F(1)=B
  \end{tikzcd}
\end{center}
gdzie pionowa strzałka istnieje i jest jedyna, bo $\varinjlim F$ to obiekt końcowy. Jeśli weźmiemy $\varinjlim F=A\times B$, a $\phi_0=\pi_A$ oraz $\phi_1=\pi_B$ będą rzutami i $f(d)=(\psi_0(d), \phi_1(d))$, to diagram nadal jest prawdziwy. 

Granica odwrotna tego samego funktora, to z kolei suma rozłączna $A\sqcup B$, bo diagram
\begin{center}
  \begin{tikzcd}[column sep=large, row sep=large]
    F(0)=A\arrow[r, "\psi_0"]\arrow[dr, "\phi_0=i_A" below left] & D & F(1)=B\arrow[l, "\psi_1" above]\arrow[dl, "\phi_1=i_B"]\\ 
                                                       & \varprojlim F= A\sqcup B \arrow[u, dashed, "\exists!f"]
  \end{tikzcd}
\end{center}
gdzie $f(x)=\phi_0(x),$ jeśli $x\in A$ oraz $f(x)=\psi_1(x)$ jeśli $x\in B$, komutuje.

\begin{definition}{(ko)produkt}{}
  \buff{Produktem} obiektów $A$ i $B$ kategorii $\Cc$ nazywamy granicę prostą (kogranicę) funktora $F:\mathcal{I}\to \Cc$ dla $\mathcal{I}$ oraz $F$ jak wyżej.

  \buff{Koproduktem} obiektów $A$ i $B$ kategorii $\Cc$ nazywamy granicę odwrotną (granicę) funktora $F:\mathcal{I}\to\Cc$
\end{definition}

\begin{example}[m]
  \item W kategorii grup produkt to iloczyn kartezjański dwóch grup, tak jak w kategorii zbiorów, tj. dla grup $A,G,H$ komutuje diagram
    \begin{center}
      \begin{tikzcd}[column sep=large, row sep=large]
        & G\times H\arrow[dl, "\pi_G" above left]\arrow[dr, "\pi_H"]\\ 
        G & A\arrow[l, "g"]\arrow[r, "h"]\arrow[u, "g\times h" below] & H
      \end{tikzcd}
    \end{center}
    Koprodukt to z kolei produkt wolny tych dwóch grup:
\begin{center}
  \begin{tikzcd}[column sep=large, row sep=large]
    G\arrow[r, "g"]\arrow[dr, "i_G" below left] & A & H\arrow[l, "h" above]\arrow[dl, "i_H"]\\ 
                                                       & H\ast G \arrow[u, dashed, "\exists!f"]
  \end{tikzcd}
\end{center}
gdzie $f$ nakłada na litery słów $G\ast H$ pochodzące z $G$ morfizm $g$, a na litery pochodzące z $H$ - morfizm $h$.
  \item Niech $F:\mathcal{I}\to (P, \leq)$ z dwuobiektowej kategorii $\mathcal{I}$ w zbiór uporządkowany. Wtedy jeśli mamy diagram 
    \begin{center}
      \begin{tikzcd}
         & \varinjlim F\arrow[dr]\arrow[dl] \\ 
        F(0)=a & d \arrow[l]\arrow[r]\arrow[dashed, u] & F(1)=b
      \end{tikzcd}
    \end{center}
    to znaczy, że $d\leq a$, $d\leq b$ oraz $d\leq \varinjlim{F}$. Żeby więc miało to sens dla dowolnego $d\leq a,b$ to $\varinjlim F=\inf\{a,b\}$. Analogicznie dostajemy, że $\varprojlim F=\sup\{a,b\}$.

  \item Jeśli $\mathcal{I}$ jest kategorią o nieskończenie wielu obiektach bez morfizmów między różnymi obiektami, a $F:\mathcal{I}\to Set$ jest funktorem w kategorię zbiorów, to wówczas kogranicą tego funktora jest nieskończony iloczyn kartezjański $\prod_{i\in\mathcal{I}_0}F(i)$, a granicą - nieskończona suma rozłączna $\bigsqcup_{i\in\mathcal{I}_0}F(i)$.
\end{example}

% kategoria nieskończenie wiele elementów, ale bez strzałek (jako $\mathcal{I}$)
 % Niech $C$ oraz $C'$ będą kogranicami tego samego funktora. Z definicji mamy
% \begin{center}
%   \begin{tikzcd}[column sep=large, row sep=large]
%     & F(i)\arrow[dr, "\phi_i"]\arrow[d, "\psi_i"]\arrow[dl, "\phi_i" above left] \\ 
%     C & C'\arrow[l, "\exists g" above] & C\arrow[ll, bend left=20, "id"] \arrow[l, "\exists f" above]
%   \end{tikzcd}
% \end{center}

\begin{fact}{}{}
  Granica i kogranica funktora, jeśli istnieje, to jest jedyna z dokładnością do izomorfizmu. Stąd również produkty i koprodukty są unikalne.
\end{fact}

\begin{proof}
  Wynika z uniwersalności obiektów końcowych i początkowych.
\end{proof}

% tutaj liczby p-adyczne
% ekwalizator, koekwalizator
%
% \begin{definition}{surjekcja, epimorfizm}{}
%   Jeśli kategoria ma obiekt początkowy równy obiektowi końcowemu...
% \end{definition}

\begin{example}
  Rozważmy funktor $F:\mathcal{I}^{op}\to Grp$, gdzie $\mathcal{I}=(\N, \leq)$ taki, że dla każdych $i,j\in\N$, $i\leq j$ mamy
  \begin{center}
    \begin{tikzcd}[column sep=large]
      F(j)=\Z/p^j\Z\arrow[r, "F(i\to j)=q"] & F(i)=\Z/p^i\Z
    \end{tikzcd}
  \end{center}
  gdzie $q$ to morfizm ilorazowy.

  Liczby $p$-adyczne to rozszerzenie liczb wymiernych różne od liczb rzeczywistych i zespolonych. Całkowite liczby $p$-adyczne to szeregi
  $$\sum_{i=k}^\infty a_ip^i,$$
  gdzie $k\in\N$ oraz $0\leq a_i < p$. Okazuje się, że całkowite liczby $p$-adyczne, $\Z_p$, można zdefiniować jako granicę funktora $F$:
  \begin{center}
    \begin{tikzcd}
      & & \Z_p \arrow[dll]\arrow[dl]\arrow[d]\arrow[drr]\arrow[drrr] \\ 
      ...\arrow[r] & \Z/p^n\Z\arrow[r] & \Z/p^{n-1}\Z\arrow[r] & ... \arrow[r]& \Z/p^2\arrow[r] & \Z/p\Z
    \end{tikzcd}
  \end{center}
  Granica prosta takiego funktora jest trywialna, ale możemy rozważyć inny funktor,z kategorii $\Z$ z porządkiem, tzn: $G:\Z\to Grp$ taki, że $G(n)=\Z/p^n\Z$, natomiast strzałkę $n+1\to n$ przekształcamy na odwzorowanie
  $$\Z/p^n\Z\ni x\mapsto p\cdot x\in \Z/p^{n+1}\Z.$$
  Wtedy granicą prostą $G$ jest $C_{p^\infty}$ - pierwiastki $p^n$-tego stopnia z $1$, dla dowolnego $n$. 
\end{example}

\subsection{Obiekty i kategorie monoidalne}

\buff{Monoid} $(M, \star, 1)$ to struktura algebraiczna z binarną operacją oraz elementem neutralnym. Dodatkowo, komutować ma diagram 
\begin{center}
  \begin{tikzcd}
    M^3\arrow[r, "\star\times id"]\arrow[d, "id\times\star" left] & M^2\arrow[d, "\star"]\\ 
    M^2\arrow[r, "\star"] & M
  \end{tikzcd}
\end{center}
co znaczy, że działanie jest łączne.

\begin{definition}{obiekt monoidalny, kategoria monoidalna}{}
  Niech $\Cc$ będzie kategorią z produktem i elementem początkowym. Niech $M\in \Cc$ będzie obiektem, dla którego mamy $\mu:M^2\to M$ oraz $\epsilon: \{1\}\to M$ takie, że komutują diagramy
  \begin{center}
    \begin{tikzcd}[row sep=large, column sep=large]
      M^3\arrow[r, "\mu\times id"]\arrow[d, "id\times \mu" left] & M^2\arrow[d, "\mu"]\\ 
      M^2\arrow[r, "\mu" below] & M
    \end{tikzcd}
  \end{center}
  \begin{center}
    \begin{tikzcd}[row sep=large, column sep=large]
      M\arrow[r, "\epsilon\times id"]\arrow[d, "id\times \epsilon" left]\arrow[dr, "=" above right] & M^2\arrow[d, "\mu"]\\ 
      M^2\arrow[r, "\mu" below] & M
    \end{tikzcd}
  \end{center}
  Wtedy $M$ jest \buff{obiektem monoidalnym}.
  
  Obiekt monoidalny w kategorii $Cat$ nazywa się \buff{kategorią monoidalną}.
\end{definition}

\begin{example}[m]
\item Dowolna kategoria $\Cc$ z koproduktem i obiektem końcowym jest kategorią monoidalna.
\item Kategoria endofunktorów ma strukturę monoidalną. To znaczy, jeśli mamy dwa endofunktory $F, G\in End(\Cc)$, to potrafimy je złożyć w dobry sposób.

  Funktor $T\in End(\Cc)$ oraz dwa naturalne przekształcenia $\mu:T^2\to T$, $\epsilon: Id\to T$, nazywa się \buff{monadą}.
\end{example}

% Czy $S^n\vee S^n$ to produkt czy produkt w kategorii $Toph_\star$. tutaj jakies zdjecie








\chapter{Blitzkrieg z GA}

\section{}{Noetherowskość, wymiary}

For any $A\subseteq K[\overline{X}]$ we define
$$V(A)=\{\overline{x}\in K^n\;:\;\forall\;F\in A\;F(\overline{x})=0\}.$$
Let $I,J\trianglelefteq K[\overline{X}]$. Then
\begin{itemize}
  \item $A_0\subseteq A_1\implies V(A_1)\subseteq V(A_0)$
  \item $V(\bigcup A_i)=\bigcap V(A_i)$
  \item $V(I\cap J)=V(IJ)=V(I)\cup V(J)$
  \item $V(I+J)=V(I)\cap V(J)$
\end{itemize}

For any $A\subseteq K[\overline{X}]$ there is a finite $A_0\subseteq A$ such that $V(A)=V(A_0)$ (by the Hilbert's basis theorem).

\begin{definition}{Noetherian ...}{}
  A topological space $X$ is called \buff{Noetherian} if any descending chain of closed subsets of $X$ stabilizes. Meanwhile, a ring is Noetherian if any ascending chain of ideals stabilizes.
\end{definition}

\begin{definition}{irreducible space}{}
  A space $X$ is irreducible if it is not a non-trivial union of its two closed subseteq, i.e. for any $Y_1,Y_2\subseteq X$ closed $X=Y_1\cup Y_2$ then $X=Y_1$ or $Y_2$.
\end{definition}

If $X$ is a Noetherian space then
\begin{itemize}
  \item $X=X_1\cup...\cup X_k$ for some $X_1,...,X_k\subseteq X$ irreducible such that $X_i\not\subseteq X_j$ for $i\neq j$
  \item this sequence is unique up to permutation of indices
\end{itemize}

An affinie algebraic set $V(A)$ is \buff{affine variety} if it is irreducble as a topological space with the Zariski topology.

\begin{definition}{dimension}{}
  $\dim(X)=n$ if there is a strictly decreasing sequence of irreducible closed subsets of $X$ such that 
  $$X_n\subsetneq X_{n-1}\subsetneq...\subsetneq X_0\subseteq X$$
\end{definition}

For $V\subseteq \mathbb{A}^n$ we define the \buff{affine coordinate ring of $V$} as
$$K[v]:=\{f\in \Func(V,K)\;:\;\exists F\in K[\overline{X}]\;:F|_V=f\}$$
$K(V)$ is the field of fractions of $K[V]$, called \buff{the field of rational functions on $V$}.

The \buff{ideal of $V$} is defined as 
$$\ker\left(K[\overline{X}]\ni F\mapsto F|_V\in K[V]\right)$$
which is the same as
$$I(V)=\{F\in K[\overline{X}]\;:\;\forall\;\overline{x}\in V\;F(\overline{x})=0\}.$$
The \buff{Zariski closure} of $V_0$ is the set $V(I(V_0))$.

\begin{theorem}{Hillbert's Nullstellensatz}{}
  \begin{description}
    \item[weak] $I\trianglelefteq K[\overline{X}]\;\land;I\neq K[\overline{X}]\implies V(I)\neq\emptyset$
    \item[strong/regular] $I\trianglelefteq K[\overline{X}]\implies I(V(I))=\sqrt{I}$
  \end{description}
\end{theorem}

If $V\subseteq \mathbb{A}^n$ is Zariski closed, then the following are equivalent
\begin{enumerate}
  \item $V$ is irreducible
  \item $I(V)$ is prime
  \item $\exists P\trianglelefteq K[\overline{X}]$ prime such that $V=V(P)$ !!the field $K$ needs to be algebraically closed!!
  \item $K[V]$ is a domain
\end{enumerate}

\begin{theorem}{}{}
  If $F\in K[\overline{X}]$ is irreducible, then $V(F)$ is an affine variety.
\end{theorem}

Let $V\subseteq\mathbb{A}^n$ be an affine algebraic set. Then there is a bijection between $\begin{matrix}\text{radical}\\\text{prime}\end{matrix}$ ideals of $K[V]$ and the set of Zariski $\begin{matrix}\text{closed}\\\text{irreducible closed}\end{matrix}$ subsets of $V$.

\begin{definition}{Krull dimension}{}
  For a ring $R$ we define its Krull dimension $\dim(R)$ as the supremum $k\in\N$ such that there is a strictly increasing (or decreasing) sequence of prime ideals
  $$P_0\subsetneq P_1\subsetneq ...\subsetneq P_k$$
  of $R$.
\end{definition}
{\large\boldmath$$\color{blue}\dim(V)=\dim(K[V])$$}

\begin{theorem}{}{}
  Let $R$ be a finitely generated $K$-algebra which is a domain. Then
  $$\dim(R)=\text{trdeg}_K(R_0)$$
  the dimension of $R$ is equal to the transcendental degree of the field of fractions of $R$ over $K$.
\end{theorem}

If $R=K[V]$ then $R_0=K(V)$ and the above statement holds.

\section{}{Kategoryje}

Let $V\subseteq \mathbb{A}^n$ and $W\subseteq \mathbb{A}^m$ be affine algebraic sets. Then $\phi:V\to W$ is a \buff{morphism} if there are $f_1,...,f_m\in K[V]$ such that
$$\phi(v)=(f_1(v),...,f_m(v)).$$

For such a morphism we define
$$\phi^*:K[W]\to K[V]$$
$$\phi^*(f)=f\circ \phi.$$
\begin{enumerate}
  \item The mapping $\phi\mapsto \phi^*$ is an isomorphism between the set of morphisms $V\to W$ and the set of morphisms $K[W]$ to $K[V]$.
  \item Any finitely generated $K$-algebra $R$ which is reduced (no nilpotent elements) is isomorphic over $V$ to $K[V]$ for some affine algebraic $V$.
\end{enumerate}

{\color{blue}\large\boldmath
$$V\cong W\iff K[V]\cong_K K[W]$$
}

For $f\in K(V)$, the \buff{domain} of $f$, $\dom f$, is the set of points $v\in V$ such that there are $f_1, f_2\in K[V]$, $f=\frac{f_1}{f_2}$ and $f_2(v)\neq0$.

\begin{definition}{}{}
  Let $f\in K(V)$ and $v\in V$.
  \begin{enumerate}
    \item $f$ is regular at $v$ if $v\in \dom f$
    \item $\Oo_{V,v}:=\{f\in K(V)\;:\;v\in \dom f\}$
    \item $f$ is regular if $f$ is regular at each $v\in V$, i.e. $\dom f=V$.
  \end{enumerate}
\end{definition}

\begin{fact}{}{}
  For any $v\in V$
  $$\Oo_{V,v}=K[V]_{I_V(v)}=\{\frac{a}{b}\;:\;a\in K[V],b\in K[V]-I_V(v)\}$$
\end{fact}

Denote by
$$\mm_{V,v}\trianglelefteq \Oo_{V,v}$$
the maximal ideal of $\Oo_{V,v}$.

$f$ is regular $\iff$ $f\in K[V]$

\begin{lemma}{}{}
  For a morphism $\phi:V\to W$ its dual $\phi^*$ is a monomorphism $\iff$ $\phi$ is \buff{dominant}, i.e. $\phi(V)$ is Zariski dense in $W$ (is an epimorphism in its category).
\end{lemma}

A function $\phi:U\subseteq V\to W$ is a \buff{rational function} between $V$ and $W$ if there are $f_1,..., f_m\in K(V)$ such that 
$$U=\dom f_1\cap...\cap \dom f_m$$
and for all $v\in U$ there is $\phi(v)=(f_1(v),..., f_m(v))$.

$\phi:V\dasharrow W$ denotes a \buff{dominant rational function} from $V$ to $W$.

For any field extension $K\subseteq L$ such that $L$ is finitely generated over $K$ there is an affine variety $V$ such that $L\cong_K K(V)$.

\textbf{The category of affine varieties and dominant rational maps is \emph{entiequivalent} or \emph{dually equivalent} to the category of finitely generated field extensions of $K$}.

\section{}{Smooooth \emph{like the fur of a newborn goat}}

Let $R$ be a ring. The map $\partial R\to R$ is called a \buff{derivation} on $R$ if for all $a,b\in R$ 
$$\partial(a+b)=\partial(a)+\partial(b)$$
$$\partial(ab)=\partial(a)b+a\partial(b).$$

The \buff{Jacobian matrix} of $\overline{F}=(F_1,...,F_m)$, $F_i\in K[\overline{X}]$ is 
$$J_{\overline{F}}:=\begin{pmatrix}\frac{\partial F_1}{\partial X_1} & \dots & \frac{\partial F_1}{\partial X_n}\\\vdots & \ddots & \vdots \\ \frac{\partial F_m}{\partial X_1}&\dots &\frac{\partial F_m}{\partial X_n}\end{pmatrix}$$

\begin{fact}{}{}
  If $(G_1,..., G_k)=I=(F_1,..., F_m)\trianglelefteq K[\overline{X}]$ and $v\in V(I)$, then 
  $$\rank(J_{\overline{G}}(v))=\rank(J_{\overline{F}}(v)).$$
\end{fact}

\begin{definition}{non-singular variety}{}
  Let $V\subseteq\A^n$ and $F_1,..., F_m\in I(V)=(F_1,..., F_m)$. We say that $a\in V$ is a \buff{non-singular} or smooth point of $V$ if 
  $$\rank(J_{\overline{F}}(a))=n-\dim(V).$$
  We say that $V$ is a non-singular variety or a smooth variety if $V$ is irreducible and all points of $V$ are smooth.
\end{definition}

$F\in K[X,Y]$ and $V=V(F)\subseteq \A^2$
\begin{enumerate}
  \item $F\notin K\implies |V|=\infty$
  \item $|V(F, \frac{\partial F}{\partial X}, \frac{\partial F}{\partial Y})|<\infty\implies\sqrt{(F)}=(F)\;\land\;I(V)=(F)$
  \item $V(F, \frac{\partial F}{\partial X}, \frac{\partial F}{\partial Y})=\emptyset \implies V$ is smooth 
\end{enumerate}

\begin{lemma}{}{}
  A point $a\in V$ is smooth $\iff$ $\dim_K(I_V(a)/I_V(a)^2)=\dim(V)$.

  If $V$ is an affine variety and $a\in V$, then we have
  $$I_V(a)/I_V(a)^2\cong_K\mm_{V,a}/\mm_{V,a}^2$$
\end{lemma}

% $(R, \mm)$ is a DVR and $r\in R$. 

A Noetherian local ring $(R, \mm)$ is \buff{regular} if $\dim(R)=\dim_{R/\mm}(\mm/\mm^2)$.

The $K$-vector space $\mm_{V,a}/\mm_{V,a}^2$ is the \buff{cotangent space} of $V$ at $a$. The dual space is the \acc{tangent space}.

\section{}{DVR}

A local ring $(R,\mm)$ is a \buff{discrete valuation ring} if
\begin{enumerate}
  \item $R$ is Noetherian domain
  \item $R$ is not a field
  \item $\mm$ is principal (generated by a single element)
\end{enumerate}

In any Noetherian domain $R$ and any $I\triangleleft R$ we have
$$\bigcup_{n\geq 1}I^n=\{0\}.$$

{\color{blue}\large\bfseries\centering Any DVR is PID (the generator of $\mm$ is the uniformizing parameter)}

Let $R$ be a UFD, $r\in R$ irreducible and $R_0$ be the field of fractions of $R$. Define
$$v_r:R_0^*\to \Z$$
$$v_r(r^n\frac{a}{b})=n,\quad r\not| a,\;r\not| b.$$
We call $v_r$ the \buff{r-addic valuation} on $R_0$.

\begin{fact}{}{}
  $R$, $r$, $R_0$ and $v_r$ as above. Then for all $\alpha, \beta\in R_0^*$
  \begin{enumerate}
    \item $\alpha+\beta\in R_0^*\implies v_r(\alpha+\beta)\geq \min\{v_r(\alpha), v_r(\beta)\}$
    \item $v_r(\alpha\beta)=v_r(\alpha)+v_r(\beta)$
    \item $v_r(R_0^*)=\Z$
  \end{enumerate}
\end{fact}

For any irreducible $r,s\in R$ if $(r)=(s)$ then $v_r=v_s$.

\begin{definition}{discrete valuation}{}
  Let $L$ be a field. Any function $v:L^*\to \Z$ satisfying 1-3 from the fact above is called a \buff{(discrete) valuation} on $L$. For any valuation $v:L^*\to\Z$ 
  \begin{itemize}
    \item $\Oo_v:=\{\alpha\in L^*\;:\; v(\alpha)\geq 0\}\cup \{0\}$ -> \acc{valuation ring} of $v$
    \item $\mm_v:=\{\alpha\in L^*\;:\; v(\alpha)>0\}\cup \{0\}$ -> \acc{valuation ideal} of $v$
  \end{itemize}
\end{definition}

For a valuation $v:L^*\to \Z$, $(\Oo_v,\mm_n)$ is a DVR.

\begin{theorem}{}{}
  Let $C$ be an affine curve and $a\in C$. Then $a$ is smooth $\iff$ $(\Oo_{C,a},\mm_{C,a})$ is a DVR.
\end{theorem}

\begin{definition}{}{}
  Let $C$ be an affine curve and $a\in C$ be a smooth point
  \begin{enumerate}
    \item a uniformizing parameter $f\in\Oo_{C,a}$ is a \buff{local parameter} for $C$ at $a$
    \item the unique valuation on $K(C)$ given by $(\Oo_{C,a},\mm_{C,a})$ is denoted $\ord_a$
    \item for $f\in K(C)-\{0\}$ and $n\in\N_{>0}$
      \begin{itemize}
        \item $\ord_a(f) =n\implies $ $f$ has a zero at $a$ of order $n$
        \item $\ord_a(f)=-n\implies$ $r$ has a pole at $a$ of order $n$
      \end{itemize}
  \end{enumerate}
\end{definition}
$$\ord_a(f)=\dim_K(\Oo_{C,a}/f\Oo_{C,a})$$
$$\dim_K(K[X]/(F))=\sum_{a\in V(F)}\ord_a(F)$$

\begin{definition}{intersection number}{}
  The intersection number of $F$ and $G$ at $a=(x,y)\in \A^2$ is
  $$I(a, F\cap G):=\dim_K(\Oo/(F,G)\Oo),$$
  where 
  $$\Oo:=K[X,Y]_{(X-x, Y-y)}=K[X,Y]_{I(a)}=\Oo_{\A^2,a}$$

  For curves $C_1, C_2$ such that $F=I(C_1)$ and $G=I(C_2)$ we define $I(a, C_1\cap C_2):=I(a, F\cap G)$.
\end{definition}

$$I(a, F\cap G)>0\iff a\in V(F,G)$$
$$|V(F,G)|<\infty\implies I(a, F\cap G)<\infty$$
\begin{center}
  $F$ irreducible and $a\in V(F)$ smooth then 
  $$I(a, F\cap G)=\ord_a(G|_{V(F)}).$$
\end{center}

$$I(a, F\cap G)=I(a, F\cap (G+HF). G,F,H\in K[X,Y]$$
$$I(a, F\cap GH)=I(a, F\cap G)+I(a, F\cap H)$$
$$I(a, F\cap G)>0\iff a\in V(F, G)$$


$L_X=V(Y)$, $L_Y=V(X)$, $C_1=V(Y^2-x^3)$, $C_2=V(Y-X^3)$
\begin{itemize}
  \item $I(0, L_X\cap C_1)=\ord_0((Y^2-X^3)|L_X)=3$
  \item $I(0, L_Y\cap C_1)=\ord_0((Y^2-X^3)|L_Y)=2$
  \item $I(0, L_X\cap C_2)=\ord_0((Y-X^3)|L_X)=3$
  \item $I(0, L_X\cap C_2)=\ord_0((Y-X^3)|L_Y)=1$
\end{itemize}

\begin{definition}{}{}
  $C$ - a plane curve, $a\in\A^2$

  \begin{itemize}
    \item $L\subseteq \A^2$ \buff{line} if $\exists \alpha, \beta, \gamma\in K$ such that $L=V(\alpha X+\beta Y+\gamma)$ and $(\alpha, \beta)\neq(0,0)$
    \item $L\subseteq\A^2$ is \buff{tangent} to $C$ at $a$ if $I(a, L\cap C)>1$
    \item $T_aC$ is the union of all tangent lines to $C$ at $a$
  \end{itemize}
\end{definition}

\begin{lemma}{}{}
  $R$ - ring, $P\trianglelefteq R$ prime, $e\in R-P$ idempotent divisible by every element of $R-P$

  $\phi: R\to R_P$ induces an isomorphism of rings $eR\cong R_P$ preserving the unit elements.
\end{lemma}

\begin{theorem}{}{}
  $V=V(F,G)$ is finite
  $$\dim_K(K[X,Y]/(F,G))=\sum_{a\in V}I(a, F\cap G)$$
\end{theorem}

\textbf{poprosić o notatki?}

\section{}{Projektywizujem siem}

If $x=[a_1:...:a_{n+1}]\in\Proj^n$, then $a_1,..., a_{n+1}$ are called the \buff{projective} or \buff{homogenous coordinates} of $x$.

\begin{definition}{homogenous polynomial}{}
  $d, k, d_1,..., d_k\in\N$ and $H\in K[X_1,..., X_k]$
  
  \begin{itemize}
    \item $H=aX_1^{d_1}...X_k^{d_k}$ is a monomial $\implies$ $\deg H=d_1+...+d_k$
    \item $H$ is \buff{homogenous polynomial of degree} $d$ if $H$ is a sum of monomials of degree $d$
  \end{itemize}
\end{definition}

$H$ of degree $d$ is homogenous $\iff$ $\forall \lambda\in K$ $H(\lambda X_1,..., \lambda X_k)=\lambda^dH$.

\begin{definition}{projective algebraic}{}
  $V\subseteq \Proj ^n$ is a \buff{projective algebraic} set if there are homogenous polynomials $F_1,..., F_k\in K[X_1,..., X_{n+1}]$ such that 
  $$V=\{x\in\Proj^n\;:\;F_1(x)=0,..., F_k(x)=0\}$$
\end{definition}

$$\psi_i:\A^n\to \Proj^n$$
$$\psi_i(a_1,..., a_n)=[a_1:...:a_{i-1}:1:a_i:...:a_n]$$
$$U_i=\{[a_1:...:a_{n+1}\in\Proj^n\;:\;a_i\neq0\}$$

A line in $\Proj^2$ is a subset $V$ such that there exists $(\alpha,\beta,\gamma)\in K^3-\{0\}$ such tat 
$$V=\{[a:b:c]\;:\;\alpha a+\beta b+\gamma c=0\}.$$

\begin{enumerate}
  \item $\Proj^n$ with topology defined by closed algebraic subsets is a Noetherian topological space
  \item $F_1,...,F_k\in K[X_1,..., X_{n+1}]$, $V=\{x\in\Proj^n\;:\;F_1(x)=0,..., F_k(x)=0\}$
    $$\psi^{-1}(V)=V(F_1|_{X_i=1},..., F_k|_{X_i=1})$$
    $F_j|_{X_i=1}$ is called the \buff{dehomogenization} of $H$ with respect to $X_i$. Denote it by $\widetilde{F}_j$
  \item $W=V(H_1,..., H_l)\subseteq\A^n$, then
    $$U_i\cap \{x\in\Proj^n\;:\;\widetilde{H}_1(x)=0,...,\widetilde{H}_l(x)=0\}=\psi_i(W)$$
    denote the $\{x\;:\;\widetilde{H}_i(x)=0\}$ by $W^*$.
  \item For closed $V\subseteq\A^n$ we have 
    \begin{itemize}
      \item $\dim V=\dim V^*$, 
      \item $V$ irreducible $\iff$ $V^*$ irreducible
    \end{itemize}
  \item $W\subseteq \Proj^n$ irreducible, $W\cap U_i\neq\emptyset$, then $W$ is the Zariski closure of $W\cap U_i$ 
\end{enumerate}

\begin{definition}{}{}
  \buff{Projective variety} is an irreducible projective algebraic set.
\end{definition}

Any projective plane curve can be expressed as $V=\{x\in\Proj^2\;:\;F(x)=0\}$ for some $F\in K[X,Y,Z]$.
{}
A point $x\in V$ is \buff{smooth} if there is $i\leq n$ such that $x\in U_i$ and $\psi_i^{-1}(x)$ is a smooth point of the affine variety $\psi_i^{-1}(V)$. A point is \buff{singular} if it is not smooth.

\section{}{Bezout theorem, czyli intersection nr, divisors}

\begin{definition}{intersection number}{}
  $x\in\Proj^2$, $F,H,G\in K[X,Y, Z]$ are homogenous
  \begin{itemize}
    \item $i$ such that $x\in U_i$ 
      $$I(x, F\cap H)=I(\psi_i^{-1}(x), (F|_{X_i=1})\cap (H|_{X_i=1}))$$
      is the \buff{intersection number}
    \item $F$, $H$ irreducible, $V,W\subseteq\Proj^2$ projective plane curves
      $$V=\{x\in\Proj^2\;:\;F(x)=0\}$$
      $$W=\{x\in\Proj^2\;:\;H(x)=0\}$$
      $$I(x, V\cap G)=I(x, F\cap G)$$
      $$I(x, W\cap V)=I(x, F\cap G)$$
  \end{itemize}
\end{definition}

\begin{theorem}{}{}
  $F,H\in K[X,Y,Z]$ homogenous such that 
  $$V=\{x\in \Proj^2\;:\;F(x)=0, H(x)=0\}$$
  is finite. Then
  $$\sum_{x\in V}I(x, F\cap H)=\deg(F)\deg(H)$$
\end{theorem}

The group of divisors on $V$ $\Div(V)$ is the free Abelian group with basis $V$, $\Z[V]$.
\begin{definition}{intersection divisor}{}
  Let $F\in K[X,Y,Z]$ be homogenous with finite $\{x\;:\;F(x)=0\}$. The \buff{intersection divisor} of $F$ is
  $$V\cdot F=\sum_{x\in V}I(x, V\cap F)\cdot x\in Div(V).$$
\end{definition}

For $D=n_1x_1+...+n_kx_k\in \Div(V)$ we define $\deg(D)=n_1+...+n_k$.
$$\deg(V\cdot F)=\deg(V)\deg(F)$$

\section{}{Elliptic curves}

\begin{definition}{}{}
  An \buff{elliptic curve} is a pair $(C, O)$ such that $C$ is a projective plane curve of degree $3$ and $O\in C$
\end{definition}

We aim to show that there is a natural commutative group structure on $C$ such that $O$ becomes the neutral element.

\begin{lemma}{}{}
  For any $x,y\in C$ there is a unique line $L$ in $\Proj^2$ and unique $z\in C$ such that 
  $$C\cdot L=x+y+z\in Div(C)$$
\end{lemma}

$$\phi:C\times C\to C$$
$\phi(x, y)=z$ $\iff$ there is a line $L$ such that
$$C\cdot L=x+y+z.$$
For $x,y,z\in C$ we have $\phi(x,y)=\phi(y,x)$
$$\phi(x,y)=z\iff \phi(y,z)=x\iff \phi(z, x)=y.$$
so this is almost a group action but lacks the neutral element.

\begin{theorem}{}{}
  $(C, \oplus, O)$ is a \buff{commutative group}, where the multiplication is defined
  $$x\oplus y=\phi(O, \phi(x,y)).$$
\end{theorem}

\begin{definition}{}{}
  $D,D'\in\Div(C)$
  $$D=\sum_{P\in C}n_PP$$
  $$D'=\sum_{P\in C}n_P'P$$
  we write $D\leq D'$ if $\forall\;P\in C$ $n_P\leq n_P'$.
\end{definition}

\begin{theorem}{}{}
  $F,G\in K[X,Y,Z]$ homogenous such that each has finitely many zeros on $C$ and
  $$\forall x\in C\;I(x, C\cap F)\geq I(x, C\cap G)$$
  Then there exists homogenous $H\in K[X,Y,Z]$ 
  $$C\cdot F=C\cdot G+C\cdot H$$
\end{theorem}

Using assumptions from the theorem:
\begin{itemize}
  \item $$C\cdot(FG)=C\cdot F+C\cdot G$$
  \item If $C\cdot F=x_1+...+x_s+y$ and $C\cdot G=x_1+...+x_s+z$, then $y=z$.
\end{itemize}

\begin{theorem}{}{}
  $$(C, \oplus, O_1)\cong (C, \oplus, O_2)$$
\end{theorem}

\begin{definition}{inflection point}{}
  $x\in C$ is \buff{inflection point} if
  $$I(x, C\cap T_xC)>2$$
\end{definition}

For an elliptic curve $C$ the following are equivalent
\begin{itemize}
  \item $x$ - inflection point
  \item $I(X, C\cap T_xC)=3$
  \item $\phi(x,x)=x$
\end{itemize}

$\text{Char}(K)\neq 3\implies $ there are $9$ inflection points on $C$


\end{document}

