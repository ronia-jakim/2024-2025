\section{25.02.2025}{cos}

\begin{definition}{obiekt początkowy i końcowy}{}
  Powiemy, że obiekt $C\in \Cc_0$ jest \buff{początkowy}, jeśli dla każdego $D\in\Cc_0$ istnieje dokładnie jeden morfizm $C\to D$, $|\Cc(C, D)|=1$. Analogicznie definiujemy \buff{obiekt końcowy} $C$: $\forall\;D\in\Cc_0\;|\Cc(D, C)|=1$.
\end{definition}

\begin{example}[m]
  \item W kategorii, której obiektami jest odcinek $\Cc_0=[0,1]$, a morfizmy to relacja $\leq$ obiektem początkowym jest $0$, a końcowym - $1$.
  \item W kategorii zbiorów obiektem początkowym jest $\emptyset$, a obiektem końcowym jest singleton.
  \item W $Gr$ grupa trywialna jest zarówno obiektem początkowym jak i końcowym.
  \item Kategoria, która ma dwa obiekty bez morfizmów między nimi nie ma obiektu końcowego ani początkowego.
\end{example}

Niech $F:\mathcal{I}\to \Cc$ będzie funktorem, gdzie o kategorii $\mathcal{I}$ myślimy jako o kategorii indeksów. Przez $\Cc^{\mathcal{I}}$ oznaczmy kategorię wszystkich takich funktorów. 
Istnieje stały funktor, tzn. taki, że $C(i)=C$ dla każdego $i\in\mathcal{I}_0$ oraz $C(f)=id_C$ dla każdego morfizmu.

Budujemy kategorię, której 
\begin{itemize}
  \item obiekty to wszystkie naturalne przekształcenia funktora $F$ w funktory stałe $C$, $\phi:F\implies C$, czyli komutujące diagramy (kostożki) 
    \begin{center}
      \begin{tikzcd}
        F(i)\arrow[rr, "F(f)"]\arrow[dr, "\phi_i" below left] & & F(j)\arrow[dl, "\phi_j"]\\ 
                                                  & C
      \end{tikzcd}
    \end{center}
  \item a morfizmy to strzałki $C\to D$ takie, że diagram
    \begin{center}
      \begin{tikzcd}
        C\arrow[rr] & & D\\ 
                    & F\arrow[ur, Rightarrow, blue, "\phi" below right]\arrow[ul, Rightarrow, orange, "\psi" below left]
      \end{tikzcd}
    \end{center}
    komutuje.
\end{itemize}

Diagram wyżej można rozpisać jako:
\begin{center}
  \begin{tikzcd}[column sep=large]
    & F(i)\arrow[d] \arrow[ddl, "\phi_i" above left, blue]\arrow[ddr, "\psi_i" above right, orange] \\ 
    & F(j)\arrow[dl, "\phi_j" above, blue]\arrow[dr, "\psi_j" above, orange]\\ 
    D & & C\arrow[ll]
  \end{tikzcd}
\end{center}

\begin{definition}{kogranica funktora}{}
  \buff{Kogranicą} (\acc{granica prosta}) funktora $F$, $\varinjlim F$, nazywamy obiekt początkowy w wyżej zdefiniowanej kategorii naturalnych przekształceń. 
  % \buff{Granica} (\acc{granica odwrotna}) to wtedy obiekt końcowy powyższej kategorii ze wszystkimi strzałkami zdualizowanymi $\varprojlim F$.
\end{definition}

Diagram wyżej możemy zdualizować i zamiast rozpatrywać naturalne przekształcenia $\phi:F\implies C$ możemy rozważyć naturalne przekształcenia $\phi:C\implies F$, czyli diagramy (stożki)
\begin{center}
  \begin{tikzcd}
    & C \arrow[dl, "\phi_i" above left] \arrow[dr, "\phi_j" above right]\\ 
    F(i)\arrow[rr, "F(f)" below] & & F(j)
  \end{tikzcd}
\end{center}
z morfizmami definiowanymi analogicznie. 

\begin{definition}{granica funktora}{}
  \buff{Granica} (\acc{granica odwrotna}) to obiekt końcowy powyższej kategorii stożków, $\varprojlim F$.
\end{definition}

% {\color{red}tutaj jest zdjecie
%
% przyklad dla kategorii zbiorów
%
% ja chyba chce wziąć dwuelementową kategorię $\mathcal{I}$ i tutaj policzyć, jeśli $F(1)=G$, a $F(2)=H$.
% }
%
Rozważmy kategorię $\mathcal{I}$, która ma dwa obiekty $\mathcal{I}_0=\{0,1\}$. Niech $F:\mathcal{I}\to Set$ będzie funktorem, dla którego $F(0)=A$, a $F(1)=B$. Niech $\phi$ oraz $\psi$ będzie parą naturalnych przekształceń, dla których
\begin{center}
  \begin{tikzcd}[column sep=large, row sep=large]
     & \varinjlim F\arrow[dl, "\phi_0" above left] \arrow[dr, "\phi_1"] \\ 
    F(0)=A & D \arrow[l, "\psi_0"] \arrow[r, "\psi_1" below right] \arrow[u, "\exists!", dashed] & F(1)=B
  \end{tikzcd}
\end{center}
gdzie pionowa strzałka istnieje i jest jedyna, bo $\varinjlim F$ to obiekt końcowy. Podobnie zachowuje się $A\oplus B$:
\begin{center}
  \begin{tikzcd}[column sep=large, row sep=large]
     & A\oplus B\arrow[dl, "\pi_A" above left] \arrow[dr, "\pi_B"] \\ 
    F(0)=A & & F(1)=B \\ 
           & D\arrow[ul, "\psi_0"] \arrow[ur, "\psi_1" below right] \arrow[uu, "{\scriptstyle d\mapsto \psi_0(d)\oplus \psi_1(d)}", sloped, dashed] & F(1)=B
  \end{tikzcd}
\end{center}

\begin{example}[m]
\item 


  \item Rozważmy kategorię grup. 
    \begin{center}
      \begin{tikzcd}
          & F\arrow[dd, dashed]\arrow[dl]\arrow[dr]  \\ 
        G & & H\\ 
          & G\times H
      \end{tikzcd}
    \end{center}
    \begin{center}
      \begin{tikzcd}
          & F \\ 
        G\arrow[ur] & & H\arrow[ul]\\ 
                    & G\ast H \arrow[uu] 
      \end{tikzcd}
    \end{center}
  \item Niech $F:\mathcal{I}\to (P, \leq)$ z dwuobiektowej kategorii $\mathcal{I}$ w zbiór uporządkowany.
\end{example}

kategoria nieskończenie wiele elementów, ale bez strzałek (jako $\mathcal{I}$)
 

Niech $C$ oraz $C'$ będą granicami tego samego funktora. Z definicji mamy
\begin{center}
  \begin{tikzcd}[column sep=large, row sep=large]
    & F(i)\arrow[dr, "\phi_i"]\arrow[d, "\psi_i"]\arrow[dl, "\phi_i" above left] \\ 
    C & C'\arrow[l, "\exists g" above] & C\arrow[ll, bend left=20, "id"] \arrow[l, "\exists f" above]
  \end{tikzcd}
\end{center}

tutaj liczby p-adyczne

ekwalizator, koekwalizator

\begin{definition}{surjekcja, epimorfizm}{}
  Jeśli kategoria ma obiekt początkowy równy obiektowi końcowemu...
\end{definition}

\buff{Monoid} $(M, \star, 1)$ to struktura algebraiczna z binarną operacją oraz elementem neutralnym. Dodatkowo, komutować ma diagram 
\begin{center}
  \begin{tikzcd}
    M^3\arrow[r, "\star\times id"]\arrow[d, "id\times\star" left] & M^2\arrow[d, "\star"]\\ 
    M^2\arrow[r, "\star"] & M
  \end{tikzcd}
\end{center}
co znaczy, że działanie jest łączne.

\begin{definition}{obiekt monoidalny, kategoria monoidalna}{}
  Niech $\Cc$ będzie kategorią z produktem i elementem początkowym. Niech $M\in \Cc$ będzie obiektem, dla którego mamy $\mu:M^2\to M$ oraz $\epsilon: \{1\}\to M$ takie, że komutują diagramy
  \begin{center}
    \begin{tikzcd}[row sep=large, column sep=large]
      M^3\arrow[r, "\mu\times id"]\arrow[d, "id\times \mu" left] & M^2\arrow[d, "\mu"]\\ 
      M^2\arrow[r, "\mu" below] & M
    \end{tikzcd}
  \end{center}
  \begin{center}
    \begin{tikzcd}[row sep=large, column sep=large]
      M\arrow[r, "\epsilon\times id"]\arrow[d, "id\times \epsilon" left]\arrow[dr, "=" above right] & M^2\arrow[d, "\mu"]\\ 
      M^2\arrow[r, "\mu" below] & M
    \end{tikzcd}
  \end{center}
  Wtedy $M$ jest \buff{obiektem monoidalnym}.
  
  Obiekt monoidalny w kategorii $Cat$ nazywa się \buff{kategorią monoidalną}.
\end{definition}

\begin{example}[m]
\item Dowolna kategoria $\Cc$ z koproduktem i elementem końcowym jest kategorią monoidalna.
\item Kategoria endofunktorów ma strukturę monoidalną. To znaczy, jeśli mamy dwa endofunktory $F, G\in End(\Cc)$, to potrafimy je złożyć w dobry sposób.

  Funktor $T\in Func(\Cc)$ oraz dwa naturalne przekształcenia $\mu:T^2\to T$, $\epsilon: Id\to T$, nazywa się \buff{monadą}.
\end{example}

Czy $S^n\vee S^n$ to produkt czy produkt w kategorii $Toph_\star$. tutaj jakies zdjecie






