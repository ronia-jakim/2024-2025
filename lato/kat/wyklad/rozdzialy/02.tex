\section{25.02.2025}{Produkty i koprodukty kategorii}

\subsection{O obiektach początkowych i końcowych słów kilka}

\begin{definition}{obiekt początkowy i końcowy}{}
  Powiemy, że obiekt $C\in \Cc_0$ jest \buff{początkowy}, jeśli dla każdego $D\in\Cc_0$ istnieje dokładnie jeden morfizm $C\to D$, $|\Cc(C, D)|=1$. Analogicznie definiujemy \buff{obiekt końcowy} $C$: $\forall\;D\in\Cc_0\;|\Cc(D, C)|=1$.
\end{definition}

\begin{example}[m]
  \item W kategorii, której obiektami jest odcinek $\Cc_0=[0,1]$, a morfizmy to relacja $\leq$ obiektem początkowym jest $0$, a końcowym - $1$.
  \item W kategorii zbiorów obiektem początkowym jest $\emptyset$, a obiektem końcowym jest singleton.
  \item W $Gr$ grupa trywialna jest zarówno obiektem początkowym jak i końcowym.
  \item Kategoria, która ma dwa obiekty bez morfizmów między nimi nie ma obiektu końcowego ani początkowego.
\end{example}

\begin{fact}{}{}
  Obiekty końcowe i początkowe, jeśli istnieją, to są jedyne z dokładnością do izomorfizmu.
\end{fact}

\begin{proof}
  Niech $C$ i $C'$ będą obiektami końcowymi kategorii $\Cc$. Wiemy, że $\Cc(C, C)=\{id_C\}$, czyli komutujący diagram
  \begin{center}
    \begin{tikzcd}
      C \arrow[rr, "id_C"]\arrow[dr, "\exists!f" below left] & & C\\ 
                           & C'\arrow[ur, "\exists!g" below right]
    \end{tikzcd}
  \end{center}
  daje $g\circ f=id_C$. Analogiczny diagram daje $f\circ g=id_{C'}$. Stąd $f$ i $g$ to para wzajemnie odwrotnych izomorfizmów między $C$ i $C'$
\end{proof}

\subsection{(Ko)granice funktorów a (ko)produtky}

Niech $F:\mathcal{I}\to \Cc$ będzie funktorem, gdzie o kategorii $\mathcal{I}$ myślimy jako o kategorii indeksów. Przez $\Cc^{\mathcal{I}}$ oznaczmy kategorię wszystkich takich funktorów. 
Powiemy, że funktor $C$ jest stały, jeżeli $C(i)=C$ dla każdego $i\in\mathcal{I}_0$ oraz $C(f)=id_C$ dla każdego morfizmu.

Budujemy kategorię, której 
\begin{itemize}
  \item obiekty to wszystkie naturalne przekształcenia funktora $F$ w funktory stałe $C$, $\phi:F\implies C$, czyli komutujące diagramy (kostożki) 
    \begin{center}
      \begin{tikzcd}
        F(i)\arrow[rr, "F(f)"]\arrow[dr, "\phi_i" below left] & & F(j)\arrow[dl, "\phi_j"]\\ 
                                                  & C
      \end{tikzcd}
    \end{center}
  \item a morfizmy to strzałki $C\to D$ takie, że diagram
    \begin{center}
      \begin{tikzcd}
        C\arrow[rr] & & D\\ 
                    & F\arrow[ur, Rightarrow, blue, "\phi" below right]\arrow[ul, Rightarrow, orange, "\psi" below left]
      \end{tikzcd}
    \end{center}
    komutuje.
\end{itemize}

Diagram wyżej można rozpisać jako:
\begin{center}
  \begin{tikzcd}[column sep=large]
    & F(i)\arrow[d] \arrow[ddl, "\phi_i" above left, blue]\arrow[ddr, "\psi_i" above right, orange] \\ 
    & F(j)\arrow[dl, "\phi_j" above, blue]\arrow[dr, "\psi_j" above, orange]\\ 
    D & & C\arrow[ll]
  \end{tikzcd}
\end{center}

\begin{definition}{kogranica funktora}{}
  \buff{Kogranicą} (\acc{granica prosta}) funktora $F$, $\varinjlim F$, nazywamy obiekt początkowy w wyżej zdefiniowanej kategorii naturalnych przekształceń. 
  % \buff{Granica} (\acc{granica odwrotna}) to wtedy obiekt końcowy powyższej kategorii ze wszystkimi strzałkami zdualizowanymi $\varprojlim F$.
\end{definition}

Diagram wyżej możemy zdualizować i zamiast rozpatrywać naturalne przekształcenia $\phi:F\implies C$ możemy rozważyć naturalne przekształcenia $\phi:C\implies F$, czyli diagramy (stożki)
\begin{center}
  \begin{tikzcd}
    & C \arrow[dl, "\phi_i" above left] \arrow[dr, "\phi_j" above right]\\ 
    F(i)\arrow[rr, "F(f)" below] & & F(j)
  \end{tikzcd}
\end{center}
z morfizmami {definiowanymi analogicznie. 

\begin{definition}{granica funktora}{}
  \buff{Granica} (\acc{granica odwrotna}) to obiekt końcowy powyższej kategorii stożków, $\varprojlim F$.
\end{definition}

% {\color{red}tutaj jest zdjecie
%
% przyklad dla kategorii zbiorów
%
% ja chyba chce wziąć dwuelementową kategorię $\mathcal{I}$ i tutaj policzyć, jeśli $F(1)=G$, a $F(2)=H$.
% }
%
Rozważmy kategorię $\mathcal{I}$, która ma dwa obiekty $\mathcal{I}_0=\{0,1\}$. Niech $F:\mathcal{I}\to Set$ będzie funktorem, dla którego $F(0)=A$, a $F(1)=B$. Niech $\phi$ oraz $\psi$ będzie parą naturalnych przekształceń, dla których
\begin{center}
  \begin{tikzcd}[column sep=large, row sep=large]
     & \varinjlim F\arrow[dl, "\phi_0" above left] \arrow[dr, "\phi_1"] \\ 
    F(0)=A & D \arrow[l, "\psi_0"] \arrow[r, "\psi_1" below right] \arrow[u, "\exists!f", dashed] & F(1)=B
  \end{tikzcd}
\end{center}
gdzie pionowa strzałka istnieje i jest jedyna, bo $\varinjlim F$ to obiekt końcowy. Jeśli weźmiemy $\varinjlim F=A\times B$, a $\phi_0=\pi_A$ oraz $\phi_1=\pi_B$ będą rzutami i $f(d)=(\psi_0(d), \phi_1(d))$, to diagram nadal jest prawdziwy. 

Granica odwrotna tego samego funktora, to z kolei suma rozłączna $A\sqcup B$, bo diagram
\begin{center}
  \begin{tikzcd}[column sep=large, row sep=large]
    F(0)=A\arrow[r, "\psi_0"]\arrow[dr, "\phi_0=i_A" below left] & D & F(1)=B\arrow[l, "\psi_1" above]\arrow[dl, "\phi_1=i_B"]\\ 
                                                       & \varprojlim F= A\sqcup B \arrow[u, dashed, "\exists!f"]
  \end{tikzcd}
\end{center}
gdzie $f(x)=\phi_0(x),$ jeśli $x\in A$ oraz $f(x)=\psi_1(x)$ jeśli $x\in B$, komutuje.

\begin{definition}{(ko)produkt}{}
  \buff{Produktem} obiektów $A$ i $B$ kategorii $\Cc$ nazywamy granicę prostą (kogranicę) funktora $F:\mathcal{I}\to \Cc$ dla $\mathcal{I}$ oraz $F$ jak wyżej.

  \buff{Koproduktem} obiektów $A$ i $B$ kategorii $\Cc$ nazywamy granicę odwrotną (granicę) funktora $F:\mathcal{I}\to\Cc$
\end{definition}

\begin{example}[m]
  \item W kategorii grup produkt to iloczyn kartezjański dwóch grup, tak jak w kategorii zbiorów, tj. dla grup $A,G,H$ komutuje diagram
    \begin{center}
      \begin{tikzcd}[column sep=large, row sep=large]
        & G\times H\arrow[dl, "\pi_G" above left]\arrow[dr, "\pi_H"]\\ 
        G & A\arrow[l, "g"]\arrow[r, "h"]\arrow[u, "g\times h" below] & H
      \end{tikzcd}
    \end{center}
    Koprodukt to z kolei produkt wolny tych dwóch grup:
\begin{center}
  \begin{tikzcd}[column sep=large, row sep=large]
    G\arrow[r, "g"]\arrow[dr, "i_G" below left] & A & H\arrow[l, "h" above]\arrow[dl, "i_H"]\\ 
                                                       & H\ast G \arrow[u, dashed, "\exists!f"]
  \end{tikzcd}
\end{center}
gdzie $f$ nakłada na litery słów $G\ast H$ pochodzące z $G$ morfizm $g$, a na litery pochodzące z $H$ - morfizm $h$.
  \item Niech $F:\mathcal{I}\to (P, \leq)$ z dwuobiektowej kategorii $\mathcal{I}$ w zbiór uporządkowany. Wtedy jeśli mamy diagram 
    \begin{center}
      \begin{tikzcd}
         & \varinjlim F\arrow[dr]\arrow[dl] \\ 
        F(0)=a & d \arrow[l]\arrow[r]\arrow[dashed, u] & F(1)=b
      \end{tikzcd}
    \end{center}
    to znaczy, że $d\leq a$, $d\leq b$ oraz $d\leq \varinjlim{F}$. Żeby więc miało to sens dla dowolnego $d\leq a,b$ to $\varinjlim F=\inf\{a,b\}$. Analogicznie dostajemy, że $\varprojlim F=\sup\{a,b\}$.

  \item Jeśli $\mathcal{I}$ jest kategorią o nieskończenie wielu obiektach bez morfizmów między różnymi obiektami, a $F:\mathcal{I}\to Set$ jest funktorem w kategorię zbiorów, to wówczas kogranicą tego funktora jest nieskończony iloczyn kartezjański $\prod_{i\in\mathcal{I}_0}F(i)$, a granicą - nieskończona suma rozłączna $\bigsqcup_{i\in\mathcal{I}_0}F(i)$.
\end{example}

% kategoria nieskończenie wiele elementów, ale bez strzałek (jako $\mathcal{I}$)
 % Niech $C$ oraz $C'$ będą kogranicami tego samego funktora. Z definicji mamy
% \begin{center}
%   \begin{tikzcd}[column sep=large, row sep=large]
%     & F(i)\arrow[dr, "\phi_i"]\arrow[d, "\psi_i"]\arrow[dl, "\phi_i" above left] \\ 
%     C & C'\arrow[l, "\exists g" above] & C\arrow[ll, bend left=20, "id"] \arrow[l, "\exists f" above]
%   \end{tikzcd}
% \end{center}

\begin{fact}{}{}
  Granica i kogranica funktora, jeśli istnieje, to jest jedyna z dokładnością do izomorfizmu. Stąd również produkty i koprodukty są unikalne.
\end{fact}

\begin{proof}
  Wynika z uniwersalności obiektów końcowych i początkowych.
\end{proof}

% tutaj liczby p-adyczne
% ekwalizator, koekwalizator
%
% \begin{definition}{surjekcja, epimorfizm}{}
%   Jeśli kategoria ma obiekt początkowy równy obiektowi końcowemu...
% \end{definition}

\begin{example}
  Rozważmy funktor $F:\mathcal{I}^{op}\to Grp$, gdzie $\mathcal{I}=(\N, \leq)$ taki, że dla każdych $i,j\in\N$, $i\leq j$ mamy
  \begin{center}
    \begin{tikzcd}[column sep=large]
      F(j)=\Z/p^j\Z\arrow[r, "F(i\to j)=q"] & F(i)=\Z/p^i\Z
    \end{tikzcd}
  \end{center}
  gdzie $q$ to morfizm ilorazowy.

  Liczby $p$-adyczne to rozszerzenie liczb wymiernych różne od liczb rzeczywistych i zespolonych. Całkowite liczby $p$-adyczne to szeregi
  $$\sum_{i=k}^\infty a_ip^i,$$
  gdzie $k\in\N$ oraz $0\leq a_i < p$. Okazuje się, że całkowite liczby $p$-adyczne, $\Z_p$, można zdefiniować jako granicę funktora $F$:
  \begin{center}
    \begin{tikzcd}
      & & \Z_p \arrow[dll]\arrow[dl]\arrow[d]\arrow[drr]\arrow[drrr] \\ 
      ...\arrow[r] & \Z/p^n\Z\arrow[r] & \Z/p^{n-1}\Z\arrow[r] & ... \arrow[r]& \Z/p^2\arrow[r] & \Z/p\Z
    \end{tikzcd}
  \end{center}
  Granica prosta takiego funktora jest trywialna, ale możemy rozważyć inny funktor,z kategorii $\Z$ z porządkiem, tzn: $G:\Z\to Grp$ taki, że $G(n)=\Z/p^n\Z$, natomiast strzałkę $n+1\to n$ przekształcamy na odwzorowanie
  $$\Z/p^n\Z\ni x\mapsto p\cdot x\in \Z/p^{n+1}\Z.$$
  Wtedy granicą prostą $G$ jest $C_{p^\infty}$ - pierwiastki $p^n$-tego stopnia z $1$, dla dowolnego $n$. 
\end{example}

\subsection{Obiekty i kategorie monoidalne}

\buff{Monoid} $(M, \star, 1)$ to struktura algebraiczna z binarną operacją oraz elementem neutralnym. Dodatkowo, komutować ma diagram 
\begin{center}
  \begin{tikzcd}
    M^3\arrow[r, "\star\times id"]\arrow[d, "id\times\star" left] & M^2\arrow[d, "\star"]\\ 
    M^2\arrow[r, "\star"] & M
  \end{tikzcd}
\end{center}
co znaczy, że działanie jest łączne.

\begin{definition}{obiekt monoidalny, kategoria monoidalna}{obiekt monoidalny}
  Niech $\Cc$ będzie kategorią z produktem i elementem początkowym. Niech $M\in \Cc$ będzie obiektem, dla którego mamy $\mu:M^2\to M$ oraz $\epsilon: \{1\}\to M$ takie, że komutują diagramy
  \begin{center}
    \begin{tikzcd}[row sep=large, column sep=large]
      M^3\arrow[r, "\mu\times id"]\arrow[d, "id\times \mu" left] & M^2\arrow[d, "\mu"]\\ 
      M^2\arrow[r, "\mu" below] & M
    \end{tikzcd}
  \end{center}
  \begin{center}
    \begin{tikzcd}[row sep=large, column sep=large]
      M\arrow[r, "\epsilon\times id"]\arrow[d, "id\times \epsilon" left]\arrow[dr, "=" above right] & M^2\arrow[d, "\mu"]\\ 
      M^2\arrow[r, "\mu" below] & M
    \end{tikzcd}
  \end{center}
  Wtedy $M$ jest \buff{obiektem monoidalnym}.
  
  Obiekt monoidalny w kategorii $Cat$ nazywa się \buff{kategorią monoidalną}.
\end{definition}

\begin{example}[m]
\item Dowolna kategoria $\Cc$ z koproduktem i obiektem końcowym jest kategorią monoidalna.
\item Kategoria endofunktorów ma strukturę monoidalną. To znaczy, jeśli mamy dwa endofunktory $F, G\in End(\Cc)$, to potrafimy je złożyć w dobry sposób.

  Funktor $T\in End(\Cc)$ oraz dwa naturalne przekształcenia $\mu:T^2\to T$, $\epsilon: Id\to T$, nazywa się \buff{monadą}.
\end{example}

% Czy $S^n\vee S^n$ to produkt czy produkt w kategorii $Toph_\star$. tutaj jakies zdjecie






