\section{03.03.2025}{Funktory dołączone}

\subsection{Motywacja abstrakcyjnego nonsensu}

Niech $V$ będzie przestrzenią wektorową nad ciałem $k$, a $B$ wybraną jej bazą. Dowolne odwzorowanie $B\to V$ możemy rozszerzyć na odwzorowanie liniowe $k[B]=V\to V$. To znaczy, mamy izomorfizm zbiorów
$$Hom(B, V)\cong Hom(V, V).$$
W języku abstrakcyjnego nonsensu możemy zdefiniować dwa funktory, 
$$Set(-, U(-)):Set^{op}\times Vect_k^{fin}\to Set$$
$$Vect_k(k[-], -):Set^{op}\times Vect_k^{fin}\to Set,$$
gdzie $U:Vect_k^{fin}\to Set$ to funktor zapominający strukturę przestrzeni wektorowej, między którymi istnieją naturalne izomorfizmy.
% Odwzorowanie na bazie $B\to V$ daje liniowe odwzorowanie $k[B]\to V$. Można to abstrakcyjnie wyrazić jako relację między morfizmami w kategorii zbiorów między zbiorem $B$ a $U(V)$, gdzie $U:Vect\to Set$ to funktor zapominania, a morfizmami w kategorii przestrzeni wektorowych, $Vect_k(k[B], V)$. To znaczy, chcemy izomorfizm
$$Set(-, U(-))\cong Vect_k(k[-], -)$$

\begin{definition}{funktory dołączone}{}
  Niech $L:\Cc\to\Dd$ oraz $R:\Dd\to \Cc$ będą funktorami. Powiemy, że $L$ jest \buff{lewo dołączony} do funktora $R$, a $R$ \buff{prawo dołączony} do $L$, jeśli funktory
  $$\Cc(-, R-), \Dd(L-, -):\Cc^{op}\times \Dd\to Set$$
  są naturalnie izomorficzne. Taką parę funktorów dołączonych oznaczamy $\color{green}L\dashv R$.
  % Powiemy, że $L:\Cc\to \Dd$ i $R:\Dd\to \Cc$ są parą funktorów \buff{dołączonych}, $L\dashv R$, jeśli funktory
  % $$\Cc(-, R-), \Dd(L-, -):\Cc^{op}\times\Dd\to Set,$$
  % są naturalnie izomorficzne.
\end{definition}

\subsection{Dużo przykładów funktorów dołączonych}

\begin{enumerate}
  \item Niech $R:Set_*\to Set$ będzie funktorem z kategorii zbiorów zbazowanych w kategorię zbiorów, który zapomina o punkcie bazowym. Chcemy teraz znaleźć funktor $L:Set\to Set_*$, który będzie do niego lewo dołączony. Niech $L(X)=X\cup\{X\}$ (lub bardziej obrazowo: $X\sqcup\{*\}$), gdzie $y_0$ poślemy na $\{X\}$, to znaczy doklejamy do $X$ singleton i staje się on punktem wyróżnionym. 

    Oba funktory są różnowartościowe na obiektach, więc wystarczy przekonać się, że
    $$Set_*(LX, (Y, y_0))\cong Set(X, R(Y, y_0))$$
    jest izomorfizmem. Dowolna funkcja $X\to Y$ rozszerza się przez posłanie $\{X\}\mapsto y_0$ na funkcję $(X, \{X\})\to (Y, y_0)$.
    % Biorąc dowolną funkcję $f:X\to Y$, rozszerzamy ją na $Lf:(X, \{X\})\to (Y, y_0)$ tak, że $Lf\restriction X=f$ i $Lf(\{X\})=y_0$. Wtedy $RLf$ to $Lf$ obcięte do $X$, czyli $f$. W drugą stronę, dowolna funkcja $g:LX\to (Y, y_0)$, $g(\{X\})=y_0$, obcina się do funkcji $Rg:X\to Y$, którą następnie możemy ponownie rozszerzyć tak, że $LRg=g$.
  \item Podobna sytuacja ma miejsce, kiedy szukamy lewo dołączony funktor do $R:Ring\to Rng$ między kategorią pierścieni z jedynką, a wszystkimi pierścieniami. Definiujemy funktor 
    $$L:Rng\to Ring$$
    jako doklejenie $\Z$, $L(S)=\Z\oplus S$ z działaniem $(n, s)(n', s')=(nn', ns'+ss'+n's)$, wtedy $(1, 0_S)$ jest jedynka w nowym pierścieniu. Pozostaje przyjrzeć się co się dzieje z morfizmami, skoro
    $$Rng(S, RT)\cong Ring(LS, T).$$
    Dowolny morfizm $\phi:S\to RT$ wystarczy, że trzyma element neutralny ze względu na dodawanie i jest addytywny. Możemy go rozszerzyć na morfizm, który całą pierwszą współrzędną $LS=\Z\oplus S$ posyła w $1_T\in T$, a drugą zgodnie z $\phi$. W drugą stronę wystarczy obciąć morfizm do drugiej współrzędnej.
  \item Niech $\Delta:Set\to Set\times Set$ będzie funktorem takim, że  $\Delta(C)=(C, C)$. Zaczniemy od szukania funktora dołączonego do niego z prawej strony, czyli $R:Set\times Set\to Set$ takiego, że 
    $$\Hom(X, R(Y,Z))\cong\Hom(\Delta(X), (Y,Z)).$$
    Od razu narzuca się $R(Y,Z)=Y\times Z$, czyli zlepiamy współrzędne $\Delta(X)$ w jedną. Przypomnijmy, że iloczyn kartezjański w kategorii zbiorów jest produktem.

    Funktor lewo dołączony musi zatem spełniać
    $$\Hom(L(X,Y), Z)\cong \Hom((X,Y),\Delta(Z)),$$
    czyli dowolną funkcję $(X, Y)\to (Z, Z)$ musimy umieć zapisać jako funkcję z pojedynczego zbioru, którym będzie suma rozłączna $L(X,Y)=X\sqcup Y$, czyli koprodukt w kategorii zbiorów. 

    Historia funktora $\Delta$ uogalnia się na dowolną kategorię, w której są produkty i koprodukty:
    $$\text{koprodukt}\dashv \Delta\dashv \text{produkt}$$
  \item Ustalmy zbiór $Y\in Set_0$ i niech $R:Set\to Set$ będzie funktorem, który zbiorowi $X$ przypisuje wszystkie funkcje z $Y$ w ten zbiór, $R(X)=Set(Y,X)$. Chcemy znaleźć funktor lewo dołączony $L:Set\to Set$ do $R$. Patrzymy na morfizmy i mamy
    $$Set(L(X), Z)\cong Set(X, \underbrace{Set(Y,Z)}_{R(Z)})$$
    zbiór po prawej to funkcje z $X$ w funkcje z $Y$ w $Z$. Można to przedstawić jako funkcje $X\times Y\to Z$, czyli $LX=X\times Y$.

    Technika tłumaczenia funkcji o więcej niż jednym argumencie na sekwencję funkcji nazywamy \acc{currying}.
  \item Analogicznie jak w poprzednim przykładzie, niech $R$ będzie pierścieniem (przemiennym z jedynką), $W$ $R$-modułem i $R$ funktorem $R:RMod\to RMod$ takim, że $R(U)=\Hom_R(W, U)$ będzie zbiorem homomorfizmów $R$-modułów. Funktorem lewo-dołączonym do $R$ będzie wtedy $L(V)=V\otimes W$:
    $$RMod(V, \Hom_R(W, U))\cong RMod(V\otimes W, U).$$
    \textbf{Uwaga:} tensor produkt zwykle nie ma funktora lewo do siebie dołączonego.
  \item Założmy, że kategoria $\Cc$ ma produkty i ustalmy $X\in \Cc$. Rozważmy funktor $L:\Cc\to \Cc$, $L(Y)=Y\times X$. Jeśli kategoria $\Cc$ posiada obiekty eksponencjalne, czyli wiemy jak uogólnić na nią przestrzeń funkcji $X\to Y$ (oznaczane $Y^X$), to funktorem prawo dołączonym do $L$ jest właśnie funktor przypisujący obiektowi $Y$ jego eksponens $Y^X$,
    $$\Cc(Y, Z^X)\cong \Cc(Y\times X, Z).$$
    Przykładem takiej kategorii są przestrzenie "core-compact".
\end{enumerate}

W ramach kontrprzykładu rozważmy funktor zapominania $U:FinGrp\to FinSet$, i załóżmy, że $L:FinSet\to FinGrp$ jest jego funktorem lewo dołączonym. Niech $p$ będzie taką liczbą pierwszą, że $p>|L(1)|$ (wystarczy, że są względnie pierwsze). Wtedy
$$FinSet(1, U(\Z_p))\cong FinGrp(L(1), \Z_p)$$
gdzie po lewej zbiór ma $|\Z_p|=p$ różnych funkcji z singletona w zbiór elementów grupy $\Z_p$, a po prawej mamy jedynie trywialny morfizm, bo żaden element $L(1)$ nie ma rzędu podzielnego przez $p$, czyli nie może przejść w żaden nietrywialny element $\Z_p$.


% Mamy funktor zapominania $U:FinGrp\to FinSet$, który nie ma lewego funktora dołączonego, bo $FinSet(1, U(G))$, to jeśli mamy $FinGrp(L(1), G)$, bierzemy $p>|L(1)|$ liczbę pierwszą i jako $G=\Z_p$. To wtedy mamy tylko trywialny mofizm $L(1)\to G$, a w zbiorach jest ich dużo.

% Załóżmy, że $\Cc$ ma produkty. funktor prawo dołączony do $-\times X$ jest funktorem eksponencjalnym $-^Y$, o ile istnieje. Core-compact spaces ma obiekty eksponencjalnie, jest tu podzbiór "lokalnie zwarte przestrzenie Hausdorffa" ($X^Y$ z topologią zwarto-otwarta: baza otoczeń indukowana przez $K\subseteq Y$ zwarty, $U\susbeteq X$ otwarty $V_{K,U}=\{f:f(K)\subseteq U\}$)

\subsection{Druga definicja}

\begin{definition}{funktory dołączone (naturalne transformacje)}{}
  Rozważmy parę funktorów
  \begin{center}
    \begin{tikzcd}
      \Cc\arrow[r, "L", bend left=20] & \Dd\arrow[l, "R", bend left=20].
    \end{tikzcd}
  \end{center}
  Powiemy, że $L$ jest lewo dołączony do $R$ i na odwrót, jeśli istnieją dwie natrualne transformacje
  $$\epsilon:LR\implies 1_{\Dd} \quad \eta:1_\Cc\implies RL$$
  takie, że komutują diagramy
  \begin{center}
    \begin{tikzcd}
      L\arrow[r, Rightarrow, "1_L\neta"]\arrow[dr, "1_L" below left, Rightarrow] & LRL\arrow[d, Rightarrow, "\epsilon 1_L"] & & R\arrow[r, "1_R" above, Rightarrow] \arrow[dr, "1_R" below left, Rightarrow] & RLR\arrow[d, "1_R\epsilon" right, Rightarrow]\\ 
                                                                                 & L & & & R
    \end{tikzcd}
  \end{center}
  $\eta$ nazywamy \acc{unit}, a $\epsilon$ to \acc{counit}.
\end{definition}

% $k[-]=L:Set\to Vect$ i $U=R:Vect\to Set$ to funktor zapominania; wtedy $RL$ to zbiór kombinacji formalnych kombinacji liniowych, czyli dla każdego $X\to RL(X)$ jest włożenie. W drugą stronę $k[V]\to V$ też działa
%
% Tutaj jeszcze raz powtórzyć przykłady z wcześniej
%
%
% \begin{theorem}{}{}
%   Obie definicje są równoważne
% \end{theorem}
%
% \begin{proof}
%   dowód strona 124 w emily
% \end{proof}



