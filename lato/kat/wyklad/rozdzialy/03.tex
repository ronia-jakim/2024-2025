\section{03.03.2025}{Funktory dołączone}

\subsection{Motywacja abstrakcyjnego nonsensu}

Niech $V$ będzie przestrzenią wektorową nad ciałem $k$, a $B$ wybraną jej bazą. Dowolne odwzorowanie $B\to V$ możemy rozszerzyć na odwzorowanie liniowe $k[B]=V\to V$. To znaczy, mamy izomorfizm zbiorów
$$Hom(B, V)\cong Hom(V, V).$$
W języku abstrakcyjnego nonsensu możemy zdefiniować dwa funktory, 
$$Set(-, U(-)):Set^{op}\times Vect_k^{fin}\to Set$$
$$Vect_k(k[-], -):Set^{op}\times Vect_k^{fin}\to Set,$$
gdzie $U:Vect_k^{fin}\to Set$ to funktor zapominający strukturę przestrzeni wektorowej, między którymi istnieją naturalne izomorfizmy.
% Odwzorowanie na bazie $B\to V$ daje liniowe odwzorowanie $k[B]\to V$. Można to abstrakcyjnie wyrazić jako relację między morfizmami w kategorii zbiorów między zbiorem $B$ a $U(V)$, gdzie $U:Vect\to Set$ to funktor zapominania, a morfizmami w kategorii przestrzeni wektorowych, $Vect_k(k[B], V)$. To znaczy, chcemy izomorfizm
$$Set(-, U(-))\cong Vect_k(k[-], -)$$

\begin{definition}{funktory dołączone}{}
  Niech $L:\Cc\to\Dd$ oraz $R:\Dd\to \Cc$ będą funktorami. Powiemy, że $L$ jest \buff{lewo dołączony} do funktora $R$, a $R$ \buff{prawo dołączony} do $L$, jeśli funktory
  $$\Cc(-, R-), \Dd(L-, -):\Cc^{op}\times \Dd\to Set$$
  są naturalnie izomorficzne. Taką parę funktorów dołączonych oznaczamy $\color{green}L\dashv R$.
  % Powiemy, że $L:\Cc\to \Dd$ i $R:\Dd\to \Cc$ są parą funktorów \buff{dołączonych}, $L\dashv R$, jeśli funktory
  % $$\Cc(-, R-), \Dd(L-, -):\Cc^{op}\times\Dd\to Set,$$
  % są naturalnie izomorficzne.
\end{definition}

\subsection{Dużo przykładów funktorów dołączonych}

\begin{enumerate}
  \item Niech $R:Set_*\to Set$ będzie funktorem z kategorii zbiorów zbazowanych w kategorię zbiorów, który zapomina o punkcie bazowym. Chcemy teraz znaleźć funktor $L:Set\to Set_*$, który będzie do niego lewo dołączony. Niech $L(X)=X\cup\{X\}$ (lub bardziej obrazowo: $X\sqcup\{*\}$), gdzie $y_0$ poślemy na $\{X\}$, to znaczy doklejamy do $X$ singleton i staje się on punktem wyróżnionym. 

    Oba funktory są różnowartościowe na obiektach, więc wystarczy przekonać się, że
    $$Set_*(LX, (Y, y_0))\cong Set(X, R(Y, y_0))$$
    jest izomorfizmem. Dowolna funkcja $X\to Y$ rozszerza się przez posłanie $\{X\}\mapsto y_0$ na funkcję $(X, \{X\})\to (Y, y_0)$.
    % Biorąc dowolną funkcję $f:X\to Y$, rozszerzamy ją na $Lf:(X, \{X\})\to (Y, y_0)$ tak, że $Lf\restriction X=f$ i $Lf(\{X\})=y_0$. Wtedy $RLf$ to $Lf$ obcięte do $X$, czyli $f$. W drugą stronę, dowolna funkcja $g:LX\to (Y, y_0)$, $g(\{X\})=y_0$, obcina się do funkcji $Rg:X\to Y$, którą następnie możemy ponownie rozszerzyć tak, że $LRg=g$.
  \item Podobna sytuacja ma miejsce, kiedy szukamy lewo dołączony funktor do $R:Ring\to Rng$ między kategorią pierścieni z jedynką, a wszystkimi pierścieniami. Definiujemy funktor 
    $$L:Rng\to Ring$$
    jako doklejenie $\Z$, $L(S)=\Z\oplus S$ z działaniem $(n, s)(n', s')=(nn', ns'+ss'+n's)$, wtedy $(1, 0_S)$ jest jedynka w nowym pierścieniu. Pozostaje przyjrzeć się co się dzieje z morfizmami, skoro
    $$Rng(S, RT)\cong Ring(LS, T).$$
    Dowolny morfizm $\phi:S\to RT$ wystarczy, że trzyma element neutralny ze względu na dodawanie i jest addytywny. Możemy go rozszerzyć na morfizm, który całą pierwszą współrzędną $LS=\Z\oplus S$ posyła w $1_T\in T$, a drugą zgodnie z $\phi$. W drugą stronę wystarczy obciąć morfizm do drugiej współrzędnej.
  \item Niech $\Delta:Set\to Set\times Set$ będzie funktorem takim, że  $\Delta(C)=(C, C)$. Zaczniemy od szukania funktora dołączonego do niego z lewej strony, czyli $L:Set\times Set\to Set$ takiego, że
    $$\Hom(L(X\times Y), Z)\cong \Hom((X, Y), \Delta(Z)),$$
    gdzie $\Hom$ to zbiory morfizmów w odpowiednich kategoriach. $L(X, Y)=X\times Y\in Set$. Wtedy dowolną funkcję $(X, Y)\to Z$ możemy przedstawić jako funkcję z $X\times Y$ i vice versa.
\end{enumerate}
\bigskip

% $Set_*$ i $Set$: trzeba dokleić punkt bazowy sumą rozłączną (to lewy)

% Pierścienie z $1$ a po prostu pierścieniami: doklejam $\Z$.

Teraz $\Delta:Set\to Set\times Set$, lewy dołączony to suma rozłączna, a prawy to iloczyn kartezjański; morał: koprodukt jest dołączony z prawej do $\Delta$, a $\Delta$ z prawej do produktu

Teraz $Set\to Set$ taki, że $X\mapsto Set(Y, X)$ morfizmy idące w ten obiekt. Wtedy lewo-dołączony to produkt $Set(L(X)=X\times Y, Z)\cong Set(X, Set(Y, Z))$; na II nazwą to Currying


Tensor produkt: jako o $R-Mod(V, Hom_R(W, U))\cong R-Mod(L(V), U)$, wtedy lewy funktor dołączony to $V\otimes W$; zwykle iloczyn tensorowy nie ma do siebie lewo dolaczonego

Mamy funktor zapominania $U:FinGrp\to FinSet$, który nie ma lewego funktora dołączonego, bo $FinSet(1, U(G))$, to jeśli mamy $FinGrp(L(1), G)$, bierzemy $p>|L(1)|$ liczbę pierwszą i jako $G=\Z_p$. To wtedy mamy tylko trywialny mofizm $L(1)\to G$, a w zbiorach jest ich dużo.

Załóżmy, że $\Cc$ ma produkty. funktor prawo dołączony do $-\times X$ jest funktorem eksponencjalnym $-^Y$, o ile istnieje. Core-compact spaces ma obiekty eksponencjalnie, jest tu podzbiór "lokalnie zwarte przestrzenie Hausdorffa" ($X^Y$ z topologią zwarto-otwarta: baza otoczeń indukowana przez $K\subseteq Y$ zwarty, $U\susbeteq X$ otwarty $V_{K,U}=\{f:f(K)\subseteq U\}$)

\subsection{Definicja bez użycia zbiorów}

\begin{definition}{}{}
$\epsilon:LR\implies 1$ to counit, $\eta:1\implies RL$ to unit
  
  \begin{center}
    \begin{tikzcd}
      L\arrow[r, Rightarrow, "1_L\neta"]\arrow[dr, "1_L", Rightarrow] & LRL\arrow[d, Rightarrow, "\epsilon 1_L"]\\ 
                                         & L
    \end{tikzcd}
  \end{center}
  \begin{center}
    \begin{tikzcd}
      R\arrow[r, Rightarrow, "\neta1_R"]\arrow[dr, "1_R", Rightarrow] & RLR\arrow[d, Rightarrow, "1_R\epsilon"]\\ 
                                         & R
    \end{tikzcd}
  \end{center}
\end{definition}

$k[-]=L:Set\to Vect$ i $U=R:Vect\to Set$ to funktor zapominania; wtedy $RL$ to zbiór kombinacji formalnych kombinacji liniowych, czyli dla każdego $X\to RL(X)$ jest włożenie. W drugą stronę $k[V]\to V$ też działa

Tutaj jeszcze raz powtórzyć przykłady z wcześniej


\begin{theorem}{}{}
  Obie definicje są równoważne
\end{theorem}

\begin{proof}
  dowód strona 124 w emily
\end{proof}



