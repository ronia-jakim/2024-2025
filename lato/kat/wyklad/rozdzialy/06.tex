\section{24.03.2025}{Kategoria algebr i Eilenberga-Moore'a}

% tutaj o przestrzeniach afinicznych jako o intuicji

{\color{red}tutaj kiedyś będzie wzmianka o modułach}

% Moduł nad pierścieniem $R$ to zbiór w którym zdefiniowane jest dodawanie elementów oraz mnożenie przez skalary z $R$. Mając więc dowolny zbiór $M$ możemy zdefiniować $TM$ jako zbiór wszystkich formalnych kombinacji liniowych elementów z $M$ o współczynnikach w $R$. $\eta_M:M\to TM$ niech przypisuje elementowi $m$ jego samego traktowanego jako kombinację liniową, a $\mu_M:T(TM)\to TM$ niech "wypłaszcza" podwójną sumę w pojedynczą sumę. Pozostaje zdefiniować ewaluowanie sumy jako $ev:TM\to M$. 

\begin{definition}{kategoria algebr}{}
  Niech $(T, \eta, \mu)$ będzie monadą na kategorii $\Cc$. Definiujemy wtedy \buff{kateogrię algebr} na $T$, oznaczane $\alg_T$, jako kategorię której
  \begin{description}[labelindent=5mm]
    \item[obiekty] to pary $(\theta, c)$, gdzie $\theta: Tc\to c$
    \item[morfizmy] $\alg_T((\theta, c), (\theta', c'))$ to odwzorowania $f\in \Cc(c, c')$ takie, że komutuje diagram
      \begin{center}
        \begin{tikzcd}
          Tc\arrow[r, "Tf"]\arrow[d, "\theta"] & Tc'\arrow[d, "\theta'"] \\ 
          c\arrow[r, "f"] & c'
        \end{tikzcd}
      \end{center}
  \end{description}
\end{definition}

Naturalnie, pytamy o istnienie obiektów początkowych i końcowych w tej kategorii.

\begin{example}
  Niech $T\equiv c$ będzie funktorem stałym. Wtedy $(\theta, x)$ jest obiektem początkowym, jeśli dla każdego $(\psi, y)$ jest dokładnie jeden komutujący diagram 
  \begin{center}
    \begin{tikzcd}[column sep=large, row sep=large]
      Tx=c\arrow[r, "Tf=id_c"]\arrow[d, "\theta" left] & c=Ty\arrow[d, "\psi"]\\ 
      x\arrow[r, "f" below] & y
    \end{tikzcd}
  \end{center}
  czyli $\psi=f\circ\theta$. Możemy wywnioskować, że $(\theta, x)=(id_c, c)$ i wtedy dla każdego innego $(\psi, y)$ będzie jedyny morfizm $f=\psi$ spełniający diagram.
\end{example}

\subsection{Kategoria Eilenberga-Moore'a}

\begin{definition}{Eilenberg-Moore}{}
  Kategoria \buff{Eilenberga-Moore'a} dla $T$ (kategorię $T$-algebra), oznaczaną jako $\Cc^T\subseteq\alg_T$, jest podkategorią $\alg_T$ w której
  \begin{description}[labelindent=5mm]
    \item[obiekty] to pary $(\theta, a)$, $a\in\Cc$, $\theta:Ta\to a$ dla których komutują diagramy w $\Cc$
      \begin{center}
        \begin{tikzcd}
          a\arrow[r, "\eta_a" above]\arrow[dr, "1_a" below left] & Ta\arrow[d, "\theta"] & & T^2a\arrow[r, "\mu_a" above]\arrow[d, "Ta" left] & Ta\arrow[d, "\theta" right] \\ 
                                                                 & a & & Ta \arrow[r, "\theta" below] & a
        \end{tikzcd}
      \end{center}
    \item[morfizmy] $f:(\theta, a)\to (\phi, b)$ są mapami $f:a\to b$ w $\Cc$ takie, że komutuje diagram
      \begin{center}
        \begin{tikzcd}
          Ta\arrow[r, "Tf" above] \arrow[d, "\theta" left] & Tb\arrow[d, "\phi" right] \\ 
          a\arrow[r, "f" below] & b
        \end{tikzcd}
      \end{center}
  \end{description}
\end{definition}

Chcemy teraz pokazać, że dla dowolnej monady możemy stworzyć parę funktorów sprzężonych. 

\begin{lemma}{}{}
  Istnieją funktory
  \begin{center}
    \begin{tikzcd}
      \Cc\arrow[r, "L"] & \Cc_T\arrow[r, "J"] & \Cc^T\arrow[r, "R"] & \Cc_T
    \end{tikzcd}
  \end{center}
  takie, że $RJL=T$.
\end{lemma}

\begin{proof}
  Zaczynamy od zdefiniowania wszystkich funktorów.

  $$L:\Cc\to\Cc_T$$
  $$L(c)=c,\quad L(f)=\eta\circ f$$
  
  Wypada sprawdzić, czy $\eta\circ(gf)=L(gf)=L(g)L(f)=(\eta\circ g)\circ(\eta\circ f)$, czyli czy komutuje największy prostokąt w diagramie
  \begin{center}
    \begin{tikzcd}[column sep=large, row sep=large]
      c\arrow[r, "f"]\arrow[d, "=", sloped] & c'\arrow[r, "g"]\arrow[d, "\eta"] & c''\arrow[r, "\eta"]\arrow[d, "T(\eta)\circ\eta?"] & Tc''\arrow[d, "=", sloped]\\ 
      c\arrow[r, "\eta\circ f" below] & Tc'\arrow[r, "\begin{matrix}T(\eta\circ g)=\\=T(\eta)\circ T(g)\end{matrix}" below] & T^2c''\arrow[r, "\mu_{c''}" below] & Tc''
    \end{tikzcd}
  \end{center}

  $$J:\Cc_T\to\Cc^T$$
  $$\color{red}J(c)=(Tc, \mu\circ\eta),\quad J(f:c\to Tc')=\mu \circ Tf:Tc\to Tc'$$

  $$R:\Cc^T\to \Cc$$
  $$R(c, \theta)=c,\quad R(f:(c, \theta)\to(c',\theta'))=f:c\to c'$$

  Tutaj składanie działa bez problemów, bo $f$ było morfizmem w $\Cc$.

  Teraz pokażemy, że $RJL=T$. Dla obiektów mamy:
  $$RJL(c)=RJ(c)=R(Tc)=Tc,$$
  a dla dowolnego morfizmu $f:c\to c'$
  $$RJL(f)=RJ(\eta\circ f)=R(\mu\circ T(\eta\circ f))=R(\mu\circ T(\eta)\circ Tf)=R(1_T\circ Tf)=R(Tf)=Tf.$$

  \emph{Pozostawiam ten dowód jako pomnik dla oryginalnego stwierdzenia, że $RJ$ i $L$ oraz $R$ i $JL$ to dwie pary funktorów dołączonych.}
\end{proof}

\begin{theorem}{}{}
  Dla dowolnej pary funktorów sprzężonych \begin{tikzcd}\Cc\arrow[r, bend left=20, "L" above] & \Dd\arrow[l, bend left=20, "R" below]\end{tikzcd} indukujących monadę $(T,\eta,\mu)$ istnieją jedyne funktory $J$ i $K$
  \begin{center}
    \begin{tikzcd}[column sep=6em, row sep=6em]
      \Cc_T\arrow[r, dashed, "\exists!J" above]\arrow[dr, yshift=2pt, xshift=5pt, "R_T"] & \Dd\arrow[r, "\exists!L" above, dashed]\arrow[d, "R" {yshift=1.5ex} right, xshift=5pt, shorten >=1.5ex] & \Cc^T\arrow[dl, "R^T" above left, xshift=-5pt, yshift=2pt]\\ 
                                                                                         & \Cc\arrow[u, "L" left, xshift=-5pt, yshift=1.5ex, shorten >=1.5ex]\arrow[ur, "L^T" below right, xshift=5pt, yshift=-2pt]\arrow[ul, "L_T" below left, xshift=-5pt, yshift=-2pt]
    \end{tikzcd}
  \end{center}
  takie, że prawy i lewy trójkąt komutuje.
\end{theorem}

\begin{proof}
  Emily Proposition 5.2.12.
\end{proof}



