\section{24.03.2025}{Druga nieobecność}

% tutaj o przestrzeniach afinicznych jako o intuicji

{\color{red}tutaj kiedyś będzie wzmianka o modułach}

% Moduł nad pierścieniem $R$ to zbiór w którym zdefiniowane jest dodawanie elementów oraz mnożenie przez skalary z $R$. Mając więc dowolny zbiór $M$ możemy zdefiniować $TM$ jako zbiór wszystkich formalnych kombinacji liniowych elementów z $M$ o współczynnikach w $R$. $\eta_M:M\to TM$ niech przypisuje elementowi $m$ jego samego traktowanego jako kombinację liniową, a $\mu_M:T(TM)\to TM$ niech "wypłaszcza" podwójną sumę w pojedynczą sumę. Pozostaje zdefiniować ewaluowanie sumy jako $ev:TM\to M$. 

\begin{definition}{kategoria algebr}{}
  Niech $(T, \eta, \mu)$ będzie monadą na kategorii $\Cc$. Definiujemy wtedy \buff{kateogrię algebr} na $T$, oznaczane $\alg_T$, jako kategorię której
  \begin{description}[labelindent=5mm]
    \item[obiekty] to pary $(\theta, c)$, gdzie $\theta: Tc\to c$
    \item[morfizmy] $\alg_T((\theta, c), (\theta', c'))$ to odwzorowania $f\in \Cc(c, c')$ takie, że komutuje diagram
      \begin{center}
        \begin{tikzcd}
          Tc\arrow[r, "Tf"]\arrow[d, "\theta"] & Tc'\arrow[d, "\theta'"] \\ 
          c\arrow[r, "f"] & c'
        \end{tikzcd}
      \end{center}
  \end{description}
\end{definition}

Naturalnie, pytamy o istnienie obiektów początkowych i końcowych w tej kategorii.

\begin{example}
  Niech $T\equiv c$ będzie funktorem stałym. Wtedy $(\theta, x)$ jest obiektem początkowym, jeśli dla każdego $(\psi, y)$ jest dokładnie jeden komutujący diagram 
  \begin{center}
    \begin{tikzcd}[column sep=large, row sep=large]
      Tx=c\arrow[r, "Tf=id_c"]\arrow[d, "\theta" left] & c=Ty\arrow[d, "\psi"]\\ 
      x\arrow[r, "f" below] & y
    \end{tikzcd}
  \end{center}
  czyli $\psi=f\circ\theta$. Możemy wywnioskować, że $(\theta, x)=(id_c, c)$ i wtedy dla każdego innego $(\psi, y)$ będzie jedyny morfizm $f=\psi$ spełniający diagram.
\end{example}

\subsection{Kategoria Eilenberga-Moore'a}

\begin{definition}{Eilenberg-Moore}{}
  Kategoria \buff{Eilenberga-Moore'a} dla $T$ (kategorię $T$-algebra), oznaczaną jako $\Cc^T\subseteq\alg_T$, jest podkategorią $\alg_T$ w której
  \begin{description}[labelindent=5mm]
    \item[obiekty] to pary $(\theta, a)$, $a\in\Cc$, $\theta:Ta\to a$ dla których komutują diagramy w $\Cc$
      \begin{center}
        \begin{tikzcd}
          a\arrow[r, "\eta_a" above]\arrow[dr, "1_a" below left] & Ta\arrow[d, "\theta"] & & T^2a\arrow[r, "\mu_a" above]\arrow[d, "Ta" left] & Ta\arrow[d, "\theta" right] \\ 
                                                                 & a & & Ta \arrow[r, "\theta" below] & a
        \end{tikzcd}
      \end{center}
    \item[morfizmy] $f:(\theta, a)\to (\phi, b)$ są mapami $f:a\to b$ w $\Cc$ takie, że komutuje diagram
      \begin{center}
        \begin{tikzcd}
          Ta\arrow[r, "Tf" above] \arrow[d, "\theta" left] & Tb\arrow[d, "\phi" right] \\ 
          a\arrow[r, "f" below] & b
        \end{tikzcd}
      \end{center}
  \end{description}
\end{definition}





% \begin{lemma}{}{}
%   Dla każdej monady $(T,\eta, \mu)$ na kategorii $\Cc$ istnieje para funktorów sprzężonych 
%   \begin{center}
%     \begin{tikzcd}
%       \Cc\arrow[r, "L" above, bend left=20] & \alg_T\arrow[l, bend left=20, "R" below] 
%     \end{tikzcd}
%   \end{center}
%   która indukuje monadę $(T,\eta,\mu)$.
% \end{lemma}
%
% \begin{proof}
%   ja kiedyś będę miała czas w życiu
% \end{proof}
%
