\section{24.03.2025}{Druga nieobecność}

tutaj o przestrzeniach afinicznych jako o intuicji

\begin{definition}{Eilenberg-Moore [kategora algebr]}{}
  Niech $\Cc$ będzie kategorią, a $(T,\eta,\mu)$ monadą na niej. Definiujemy kategorię \buff{Eilenberga-Moore'a} dla $T$ (kategorię $T$-algebra), oznaczaną jako $\boldsymbol{\alg_T}$, jako kategorię której
  \begin{itemize}
    \item obiekty to pary $(\theta, a)$, $a\in\Cc$, $\theta:Ta\to a$ dla których komutują diagramy w $\Cc$
      \begin{center}
        \begin{tikzcd}
          a\arrow[r, "\eta_a" above]\arrow[dr, "1_a" below left] & Ta & & T^2a\arrow[r, "\mu_a" above]\arrow[d, "Ta" left] & Ta\arrow[d, "a" right] \\ 
                                                                 & A & & TA \arrow[r, "a" below] & A
        \end{tikzcd}
      \end{center}
    \item morfizmy $f:(\theta, a)\to (\phi, b)$ są mapami $f:a\to b$ w $\Cc$ takie, że komutuje diagram
      \begin{center}
        \begin{tikzcd}
          Ta\arrow[r, "Tf" above] \arrow[d, "\theta" left] & Tb\arrow[d, "\phi" right] \\ 
          a\arrow[r, "f" below] & b
        \end{tikzcd}
      \end{center}
  \end{itemize}
\end{definition}

\begin{example}[m]
\item Niech $T=c$ będzie funktorem stałym. Pokażemy, że obiektem początkowym w kategorii $\alg_T$ jest para $(id_c, c)$. Niech $(\theta, x)$ będzie dowolnym obiektem $\alg_T$. Mamy wtedy diagram
  \begin{center}
    \begin{tikzcd}[column sep=large, row sep=large]
      Tc=c \arrow[r, "T\theta'=id_c"]\arrow[d, "id_c" left] & Tx=c\arrow[d, "\theta" right] \\ 
      c\arrow[r, dashed, "\exists!\theta'=\theta" below] & x
    \end{tikzcd}
  \end{center}
\end{example}

\begin{lemma}{}{}
  Dla każdej monady $(T,\eta, \mu)$ na kategorii $\Cc$ istnieje para funktorów sprzężonych 
  \begin{center}
    \begin{tikzcd}
      \Cc\arrow[r, "L" above, bend left=20] & \alg_T\arrow[l, bend left=20, "R" below] 
    \end{tikzcd}
  \end{center}
  która indukuje monadę $(T,\eta,\mu)$.
\end{lemma}

\begin{proof}
  ja kiedyś będę miała czas w życiu
\end{proof}

