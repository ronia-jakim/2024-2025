\section{10.03.2025}{Funktory dołączone własności [wieczny WIP]}

\subsection{Dowód równoważności}

\begin{theorem}{}{}
  Dwie definicje funktorów dołączonych z poprzedniego wykładu są równoważne, tzn. naturalne transformacje $H$, $E$
  \begin{center}
    \begin{tikzcd}
      \Dd(L-,-)\arrow[Rightarrow, r, bend right=20, "H" below] & \Cc(-, R-)\arrow[l, Rightarrow, bend right=20, "E" above]
    \end{tikzcd}
  \end{center}
  istnieją $\iff$ istnieją dwie naturalne transformacje $\epsilon:LR\to 1_\Dd$ oraz $\eta:1_\Cc\to RL$ dla których komutują diagramy
  \begin{center}
    \begin{tikzcd}
      L\arrow[r, Rightarrow, "1_L\eta"]\arrow[dr, "1_L" below left, Rightarrow] & LRL\arrow[d, Rightarrow, "\epsilon 1_L"] & & R\arrow[r, "\eta1_R" above, Rightarrow] \arrow[dr, "1_R" below left, Rightarrow] & RLR\arrow[d, "1_R\epsilon" right, Rightarrow]\\ 
                                                                                 & L & & & R
    \end{tikzcd}
  \end{center}
\end{theorem}

\begin{proof}
  Niech $f:c'\to c$ będzie morfizmem w $\Dd$, a $g:d'\to d$ - morfizmem w $\Cc$.
  % $\implies$

  % Niech $f:c'\to c$ będzie morfizmeme w $\Dd$, a $g:d'\to d$ - morfizmem w $\Cc$. Niech $\phi\in \Dd(Lc, d)$, wtedy mamy diagram
  % \begin{center}
  %   \begin{tikzcd}
  %     Lc' \arrow[r, "Lf" above] & Lc \arrow[d, "\phi" right] \\ 
  %     d' & d\arrow[l, "g" below] \\ 
  %     Rd' & Rd\arrow[l, "Rg" below]
  %   \end{tikzcd}
  % \end{center}
  Zacznijmy od zdefiniowania szukanych przekształceń naturalnych na obiektach. Niech $\eta_c$
  \begin{center}
    \begin{tikzcd}[row sep=small]
      \Dd(Lc, Lc)\arrow[r, "H_{c, Lc}"] & \Cc(c, RLc)\\ 
      1_{Lc}\arrow[r] & \eta_c=H(1_{Lc})
    \end{tikzcd}
  \end{center}
  a $\epsilon_d$ definiujemy analogicznie używając $E$.

  W drugą stronę, $H(\phi)$ definiujemy mając $\eta$ oraz $\epsilon$. Dla $\phi:Lc\to d$ definiujemy
  $$H(\phi):=R\phi\circ\eta_c,$$
  które bierze coś z $c$ i oddaje $RLc$. Z drugiej strony bierzemy $\psi: c\to Rd$
  $$E(\psi):=\epsilon_{d}\circ L\psi.$$

  $\implies$

  Zakładamy, że $H$ i $E$ są naturalne i pokazujemy naturalność $\eta$, czyli komutowanie diagramu
  \begin{center}
    \begin{tikzcd}[column sep=large, row sep=large]
      RL(c')\arrow[d, "RL(f)" left] & \arrow[l, "\eta_{c'}" above] 1_\Cc(c')\arrow[d, "1_\Cc(f)"]\\ 
      RL(c) & 1_\Cc(c) \arrow[l, "\eta_{c}"]\\ 
    \end{tikzcd}
  \end{center}

  \begin{align*}
    RLf\circ \eta_{c'}&\overset{\text{def. }\eta}{=}RLf\circ H(1_{Lc'})=\\ 
                      &\overset{\text{funktorialność }H}{=}H(Lf\circ 1_{Lc'})=\\ 
                      &= H(1_{Lc}\circ Lf)=\\ 
                      &=H(1_{Lc})\circ f=\\ 
                      &\overset{\text{def. }\eta}{=}\eta_c\circ f.
  \end{align*}
  Analogicznie należy sprawdzić naturalność $\epsilon$.

  Pozostaje jeszcze udowodnić, że zdefiniowane przez nas $\eta$ i $\epsilon$ spełnia warunek trójkąta w definicji, tzn. komutują diagramy
  \begin{center}
    \begin{tikzcd}
      L\arrow[r, orange, Rightarrow, "1_L\eta"]\arrow[dr, "1_L" below left, Rightarrow] & LRL\arrow[d, Rightarrow, blue, "\epsilon 1_L"] & & R\arrow[r, "\eta1_R" above, Rightarrow] \arrow[dr, "1_R" below left, Rightarrow] & RLR\arrow[d, "1_R\epsilon" right, Rightarrow]\\ 
                                                                                 & L & & & R
    \end{tikzcd}
  \end{center}
  Ograniczymy się do sprawdzenia lewego diagramu. 
  \begin{center}
    \begin{tikzcd}
      Lc\arrow[r, "L(\eta_c)"]\arrow[dr, "1_{Lc}" below left] & LRLc\arrow[d, "\epsilon_{Lc}"] \\ 
                               & Lc
    \end{tikzcd}
  \end{center}
  $$\epsilon_{Lc}L(\eta_c)=E(1_{RLc})L(\eta_c)=E(1_{RLc}\eta_c)=EH(1_{Lc})=1_{Lc}$$

  % {\color{red} należy jeszcze pokazać trójkąty}

  % {\color{red} tutaj pokazać, że $H$ jest naturalne}
  $\impliedby$

  Wychodzimy teraz z założenia, że $\eta:1_\Cc\implies RL$ i $\epsilon:LR\implies 1_\Dd$ to naturalne przekształcenia, czyli z komutowania diagramów
  \begin{center}
    \begin{tikzcd}
      c'\arrow[d, "f" left]\arrow[r, "\eta_{c'}"] & RLc'\arrow[d, "RLf"] & & d\arrow[d, "g" left] & LRd\arrow[d, "LRg"]\arrow[l, "\epsilon_d" above]\\ 
      c\arrow[r, "\eta_c" below] & RLc & & d' & LRd'\arrow[l, "\eta_{d'}"]
    \end{tikzcd}
  \end{center}
  dostajemy równości
  $$RLf\circ \eta_{c'}=\eta_c\circ f$$
  $$g\circ \epsilon_d=\epsilon_{d'}\circ LR g.$$
  Powinniśmy najpierw pokazać, że $H$ i $E$ są naturalne. Zrobimy to tylko dla $H$. Interesuje nas diagram
  \begin{center}
    \begin{tikzcd}[column sep=large, row sep=large]
      \Dd(Lc, d)\arrow[d, "(Lf; g)" left]\arrow[r, "H_{c;d}" above] & \Cc(c, Rd) \arrow[d, "(f; Rg)"]\\ 
      \Dd(Lc', d')\arrow[r, "H_{c';d'}" below] & \Cc(c', Rd')
    \end{tikzcd}
  \end{center}

  $$EH(\phi)=E(R\phi\circ \eta)=E(R\phi)E(\eta)=\epsilon LR\phi \epsilon L\eta$$
\end{proof}

\begin{theorem}{}{}
  Istnieje bijekcja między zbiorem par naturalnych przekształceń $(H, E)$ oraz $(\eta, \epsilon)$.
\end{theorem}

\begin{proof}
  \color{red}TO DO
\end{proof}

\subsection{Funktory dołączone a granice}

\begin{theorem}{}{}
  Niech $L\dashv R$ będzie parą funktorów dołączonych. Wtedy $L$ zachowuje granice proste, a $R$ - granice odwrotne.
\end{theorem}

Przypomnijmy że kogranica (granica prosta) funktora $f:\mathcal{I}\to\Cc$ spełnia dla każdego $g$ diagram
\begin{center}
  \begin{tikzcd}
    f_i\arrow[rr, "\alpha" above]\arrow[dr, "L\phi_i" below left]\arrow[ddr, bend right=30, "\psi_i" below left] & & f_j\arrow[dl, "\phi_j" below right]\arrow[ddl, "\psi_j" below right, bend left=30]\\ 
                                                                                                 & \varinjlim f\arrow[d, dashed, "\exists!"]\\ 
    & g
  \end{tikzcd}
\end{center}

\begin{proof}
  Pokażemy tylko, że lewo dołączony funktor zachowuje kogranice, tj. dla dowolnego $d\in \Dd$ zachodzi diagram
  \begin{center}
    \begin{tikzcd}
      Lf_i\arrow[rr, "L\alpha" above]\arrow[dr, "L\phi_i" below left]\arrow[ddr, bend right=30, "\psi_i" below left] & & Lf_j\arrow[dl, "L\phi_j" below right]\arrow[ddl, "\psi_j" below right, bend left=30]\\  
                                                                                                                         & L\varinjlim f\arrow[d, dashed, "\exists!"]\\ 
                                                                                                                         & d
    \end{tikzcd}
  \end{center}

  Z uniwersalnej własności kogranicy mamy diagram
  \begin{center}
    \begin{tikzcd}
      f_i\arrow[rr, "\alpha" above]\arrow[dr, "L\phi_i" below left]\arrow[ddr, bend right=30, "\theta_i" below left] & & f_j\arrow[dl, "\phi_j" below right]\arrow[ddl, "\theta_j" below right, bend left=30]\\ 
                                                                                                   & \varinjlim f\arrow[d, dashed, "\exists!u"]\\ 
      & Rd
    \end{tikzcd}
  \end{center}
  Nakładamy na niego funktor $L$. Potrzebujemy też strzałek $L\eta_i:Lf_i\to LRLf_i$ przychodzących z naturalnej transformacji $\eta: 1_\Cc\implies RL$. Dodatkowo wiemy, że $\epsilon_d:Lrd\to d$ istnieje i jest w dodatku jedyne. Mamy więc diagram
  \begin{center}
    \begin{tikzcd}[
      column sep=large, row sep=large, every matrix/.append style={name=mymatrix}
      ]
      & Lf_i
      \arrow[rr, "L\alpha"]\arrow[dl, "L\eta_i" above left]\arrow[dr, "L\phi_i" below left]
      \arrow[ddr, bend right=30, "L\theta_i" left] 
      \arrow[thick, "\psi_i", dddr, green, rounded corners, to path={
        (\tikztostart.south)
        --(mymatrix-2-1.east)
        to[bend right=20] (mymatrix-3-3.north west)
        -- ([xshift=-.5em]mymatrix-4-3.west)
        |- ([xshift=-.2em]mymatrix-4-3)
      }]
      & & Lf_j\arrow[dl, "L\phi_j" below right]\arrow[dr, "L\eta_j" above right] 
      \arrow[ddl, bend left=30, "L\theta_j" right] 
      \arrow[thick, "\psi_j" above, dddl, green, rounded corners, to path={
        (\tikztostart.south)
        --(mymatrix-2-5.west)
        to[bend left=20] (mymatrix-3-3.north east)
        -- ([xshift=.5em]mymatrix-4-3.east)
        |- ([xshift=.2em]mymatrix-4-3)
      }]
      \\ 
      LRLf_i\arrow[drr, "LR\psi_i" below left, bend right=20] & & L\varinjlim f\arrow[d, "\exists! Lu", dashed] & & LRLf_j \arrow[dll, "LR\psi_j" below right, bend left=20] \\ 
             & & LRd \arrow[d, "\exists!\epsilon_d"] \\ 
             & & d
    \end{tikzcd}
  \end{center}
  w którym długie zielone strzałki są konsekwencją złożenia $\epsilon_d\circ LR\psi_i=\psi_i$. Dostajemy więc $\epsilon_d\circ Lu$ jako jedyną strzałkę $L\varinjlim f\to d$ komutującą z $\psi_i$ oraz $\psi_j$.


  % Chcemy pokazać, że diagram z definicji granicy po nałożeniu funktora $L$,
  % \begin{center}
  %   \begin{tikzcd}
  %     Lf_i\arrow[r, "L\phi_i" above] \arrow[dr] & L\varinjlim f\arrow[d, "\exists!", dashed] \\ 
  %                                  & Lg
  %   \end{tikzcd}
  % \end{center}
  % jest nadal prawdziwy.
  %
  % \begin{center}
  %   \begin{tikzcd}[row sep=large, column sep=large]
  %     f_i\arrow[r, "\phi_i" above]\arrow[d] & \varinjlim f\arrow[d] \arrow[dd, dashed, bend left=30, "\phi_\infty" right]\\ 
  %     RL f_i\arrow[r, "RL\phi_i" above]\arrow[dr, "R\phi_i" below left] & RL\varinjlim f  \\ 
  %                                       & Rg
  %   \end{tikzcd}
  % \end{center}
\end{proof}

% tutaj cos o kojadrze jak mamy funkcje w przestrzeniach wektorowych

\subsection{Moduły}

Jeśli $R$ jest pierścieniem z $1$, to powiemy, że $M$ jest $R$-modułem, jeśli
\begin{itemize}
  \item $(M, +)$, jest grupą abelową
  \item oraz $R$ działa na $M$ tak, że 
    $$1x=x,\quad rsx=r(sx)$$
    $$(r+s)(x+y)=(r+s)x+(r+s)y=rx+sx+ry+sy$$
\end{itemize}

Grupy abelowe to $\Z$-moduły. Przestrzenie liniowe nad ciałem $k$ to $k$-moduły.

\begin{definition}{moduł projektywny}{}
  Mówimy, że $R$-moduł $P$ w kategorii $R$-modułów jest projektywny, jeśli dla każdego surjektywnego homomorfizmu $f:N\twoheadrightarrow M$ oraz każdego homomorfizmu $g:P\to M$ istnieje homomorfizm modułów $h:P\to N$ taki, że $fh=p$. Innymi słowy, komutuje diagram
  \begin{center}
    \begin{tikzcd}
      & N\arrow[d, twoheadrightarrow, "f"]\\ 
      P\arrow[ur, dashed, "\exists\;h"]\arrow[r, "g" below] & M
    \end{tikzcd}
  \end{center}
\end{definition}

\begin{example}
  Dla każdego $R$ oraz $n$ wolny moduł $R^n$ jest modułem projektywnym. Niech $x_1,..., x_n$ będą generatorami $R^n$. Dla każdego $i$ wybieramy jeden element $n_i\in f^{-1}(g(x_i))$. Definiujemy $h(x_i)=n_i$.
\end{example}

Argument z przykładu uogólnia się na stwierdzenie, że każdy moduł jest \textbf{obrazem pewnego modułu projektywnego}.

Dualnie definiujemy moduły injektywne.

\begin{definition}{moduł injektywny}{}
  Dla każdego injektywnego $f:M\to N$ oraz dla każdego $g:M\to Q$ istnieje $h:N\to Q$ takie, że komutuje diagram
  \begin{center}
    \begin{tikzcd}
      M\arrow[r, "f" above, hookrightarrow]\arrow[d, "g" left] & N\arrow[dl, "\exists\;h" below right, dashed]\\ 
      Q
    \end{tikzcd}
  \end{center}
\end{definition}

\begin{example}
Liczby wymierne $\Q$ są injektywnym $\Z$-modułem.
\end{example}

\begin{theorem}{}{}
  Dla każdego $R$-modułu $M$ istnieje injektywny moduł $Q$ taki, że $M\hookrightarrow Q$.
\end{theorem}

\begin{proof}
  \color{red}a świat pali się \emoji{fire}
\end{proof}



