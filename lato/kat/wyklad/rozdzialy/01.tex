\chapter{Początek końca}

W 1945 Eilenberg oraz Mac Lane napisali książkę "General theory of natural equivalences". Jest to powszechnie uznawane jako początek ery abstrakcyjnego nonsenu.

\section{24.02.2025}{Podstawowe definicje}

\subsection{Przykłady kategorii}

\begin{definition}{kategoria}{}
  Kategoria (lokalnie mała) $\Cc$ składa się z:
  \begin{itemize}
    \item obiektów $\Ob(\Cc)$
    \item oraz zbiorów morfizmów dla wszystkich par $A,B\in\Ob(\Cc)$ oznaczanego $\Cc(A, B)=\Hom_\Cc(A, B)$, które spełniają:
      \begin{itemize}
        \item $id_X\in\Cc(X, X)$
        \item składają się w dobry sposób, tzn. mamy dobrze określone odwzorowanie
          $$\Cc(A, B)\times\Cc(B, C)\to \Cc(A, C),$$
          które jest łączne.
      \end{itemize}
  \end{itemize}
\end{definition}

Powiemy, że kategoria jest \buff{mała}, jeśli jej obiekty są zbiorem, a nie klasą. 

Dla wygody oznaczymy
$$\Cc_0:=\Ob(\Cc)$$
a jako $\Cc_1$ będziemy rozumieć wszystkie morfizmy w kategorii $\Cc$.

Rozważmy kilka prostych przykładów kategorii.

\begin{example}[m]
  \item Kategoria $Set$, której obiekty $Set_0$ to wszystkie zbiory, a $Set_1$ to funkcje między zbiorami z normalnym składaniem funkcji.
  \item $Set_*$ to kategoria zbazowanych zbiorów, tzn. jej obiektami są pary $(X, x_0)$, gdzie $X$ to zbiór, a $x_0\in X$. Morfizmy muszą wtedy zachowywać wyróżniony punkt: $f:(X, x_0)\to (Y, y_0)$, $f(x_0)=y_0$.
  \item $Top$ to kategoria, której obiekty to przestrzenie topologiczne, a $Top_1$ to funkcje ciągłe między nimi.
  \item $Toph$ to kategoria przestrzeni topologicznych, w której morfizmy to klasy homotopii odwzorowań między przestrzeniami. To znaczy, jeśli $X,Y\in\Ob(Toph)$ oraz $f_0,f_1:X\to Y$ jest ciągłym odwzorowaniem, dla którego istnieje ciągłe przekształcenie
    $$F:X\times[0,1]\to Y$$
    takie, że $F(x, 0)=f_0(x)$ oraz $F(x, 1)=f_1(x)$, to $f_0=f_1$ jako morfizm w kategorii $Toph$.

    Pozostaje sprawdzić, że jeśli $f$, $f'$ oraz $g$, $g'$ to pary homotopijnie równoważnych odwzorowań, to wówczas $f\circ g$ jest homotopijnie równoważne $f'\circ g'$.
  \item Kategoria $Hask$, której obiekty to typy w Haskelly, a morfizmy to klasy programów.
  \item Kategoria relacji $Rel$, w której obiektami $Rel_0$ są zbiory, a morfizmami są podzbiory produktu, tzn. $Rel(X, Y)$ zawiera wszystkie $S\subseteq X\times Y$. Wówczas składanie $S\subseteq X\times Y$ oraz $R\subseteq Y\times Z$ definiujemy jako zbiór
    $$S\circ R=\{(x, z)\;:\;(\exists\;y\in Y)\;xRy\;\land\;ySz\},$$
    gdzie $xRy$ oznacza, że $(x,y)\in R$. Złożenie to działa jak połączenie dwóch relacji spójnikiem "i".
  \item Niech $R$ będzie tranzytywną i zwrotną relacją na zbiorze $X$. Definiujemy wtedy kategorię $\Cc$ o obiektach $\Cc_0=X$ będących elementami zbioru $X$, a morfizmy między $a,b\in X$ to zbiór $1$-elementowy $\Cc(a, b)=\{\star\}$, gdy $xRy$ jest prawdą lub zbiór pustym w przeciwnym wypadku.

    Szczególnym przypadkiem tej kategorii jest topologia na przestrzeni topologicznej, gdzie relacja $R$ to zawieranie zbiorów otwartych.
  \item Graf skierowany tworzy kategorię, której obiektami są jego wierzchołki, a morfizmy to zorientowane ścieżki.
\end{example}

\subsection{Funktory}

\begin{definition}{funktor}{}
  Funktor $F$ między kategoriamii $\Cc$ a $\Dd$
  \begin{itemize}
    \item każdemu obiektowi $X$ kategorii $\Cc$ przypisuje obiekt $F(X)$ kategorii $\Dd$
    \item każdemu morfizmowi $\phi\in\Cc(X, Y)$ przypisuje morfizm $F(\phi):F(X)\to F(Y)$ w kategorii $\Dd$ taki, że 
      \begin{itemize}
        \item $F(\psi\circ\phi)=F(\psi)\circ F(\phi)$
        \item $F(id_X)=id_{F(X)}$
      \end{itemize}
  \end{itemize}
\end{definition}

\begin{example}{}{}
  $Ab:Gr\to Ab$ to funktor między kategorią wszystkich grup a kategorią grup abelowych, który grupie $G$ przypisuje jej abelianizację $Ab(G)=G/[G,G]=G^{ab}$.
\end{example}

\begin{definition}{kategoria odwrotna}{}
  Przez \textbf{kategorię odwrotną} do kategorii $\Cc$ rozumiemy kategorię $\Cc^{op}$, której
  \begin{itemize}
    \item obiekty to obiekty oryginalnej kategorii: $\Ob(\Cc^{op})=\Ob(\Cc)$
    \item morfizmy $\Cc(X, Y)$ "odwracają się" $\Cc^{op}(Y, X)$.
  \end{itemize}
\end{definition}
%
% Istnieje funktor między kategorią a kategorią do niej odwrotną.

Mówimy, że funktor $F:\Cc\to \Dd$ jest \textbf{kowariantny}, a funktor $F:\Cc\to\Dd^{op}$ \text{kontrawariantny}.

Zdefiniujmy teraz \buff{kategorię funktorów} między kategoriami $\Cc$ a $\Dd$, $Fun(\Cc, \Dd)$, której obiekty to wszystkie funktory $F:\Cc\to\Dd$, a morfizmy to $\phi$ takie, że dla dowolnych $X, Y\in\Ob\Cc$ oraz $f:X\to Y$ komutuje diagram
\begin{center}
  \begin{tikzcd}
    F(X)\arrow[r, "\phi_X"] \arrow[d, "F(f)" left] & G(X)\arrow[d, "G(f)"] \\
    F(Y)\arrow[r, "\phi_Y"] & G(Y)
  \end{tikzcd}
\end{center}
Zbiór morfizmów w tej kategorii oznaczymy $\Nat(F, G)$ - \buff{naturalne przekształcenia} funktora $F$ w funktor $G$.

\begin{example}
  Cup product na kohomologiach $\cup:H^m(X)\otimes H^n(X)\to H^{m+n}(X)$ jest naturalnym przekształceniem między funktorami $H^m(-)\otimes H^n(-)$ i $H^{m+n}(-)$.
\end{example}

\begin{definition}{równoważność kategorii}{}
  Powiemy, że kategorie $\Cc$ i $\Dd$ są \buff{równoważne}, jeśli istnieją funktory $F:\Cc\to\Dd$ oraz $G:\Dd\to\Cc$ takie, że złożenie $F\circ G$ jest naturalnie izomorficzne do $Id_{\Dd}$, a $G\circ F$ - do $Id_{\Cc}$.
\end{definition}

\begin{example}
  Kategoria skończenie wymiarowych przestrzeni wektorowych nad ciałem $k$, $Vect_{k}^{fin}$, jest równoważna kategorii skończenie wymiarowych macierzy nad ciałem $k$, $Mat^{fin}(k)$.
\end{example}

{\color{red}GRUPOID PODSTAWOWY - dla p. top  $X$ obiekty to punkty $X$, a morfizmy to klasy homotopii ścieżek; jak weźmiemy konkretny punkt i popatrzymy na morfizmy $x\to x$ to mamy grupe podstawową zbazowaną w tym punkcie; grupoid to funktor z p. top w kategorię kategorii (zawęzić: kat. grupoidów); wtedy funkcja ciągła to morfizm między dwoma grupoidami, a homotopia to naturalna transformacj}
