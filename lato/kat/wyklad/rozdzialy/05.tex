\section{17.03.2025}{Pierwsza nieobecność}

\begin{adjustwidth}{200pt}{0pt}
\begin{flushright}\slshape
  A monad is just a monoid in the category of endofunctors, what's the problem
\end{flushright}
\end{adjustwidth}

\subsection{Co to właściwie ta monada?}

W programowaniu: sposób na nałożenie funkcji na wartość w opakowaniu. W OCamlu jest definiowana jako

\begin{lstlisting}[language=caml]
module type Monad = sig
  type 'a t
  val return : 'a -> 'a t
  val bind : 'a t -> ('a -> 'b t) -> 'b t
end
\end{lstlisting}

\begin{definition}{monada}{}
  Monada na kategorii $\Cc$ składa się z
  \begin{itemize}
    \item endofunktora $T:\Cc\to\Cc$,
    \item naturalnej transformacji $\eta:1_\Cc\to T$ (unit z funktorów dołączonych),
    \item naturalnej transformacji $\mu:T^2\to T$, która definiuje mnożenie na funktorze $T$
  \end{itemize}
  takich, że poniższe diagramy komutują w kategorii $\Cc^\Cc$
  \begin{center}
    \begin{tikzcd}[column sep=large, row sep=large]
      T^3\arrow[r, Rightarrow, "T\mu"] \arrow[d, Rightarrow, "\mu T"] & T^2\arrow[d, Rightarrow, "\mu"] & & T\arrow[r, "\eta T", Rightarrow] \arrow[dr, "1_T", Rightarrow] & T^2 \arrow[d, "\mu", Rightarrow] & T\arrow[l, Rightarrow, "T\eta"]\arrow[dl, "1_T", Rightarrow]  \\
      T^2\arrow[r, Rightarrow, "\mu"] & T & & & T
    \end{tikzcd}
  \end{center}
\end{definition}

Diagramy te są bardzo podobne do tych, które pojawiły się przy definiowaniu obiektu monoidalnego [\ref{deff:obiekt monoidalny}]. Nie jest to przypadkiem: monady są obiektem monoidalnym w kategorii endofunktorów $\Cc^\Cc$ z binarnym działaniem $\Cc^\Cc\times\Cc^\Cc\to \Cc^\Cc$ będącym składaniem funktorów.

\begin{example}
  Rozważmy parę funktorów sprzężonych znaną z poprzednich wykładów
  \begin{center}
    \begin{tikzcd}
      Set\arrow[r, "F" above] & Ab\arrow[l, "U" below]
    \end{tikzcd}
  \end{center}
  gdzie $F$ to funktor rozpinający wolną grupę abelową o generatorach równych zbiorowi, a $U$ zapomina strukturę grupy. Niech $\eta:1_{Set}\implies UF$ oraz $\epsilon:FY\implies 1_{Ab}$ będą unitem oraz counitem z definicji gunktorów sprzężonych.

  Widzimy tutaj endofunktor $UF$ oraz naturalną transformację $\eta$ jak z definicji monady. Potrzebujemy jeszcze mnożenia na $UF$.

  Naturalne przekształcenie $\epsilon:FU\implies 1_{Ab}$ na dowolnej grupie $A$ jest homomorfizmem ewaluującym formalną sumę jej elementów (obiekt z $FUA$) jako właściwy element grupy $A$. Możemy ten homomorfizm wyrazić jako funkcję, podkładając funktory $U$ i $F$ z odpowiednich stron, tzn. rozważając złożenie
  $$U\epsilon F:UFUF\to UF.$$
  Jest to brakujący w definicji monady sposób mnożenia na funktorze.
\end{example}

\begin{lemma}{}{}
  Każda para $L\vdash R$ funktorów sprzężonych zadaje monadę, gdzie
  \begin{itemize}
    \item $RL$ jest endofunktorem $T$,
    \item unit z definicji pary funktorów sprzężonych $\eta:1_\Cc\to RL$ jest unitem z definicji monady,
    \item counit z nałożonymi funktorami, $R\epsilon L:RLRL\implies RL$ jest mnożeniem $\mu:T^2\to T$.
  \end{itemize}
\end{lemma}

\subsection{Konstruowanie funktorów sprzężonych z monad}
\begin{definition}{}{}
  Niech $\Cc$ będzie kategorią z monadą $(T, \eta, \mu)$. Wtedy \buff{kategorią Kleislego}, oznaczane $\Cc_T$, na $\Cc$ nazwiemy kategorię której
  \begin{itemize}
    \item obiekty są obiektami z $\Cc$ 
    \item morfizmy z $A$ do $B$ w $\Cc_T$, oznaczane $A\rightsquigarrow B$, jest morfizmem $A\to TB$ w kategorii $\Cc$.
  \end{itemize}
\end{definition}

\begin{lemma}{}{}
  to do: łączność w $\Cc_T$ z emily
\end{lemma}

o konstrukcji (trojce) Kleisego





