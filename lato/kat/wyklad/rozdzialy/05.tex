\section{17.03.2025}{Pierwsza nieobecność}

\begin{adjustwidth}{200pt}{0pt}
\begin{flushright}\slshape
  A monad is just a monoid in the category of endofunctors, what's the problem?
\end{flushright}
\end{adjustwidth}

\subsection{Po co właściwie te monady?}

W programowaniu monady są używane do modelowania "robienia czegoś więcej" jako efektu działania funkcji. W OCamlu (autorka notatek dostaje oczopląsu na widok Haskella) jest definiowana jako

\begin{lstlisting}
module type Monad = sig
  type 'a t
  val return : 'a -> 'a t
  val bind : 'a t -> ('a -> 'b t) -> 'b t
end
\end{lstlisting}
Przykładem namacalnej monady jest tzw. monada Maybe, która opakowuje dane w pudełko, tym samym pozwalając zwracać pudełka puste. 

Powiedzmy, że potrzebujemy znaleźć element maksymalny listy, czyli \lstinline{maxElem : int list -> int}. Co, jeśli nasza lista jest pusta? Możemy opakować zwracaną wartość i zmienić ją w \lstinline{int option}. Wtedy w wypadku pustej listy zwracamy \lstinline{None}. 

\begin{lstlisting}
let maxElem (x : int list) : int option = 
  match x with 
  | [] -> None 
  | x::xs -> 
    match maxElem(xs) with 
    | None -> Some x
    | Some y -> Some max(x, y)
\end{lstlisting}

Pojawia się kolejny problem: zmiana zwracanego typu z \lstinline{int} na \lstinline{int option} nie pozwala nam dodawać elementów maksymalnych z różnych list, ani (po napisaniu \lstinline{minElem}) odjąć od elementu maksymalnego elementu minimalnego. Potrzebujemy więc w elegancki sposób zmienić również operacje arytmetyczne. Zacznijmy od zdefiniowania funkcji potrzebnych w monadzie.

\begin{lstlisting}
type 'a t = a' option

let return (x : int) : int option = 
  Some x

let bind (x : int option) (op : int -> int option) : int option = 
  match x with 
  | None -> None
  | Some a -> op a
\end{lstlisting}

Funkcja \lstinline{return} nie robi nic poza opakowaniem \lstinline{int} w \lstinline{int option}, natomiast funkcja \lstinline{bind} wyjmuje \lstinline{int} z pudełka i dopiero wtedy nakłada funkcję i pakuje z powrotem do pudła. Dla przykładu napiszemy tylko nową implementację dodawania, która będzie teraz pobierać dwa argumenty typu \lstinline{int option} i zwracać \lstinline{int option}.

\begin{lstlisting}
let ( + ) : (x : int option) (y : int option) : int option = 
  bind ( x, fun a -> bind(y, fun b -> Some(a+b)) )
\end{lstlisting}

Możemy teraz odpalić
\begin{lstlisting}
maxElem([1; 4; 45]) + maxElem([44; -10; 9])
\end{lstlisting}
i na konsoli zobaczymy \lstinline{Some 69}.

\subsection{Definicja i przykłady}

\begin{definition}{monada}{}
  Monada na kategorii $\Cc$ składa się z
  \begin{itemize}
    \item endofunktora $T:\Cc\to\Cc$,
    \item naturalnej transformacji $\eta:1_\Cc\to T$ (unit z funktorów dołączonych),
    \item naturalnej transformacji $\mu:T^2\to T$, która definiuje mnożenie na funktorze $T$
  \end{itemize}
  takich, że poniższe diagramy komutują w kategorii $\Cc^\Cc$
  \begin{center}
    \begin{tikzcd}[column sep=large, row sep=large]
      T^3\arrow[r, Rightarrow, "T\mu"] \arrow[d, Rightarrow, "\mu T" left] & T^2\arrow[d, Rightarrow, "\mu"] & & T\arrow[r, "\eta T", Rightarrow] \arrow[dr, "1_T" below left, Rightarrow] & T^2 \arrow[d, "\mu" right, Rightarrow] & T\arrow[l, Rightarrow, "T\eta" above]\arrow[dl, "1_T", Rightarrow]  \\
      T^2\arrow[r, Rightarrow, "\mu" below] & T & & & T
    \end{tikzcd}
  \end{center}
\end{definition}

Diagramy te są bardzo podobne do tych, które pojawiły się przy definiowaniu obiektu monoidalnego [\ref{def:obiekt monoidalny}]. Nie jest to przypadkiem: monady są obiektem monoidalnym w kategorii endofunktorów $\Cc^\Cc$ z binarnym działaniem $\Cc^\Cc\times\Cc^\Cc\to \Cc^\Cc$ będącym składaniem funktorów.

\begin{example}[m]
  \item Rozważmy parę funktorów sprzężonych znaną z poprzednich wykładów
  \begin{center}
    \begin{tikzcd}
      Set\arrow[r, "F" above] & Ab\arrow[l, "U" below]
    \end{tikzcd}
  \end{center}
  gdzie $F$ to funktor rozpinający wolną grupę abelową o generatorach równych zbiorowi, a $U$ zapomina strukturę grupy. Niech $\eta:1_{Set}\implies UF$ oraz $\epsilon:FY\implies 1_{Ab}$ będą unitem oraz counitem z definicji gunktorów sprzężonych.

  Widzimy tutaj endofunktor $UF$ oraz naturalną transformację $\eta$ jak z definicji monady. Potrzebujemy jeszcze mnożenia na $UF$.

  Naturalne przekształcenie $\epsilon:FU\implies 1_{Ab}$ na dowolnej grupie $A$ jest homomorfizmem ewaluującym formalną sumę jej elementów (obiekt z $FUA$) jako właściwy element grupy $A$. Możemy ten homomorfizm wyrazić jako funkcję, podkładając funktory $U$ i $F$ z odpowiednich stron, tzn. rozważając złożenie
  $$U\epsilon F:UFUF\to UF.$$
  Jest to występujący w definicji monady sposób mnożenia funktorów.
\item W przykładzie z funkcją \lstinline{maxElem}, endofunktorem $T$ jest zmiana typów \lstinline{int -> int option}. Naturalnym przekształceniem $\eta:1_\Cc\to T$ jest funkcja \lstinline{return}, a funkcja \lstinline{bind} mówi nam jak nałożyć funkcję \lstinline{int -> int option} na element typu \lstinline{int option}, czyli element poddany już działaniu endofunktora $T$.
\item Rozważmy kategorię $Set$ i funktor $T:Set\to Set$, $T(X)=X\cup\{X\}$. Przypomnijmy, że jest to funktor będący złożeniem zapominającego funktora z kategorii zbiorów z wyróżnionym punktem z funktorem do niego dołączonym. $\eta:1_{Set}\to T$ posyła elementy $X$ w elementy $X$, tj. singleton $\{X\}$ nie jest w obrazie. $\mu_X:T^2X\to TX$ pośle elementy $X$ w $X$, a zbiory $\{X\}$ oraz $\{X\cup\{X\}$ w singleton $\{X\}$. Czy widzisz podobieństwo z przykładem wyżej? 
\end{example}

\begin{lemma}{}{}
  Każda para $L\vdash R$ funktorów sprzężonych zadaje monadę, gdzie
  \begin{itemize}
    \item $RL$ jest endofunktorem $T$,
    \item unit z definicji pary funktorów sprzężonych $\eta:1_\Cc\to RL$ jest unitem z definicji monady,
    \item counit z nałożonymi funktorami, $R\epsilon L:RLRL\implies RL$ jest mnożeniem $\mu:T^2\to T$.
  \end{itemize}
\end{lemma}

\subsection{Konstruowanie funktorów sprzężonych z monad}
\begin{definition}{}{}
  Niech $\Cc$ będzie kategorią z monadą $(T, \eta, \mu)$. Wtedy \buff{kategorią Kleislego}, oznaczane $\Cc_T$, na $\Cc$ nazwiemy kategorię której
  \begin{itemize}
    \item obiekty są obiektami z $\Cc$ 
    \item morfizmy z $A$ do $B$ w $\Cc_T$, oznaczane (niekoniecznie konsekwentnie) $A\rightsquigarrow B$, jest morfizmem $A\to TB$ w kategorii $\Cc$.
  \end{itemize}
  Identyczność $id_A:A\rightsquigarrow A$ definiujemy, posiłkując się monadą, jako $\eta_A:A\to TA$. Złożenie morfizmów $f:A\rightsquigarrow B$ oraz $g:B\rightsquigarrow C$ to z kolei
  \begin{center}
    \begin{tikzcd}
      A\arrow[r, "f"] & TB \arrow[r, "Tg"] & T^2C\arrow[r, "\mu_C"] & TC
    \end{tikzcd}
  \end{center}
\end{definition}

\begin{lemma}{}{}
  Składanie morfizmów w kategorii $\Cc_T$ jest łączne.
\end{lemma}

\begin{proof}
  Niech $f:A\rightsquigarrow B$, $g:B\rightsquigarrow C$ oraz $h:C\rightsquigarrow D$ będą morfizmami w kategorii $\Cc_T$. Chcemy pokazać, że $h\circ (g\circ f)=(h\circ g)\circ f$. Z definicji wiemy, że $h\circ g=\mu_D\circ Th\circ g$, ale ponieważ podkładamy pod to $f$, to musimy nałożyć na niego funktor $T$. Mamy diagram 
  \begin{center}
    \begin{tikzcd}[column sep=large]
      A\arrow[r, "f"]\arrow[d, "=", leftrightarrow, sloped] & TB\arrow[r, "Tg"]\arrow[d, "=", leftrightarrow, sloped] & T^2C\arrow[r, "\mu_C", orange]\arrow[d, "=", leftrightarrow, sloped] & TC\arrow[r, "Th", orange]\arrow[d, phantom, "?"] & T^2D\arrow[d, "=", leftrightarrow, sloped] \arrow[r, "\mu_D"] & TD\arrow[d, "=", leftrightarrow, sloped] & h\circ(g\circ f)\\ 
      A\arrow[r, "f" below] & TB\arrow[r, "Tg" below] & T^2C\arrow[r, "T^2h" below, orange] & T^3D\arrow[r, "T\mu_D=\mu_{TD}" below, orange] & T^2D \arrow[r, "\mu_D" below] & TD & (h\circ g)\circ f
    \end{tikzcd}
  \end{center}
  Punktem zapalnym jest $?$ w diagramie. Jeśli ten prostokąt komutuje, to koniec. 

  Z naturalności $\mu:T^2\to T$ dostajemy komutujący diagram
  \begin{center}
    \begin{tikzcd}[column sep=large, row sep=large]
      T^2C\arrow[r, "T^2h"]\arrow[d, "\mu_{C}" left] & T^3D=T^2(TD)\arrow[d, "\mu_{TD}" right]\\ 
      TC\arrow[r, "Th" below]  & T(TD)=T^2D
    \end{tikzcd}
  \end{center}
  czyli 
  $$\mu_{TD}T^2h=TH\mu_C,$$
  co daje nam równość przejścia po pomarańczowych strzałkach na górze (prawa strona równości) i na dole (lewa strona równości).
\end{proof}

\begin{example}
  Dla monady $T:Set\to Set$, $T(X)=X\cup\{X\}$ z przykładów wyżej, kategoria Kleisliego zawiera jako obiekty wszystkie zbiory. Morfizmy $A\rightsquigarrow B$ posyłają część elementów $A$ w "kosmos", czyli singleton $\{B\}$. Są to funkcje częsciowe! Czyli $Set_T=Set^\delta$ jest kategorią zbiorów z funkcjami częsciowymi.
\end{example}

% o konstrukcji (trojce) Kleisego



