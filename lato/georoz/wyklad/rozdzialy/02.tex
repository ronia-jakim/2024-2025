\section{06.03.2025}{}

\begin{definition}{}{}
  Krzywe $\gamma:(a,b)\to \R^3$ sparametryzowane długością łuku, $\gamma''\neq0$. Weźmy $\gamma'\perp \gamma''$. Definiujemy $N=\frac{\gamma''}{|\gamma''|}$. Definiujemy krzywiznę takiej krzywej jako $\kappa_\gamma=|\gamma''|$.

  $B=T\times N$ (wektor, który tworzy z $T$ i $N$ bazę dodatno zorientowaną.

  $(T,N,B)$ to trójnóg Freneta.
\end{definition}

\begin{theorem}{}{}
  $$(T,N,B)'=(T,N,B)\begin{pmatrix}0 & -\kappa & 0\\ 
  \kappa & 0 & -\tau\\ 
0 & \tau & 0\end{pmatrix}$$
\end{theorem}

\begin{proof}
  Pierwszy wiersz to definicja, że $T'=\kappa N$.
\end{proof}




